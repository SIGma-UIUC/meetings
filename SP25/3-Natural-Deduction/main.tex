\documentclass[aspectratio=169]{beamer}

\usepackage{amsmath,amssymb,amsthm}
\usepackage{moredefs,lips}
\usepackage{mathrsfs}
\usepackage{simplebnf}
\usepackage[indent=1em]{fitch}

\renewcommand{\implies}{\Rightarrow}
\newcommand{\emphc}[1]{\textcolor{sigma@mainblue}{\emph{#1}}}
\newcommand{\urlc}[1]{\textcolor{sigma@mainblue}{\url{#1}}}

% Set a title
\title{Natural Deduction}

% Whoever worked on the presentation:
\author{Ethan Zhang}

% A date, if you'd like.
\date{}

% An institute name, if you're so inclined
\institute{University of Illinois Urbana-Champaign}

\usetheme{sigma}

\begin{document}

\begin{frame}
	\titlepage
\end{frame}

\begin{frame}{Outline}
	\tableofcontents
\end{frame}

\section{Logic Refresher}
\frame{\sectionpage}

\begin{frame}{Basic Notions From Logic}
	\begin{definition}
		A \emphc{formula} is a string in a formal language.
		A \emphc{sentence} is a formula with no free variables.
		A \emphc{statement} is a sentence with a definite truth value.
	\end{definition}
	\pause
	\begin{definition}
		An \emphc{argument} is a sequence of statements.
		We say an argument is \emphc{valid} if every statement deductively follows from previous statements,
		and call it \emphc{sound} if it is valid and its premises hold.
	\end{definition}
\end{frame}

\begin{frame}{Propositional (or Zeroth-Order) Logic}
	\only<1-2>{
		\emphc{Propositional Logic} is the most basic form of symbolic logic; its language is inductively defined by atomic propositions and truth functions.
		\pause
		\begin{definition}
			A \emphc{truth function} is a function that takes truth values to truth values.
			For example, a \emphc{logical connective} is a truth function since the truth of a statement $P \land Q$ depends solely on the truth values of the substatements $P$ and $Q$.
		\end{definition}
	}
	\only<3->{
		\centering
		\begin{bnf}
			$\mathscr{A}$ : \textsf{Formulae} ::=
			| $P$ : propositions
			| $\top$ : top (true)
			| $\bot$ : bottom (false)
			| $\mathscr{P} \land \mathscr{Q}$ : conjunction
			| $\mathscr{P} \lor \mathscr{Q}$ : disjunction
			| $\mathscr{P} \implies \mathscr{Q}$ : implication
			| $\neg \mathscr{P}$ : negation
		\end{bnf}
	}
\end{frame}

\begin{frame}{First-Order Logic (FOL)}
	\only<1-2>{
		\emphc{First-Order Logic} extends propositional logic with predicates and quantification over objects.
		\pause
		\begin{definition}
			An \emphc{object} belongs to the \emphc{domain} of a theory, e.g. natural numbers in number theory, sets in set theory, etc.
			A \emphc{predicate} can be thought of like a function on objects: it takes in an object and outputs a truth value depending on said object and the \emphc{interpretation} in the model.
		\end{definition}
	}
	\only<3->{
		\centering
		\begin{bnf}
			$\tau$ : \textsf{Terms} ::=
			| $x$ : variables
			;;
			$\mathscr{A}$ : \textsf{Formulae} ::=
			| $P[\tau_1, \dots, \tau_n]$ : predicates
			| $\top$ : top (true)
			| $\bot$ : bottom (false)
			| $\mathscr{P} \land \mathscr{Q}$ : conjunction
			| $\mathscr{P} \lor \mathscr{Q}$ : disjunction
			| $\mathscr{P} \implies \mathscr{Q}$ : implication
			| $\neg \mathscr{P}$ : negation
			| $\forall x. \mathscr{P}[x]$ : universal quantification
			| $\exists x. \mathscr{P}[x]$ : existential quantification
		\end{bnf}
	}
\end{frame}

\begin{frame}{Intuitionistic vs. Classical Logic}
	\begin{itemize}
		\item \textbf{Truth is objectively real:}
		      Logical sentences have definite truth values, even if we can't (or don't yet know how to) prove them.
		\item \textbf{Truth is a mental construct:}
		      Truth is subjective and dependent on the mind of the mathematician.
		      Hence, a sentence is only true or false when we can provide an explicit construction for an object that witnesses its truth value.
	\end{itemize}
	In particular, \emphc{Classical Logic} admits the \emphc{Law of Excluded Middle} whereas \emphc{Intuitionistic Logic} rejects it.
\end{frame}

\section{A Bit of Proof Theory}
\frame{\sectionpage}

\begin{frame}{Proof Theory?}
	\only<1>{
		\emphc{Proof theory} is the study of ``formal'' proofs as mathematical objects.
		Historically, it stemmed out of a desire to reduce mathematics to a \emphc{syntactical} game governed by axioms and inference rules.
	}
	\only<2>{
		In particular, it studies \emphc{proof systems} and their strengths:
		\begin{itemize}
			\item What is the ``structure'' of a proof?
			\item Can a particular proof system prove (derive) some sentence?
			\item Is there any extra ``meaning''/``insight'' we can get from a particular proof?
		\end{itemize}
		This is in contrast to \emphc{model theory}, which studies the semantics of mathematical theories, i.e. in what contexts are so-and-so sentences true?
	}
\end{frame}

\begin{frame}{Proof Calculi}
	\begin{definition}[from the Pr$\infty$f Wiki]
		A \emphc{proof system} $\mathscr{P}$ for a formal language $\mathscr{L}$ comprises:
		\begin{itemize}
			\item \emphc{Axioms} and/or \emphc{axiom schemata};
			\item \emphc{Rules of inference} for deriving theorems.
		\end{itemize}
	\end{definition}
	\pause
	We write $\Gamma \vdash_{\mathscr{P}} \varphi$ if the formula $\varphi$ is \emphc{provable} (or \emphc{derivable}) in $\mathscr{P}$ from a set of assumptions $\Gamma$;
	that is, if there is some sequence of deductions (governed by $\mathscr{P}$) that starts at $\Gamma$ and ends at $\varphi$.
\end{frame}

\begin{frame}{``Examples''}
	\begin{columns}[t]
		\begin{column}{.5\textwidth}
			\textbf{Hilbert System:}
			\begin{itemize}
				\item Many axioms.
				\item Few inference rules (typically only Modus Ponens).
				\item Reasoning is primarily justified by axioms.
				\item Proofs consist of a sequence of formulae, each of which either is an axiom or follows from a deduction.
			\end{itemize}
		\end{column}
		\begin{column}{.5\textwidth}
			\textbf{Natural Deduction:}
			\begin{itemize}
				\item Few (or no) axioms.
				\item ``Natural'' inference rules.
				\item Proofs are expressed by inference, mimicking the ``natural'' way of reasoning.
				\item \textbf{Fitch-style} proofs look similar to those in a Hilbert system.
				      \textbf{Gentzen-style} proofs make use of ``proof trees'' (\`a la CS 421).
			\end{itemize}
		\end{column}
	\end{columns}
\end{frame}

\begin{frame}{Applications}
	\begin{itemize}
		\item \textbf{Formal verification} of software systems;
		\item \textbf{Proof automation} in mathematics;
		\item \textbf{Logic programming};
		\item \textbf{Semantics} for programming (and natural!) languages.
	\end{itemize}
\end{frame}

\section{(Fitch-Style) Natural Deduction for (Intuitionistic) Propositional Logic}
\frame{\sectionpage}

\begin{frame}{Idea/Motivation}
	\only<1>{
		\begin{itemize}
			\item Truth tables don't give us very much insight into why an argument is true (and don't really work for intuitionistic logics).
			      More importantly, verifying complex arguments becomes intractable!
			\item Arguments involving substituting logical equivalences are hard to come up with, hard to follow, and can hide implicit assumptions.
		\end{itemize}
	}
	\only<2>{
		What if we instead focus on reasoning in the ``natural'' way?

		\pause
		\vspace{1em}
		\begin{quote}
			One of Gentzen’s main motivations was to devise rules that model mathematical reasoning as directly as possible, although clearly in much more detail than in a typical mathematical argument.

			\vspace{1em}\textnormal{--- Frank Pfenning (\textcolor{sigma@mainblue}{2023})}
		\end{quote}
	}
\end{frame}

\begin{frame}{Language}
	\centering
	\begin{bnf}
		$\mathscr{A}$ : \textsf{Formulae} ::=
		| $P$ : propositions
		| $\top$ : top (true)
		| $\bot$ : bottom (false)
		| $\mathscr{P} \land \mathscr{Q}$ : conjunction
		| $\mathscr{P} \lor \mathscr{Q}$ : disjunction
		| $\mathscr{P} \implies \mathscr{Q}$ : implication
		| $\neg \mathscr{P}$ : negation
	\end{bnf}
\end{frame}

\begin{frame}{Reiteration}
	\[
		\begin{nd}
			\have[m] {m} {\mathscr{A}}
			\have[n] {n} {\mathscr{A}} \r{m}
		\end{nd}
	\]
\end{frame}

\begin{frame}{Conjunction}
	\pause
	\begin{columns}
		\begin{column}{.5\textwidth}
			\[
				\begin{nd}
					\have[m] {m} {\mathscr{A}}
					\have[n] {n} {\mathscr{B}}
					\have[l] {l} {\mathscr{A} \land \mathscr{B}} \ai{m,n}
				\end{nd}
			\]
		\end{column}
		\pause
		\begin{column}{.5\textwidth}
			\[
				\begin{nd}
					\have[m] {m} {\mathscr{A} \land \mathscr{B}}
					\have[n] {n} {\mathscr{A}} \ae{m}
				\end{nd}
			\]
			\[
				\begin{nd}
					\have[m] {m} {\mathscr{A} \land \mathscr{B}}
					\have[n] {n} {\mathscr{B}} \ae{m}
				\end{nd}
			\]
		\end{column}
	\end{columns}
\end{frame}

\begin{frame}{Implication}
	\pause
	\begin{columns}
		\begin{column}{.5\textwidth}
			\[
				\begin{nd}
					\open
					\hypo[m] {m} {\mathscr{A}}
					\have[n] {n} {\mathscr{B}}
					\close
					\have[l] {l} {\mathscr{A} \implies \mathscr{B}} \ii{m-n}
				\end{nd}
			\]
		\end{column}
		\pause
		\begin{column}{.5\textwidth}
			\[
				\begin{nd}
					\have[m] {m} {\mathscr{A} \implies \mathscr{B}}
					\have[n] {n} {\mathscr{A}}
					\have[l] {l} {\mathscr{B}} \ie{m, n}
				\end{nd}
			\]
		\end{column}
	\end{columns}
\end{frame}

\begin{frame}{Disjunction}
	\pause
	\begin{columns}
		\begin{column}{.5\textwidth}
			\[
				\begin{nd}
					\have[m] {m} {\mathscr{A}}
					\have[n] {n} {\mathscr{A} \lor \mathscr{B}} \oi{m}
				\end{nd}
			\]
			\[
				\begin{nd}
					\have[m] {m} {\mathscr{B}}
					\have[n] {n} {\mathscr{A} \lor \mathscr{B}} \oi{m}
				\end{nd}
			\]
		\end{column}
		\pause
		\begin{column}{.5\textwidth}
			\[
				\begin{nd}
					\have[m] {m} {\mathscr{A} \lor \mathscr{B}}
					\open
					\hypo[i] {i} {\mathscr{A}}
					\have[j] {j} {\mathscr{C}}
					\close
					\open
					\hypo[k] {k} {\mathscr{B}}
					\have[l] {l} {\mathscr{C}}
					\close
					\have[n] {n} {\mathscr{C}} \oe{m, i-j, k-l}
				\end{nd}
			\]
		\end{column}
	\end{columns}
\end{frame}

\begin{frame}{Negation}
	\pause
	\begin{columns}
		\begin{column}{.5\textwidth}
			\[
				\begin{nd}
					\open
					\hypo[m] {m} {\mathscr{A}}
					\have[n] {n} {\bot}
					\close
					\have[l] {l} {\neg \mathscr{A}} \ni{m-n}
				\end{nd}
			\]
		\end{column}
		\pause
		\begin{column}{.5\textwidth}
			\[
				\begin{nd}
					\have[m] {m} {\neg \mathscr{A}}
					\have[n] {n} {\mathscr{A}}
					\have[l] {l} {\bot} \ne{m, n}
				\end{nd}
			\]
		\end{column}
	\end{columns}
\end{frame}

\begin{frame}{Explosion}
	\[
		\begin{nd}
			\have[m] {m} {\bot}
			\have[n] {n} {\mathscr{A}} \be{m}
		\end{nd}
	\]
\end{frame}

\subsection{Examples}
\frame{\subsectionpage}

\begin{frame}{$A \implies (B \implies C) \vdash (A \land B) \implies C$}
	\[
		\begin{nd}
			\hypo {h} {A \implies (B \implies C)}
			\open
			\hypo {hab} {A \land B}
			\have {a} {A} \ae{hab}
			\have {b} {B} \ae{hab}
			\have {ib} {B \implies C} \ie{h, a}
			\have {c} {C} \ie{ib, b}
			\close
			\have {r} {(A \land B) \implies C} \ii{hab-c}
		\end{nd}
	\]
\end{frame}

\begin{frame}{$P \land (Q \lor R), P \implies \neg R \vdash Q \lor E$}
	\begin{columns}
		\begin{column}{.4\textwidth}
			\[
				\begin{nd}
					\hypo{hpqr} {P \land (Q \lor R)}
					\hypo{hpnr} {P \implies \neg R}
					\have{p} {P} \ae{hpqr}
					\have{qr} {Q \lor R} \ae{hpqr}
					\have{nr} {\neg R} \ie{hpnr, p}
				\end{nd}
			\]
		\end{column}
		\begin{column}{.6\textwidth}
			\[
				\begin{nd}
					\open
					\hypo[6] {hq} {Q}
					\have[7] {qe1} {Q \lor E} \oi{hq}
					\close
					\open
					\hypo[8] {hr} {R}
					\have[9] {b} {\bot} \ne{nr, hr}
					\have[10] {qe2} {Q \lor E} \be{b}
					\close
					\have[11] {qe} {Q \lor E} \oe{qr, hq-qe1, hr-qe2}
				\end{nd}
			\]
		\end{column}
	\end{columns}
\end{frame}

\section{Extending Natural Deduction to FOL}
\frame{\sectionpage}

\begin{frame}{Language}
	\centering
	\begin{bnf}
		$\tau$ : \textsf{Terms} ::=
		| $x$ : variables
		;;
		$\mathscr{A}$ : \textsf{Formulae} ::=
		| $P[\tau_1, \dots, \tau_n]$ : predicates
		| $\top$ : top (true)
		| $\bot$ : bottom (false)
		| $\mathscr{P} \land \mathscr{Q}$ : conjunction
		| $\mathscr{P} \lor \mathscr{Q}$ : disjunction
		| $\mathscr{P} \implies \mathscr{Q}$ : implication
		| $\neg \mathscr{P}$ : negation
		| $\forall x. \mathscr{P}[x]$ : universal quantification
		| $\exists x. \mathscr{P}[x]$ : existential quantification
	\end{bnf}
\end{frame}

\begin{frame}{Universal Quantifier}
	\pause
	\begin{columns}
		\begin{column}{.5\textwidth}
			\[
				\begin{nd}
					\open
					\hypo[m] {m} {a}
					\have[n] {n} {\mathscr{P}[a/x]}
					\close
					\have[l] {l} {\forall x. \mathscr{P}[x]} \Ai{m-n}
				\end{nd}
			\]
		\end{column}
		\pause
		\begin{column}{.5\textwidth}
			\[
				\begin{nd}
					\have[m] {m} {\forall x. \mathscr{P}[x]}
					\have[n] {n} {\mathscr{P}[a/x]} \Ae{m}
				\end{nd}
			\]
		\end{column}
	\end{columns}
\end{frame}

\begin{frame}{Existential Quantifier}
	\pause
	\begin{columns}
		\begin{column}{.5\textwidth}
			\[
				\begin{nd}
					\have[m] {m} {\mathscr{P}[a]}
					\have[n] {n} {\exists x. \mathscr{P}[x/a]} \Ei{m}
				\end{nd}
			\]
		\end{column}
		\pause
		\begin{column}{.5\textwidth}
			\[
				\begin{nd}
					\have[m] {m} {\exists x. \mathscr{P}[x]}
					\open
					\hypo[i] {i} {\mathscr{P}[a/x]}
					\have[j] {j} {\mathscr{A}}
					\close
					\have[n] {n} {\mathscr{A}} \Ee{m, i-j}
				\end{nd}
			\]
		\end{column}
	\end{columns}
\end{frame}

\subsection{Examples}
\frame{\subsectionpage}

\begin{frame}{$\forall x. Fx \implies Gx, \exists x. Fx \vdash \exists x.Gx$}
	\[
		\begin{nd}
			\hypo {h1} {\forall x. Fx \implies Gx}
			\hypo {h2} {\exists x. Fx}
			\open
			\hypo {hf} {Fa}
			\have {faga} {Fa \implies Ga} \Ae{h1}
			\have {ga} {Ga} \ie{faga, hf}
			\have {Exgx1} {\exists x. Gx} \Ei{ga}
			\close
			\have {Exgx} {\exists x. Gx} \Ee{h2, hf-ga}
		\end{nd}
	\]
\end{frame}

\section{``Upgrading'' to Classical Logic}
\frame{\sectionpage}

\begin{frame}{Excluded Middle (and Related Axioms)}
	\begin{columns}
		\begin{column}{.5\textwidth}
			\[
				\begin{nd}
					\open
					\hypo[i] {i} {\mathscr{A}}
					\have[j] {j} {\mathscr{B}}
					\close
					\open
					\hypo[k] {k} {\neg\mathscr{A}}
					\have[l] {l} {\mathscr{B}}
					\close
					\have[m] {m} {\mathscr{B}} \by{LEM}{i-j, k-l}
				\end{nd}
			\]
		\end{column}
		\begin{column}{.5\textwidth}
			\[
				\begin{nd}
					\have[m] {m} {\neg \neg \mathscr{A}}
					\have[n] {n} {\mathscr{A}} \by{DNE}{m}
				\end{nd}
			\]
			\[
				\begin{nd}
					\open
					\hypo[m] {m} {\neg \mathcal{A}}
					\have[n] {n} {\bot}
					\close
					\have[l] {l} {\mathcal{A}} \by{IP}{m-n}
				\end{nd}
			\]
		\end{column}
	\end{columns}
\end{frame}

\section{Exercises}
\frame{\sectionpage}

\begin{frame}{$\text{LEM} \vdash \text{DNE}$}
	\pause
	\[
		\begin{nd}
			\hypo {h} {\neg \neg P}
			\open
			\hypo {hp} {P}
			\have {p1} {P} \r{hp}
			\close
			\open
			\hypo {hnp} {\neg P}
			\have {b} {\bot} \ne{h, hnp}
			\have {p2} {P} \be{b}
			\close
			\have {p} {P} \by{LEM}{hp-p1, hnp-p2}
		\end{nd}
	\]
\end{frame}

\begin{frame}{$\text{DNE} \vdash \text{IP}$}
	\pause
	\[
		\begin{nd}
			\hypo {h} {\neg P \implies \bot}
			\open
			\hypo {hnp} {\neg P}
			\have {b} {\bot} \ie{h, hnp}
			\close
			\have {nnp} {\neg \neg P} \ni{hnp-b}
			\have {p} {P} \by{DNE}{nnp}
		\end{nd}
	\]
\end{frame}

\begin{frame}{$\text{IP} \vdash \text{LEM}$}
	\pause
	\[
		\begin{nd}
			\open
			\hypo {h} {\neg(P \lor \neg P)}
			\open
			\hypo {hnp} {\neg P}
			\have {pnp1} {P \lor \neg P} \oi{hnp}
			\have {b1} {\bot} \ne{h, pnp1}
			\close
			\have {p} {P} \by{IP}{hnp-b1}
			\have {pnp2} {P \lor \neg P} \oi{p}
			\have {b2} {\bot} \ne{h, pnp2}
			\close
			\have {pnp} {P \lor \neg P} \by{IP}{h-b2}
		\end{nd}
	\]
\end{frame}

\begin{frame}{$\forall x. \neg Mx \lor Ljx, \forall x. Bx \implies Ljx, \forall x. Mx \lor Bx \vdash \forall x. Ljx$}
	\pause
	\begin{columns}
		\begin{column}{.4\textwidth}
			\[
				\begin{nd}
					\hypo {h1} {\forall x. \neg Mx \lor Ljx}
					\hypo {h2} {\forall x. Bx \implies Ljx}
					\hypo {h3} {\forall x. Mx \lor Bx}
					\have {nMALja} {\neg Ma \lor Lja} \Ae{h1}
					\have {BaLja} {Ba \implies Lja} \Ae{h2}
					\have {MaBa} {Ma \lor Ba} \Ae{h3}
					\open
					\hypo {hLja} {Lja}
					\have {lja4} {Lja} \r{hLja}
					\close
				\end{nd}
			\]
		\end{column}
		\begin{column}{.6\textwidth}
			\[
				\begin{nd}
					\open
					\hypo[9] {hnMa} {\neg Ma}
					\open
					\hypo[10] {hMa} {Ma}
					\have[11] {b} {\bot} \ne{hnMa, hMa}
					\have[12] {Lja1} {Lja} \be{b}
					\close
					\open
					\hypo[13] {hBa} {Ba}
					\have[14] {Lja2} {Lja} \ie{BaLja, hBa}
					\close
					\have[15] {Lja3} {Lja} \oe{h3, hMa-Lja1, hBa-Lja2}
					\close
					\have[16] {Lja} {Lja} \oe{h1, hnMa-Lja3, hLja-lja4}
					\have[17] {AxLjx} {\forall x. Ljx} \Ai{Lja}
				\end{nd}
			\]
		\end{column}
	\end{columns}
\end{frame}

\begin{frame}{}
	\begin{center}
		{\color{sigma@mainblue} \LARGE Questions?}
	\end{center}
\end{frame}

\begin{frame}
	\begin{quote}
		There's this thing which I like to call the \lips ``Holy Trinity'' here of computer science[:] \lips
		the correspondences between proof theory,
		which is \lips the theory of logic and proofs;
		algebra and category theory;
		and then the subject in the computer science side[, which are] programs, or type theory.

		\vspace{1em}
		\textnormal{--- Robert Harper (\textcolor{sigma@mainblue}{2012})}
	\end{quote}
\end{frame}

% Remove this slide if you came up with all the material yourself
\begin{frame}{Bibliography}
	\begin{itemize}
		\item Open Logic Project. 2023. \emph{forall x: Calgary}. \urlc{https://forallx.openlogicproject.org/}.
		\item Frank Pfenning. 2023. \emph{Lecture Notes on What is (Constructive) Logic?}. \urlc{https://www.cs.cmu.edu/~fp/courses/15317-s23/lectures/01-whatis.pdf}.
		\item Samuel Buss. 1998. \emph{An Introduction to Proof Theory}. \urlc{https://mathweb.ucsd.edu/~sbuss/ResearchWeb/handbookI/ChapterI.pdf}.
		\item Pr$\infty$f Wiki. 2024. \emph{Pr$\infty$f Wiki}. \urlc{https://proofwiki.org/wiki/Definition:Proof_System}.
	\end{itemize}
\end{frame}

\end{document}
