% Mirror: https://github.com/SIGma-UIUC/presentation-format
% --------------------------------------------------------------------
% This is a simple Beamer document that uses beamerthemesigma.sty
% Reading the comments should help you create a presentation even if
% you've never used Beamer before.
% --------------------------------------------------------------------

% Set our document class to Beamer
\documentclass[aspectratio=169]{beamer}
% \documentclass[aspectratio=169, handout]{beamer}
% Add handout option to ignore pauses

% From Jeff E
\usepackage{algo}
% Some more macros
\usepackage{sigmastyle}


% Set a title
\title{Doing Math with Computers for Fun and Profit}

% Set a subtitle if you desire
\subtitle{\cite{classifying}}

% Whoever worked on the presentation:
\author{Anakin}

% Date looks ugly, so leave blank
\date{}

% An institute name, if you're so inclined
% \institute{University of Illinois Urbana-Champaign}

% Use the SIGma theme for this Beamer presentation
\usetheme{sigma}
% --------------------------------------------------------------------

% Begin document
\begin{document}

% Beamer calls each slide a "frame", defined within the environment:
% \begin{frame}
%   <frame content here>
% \end{frame}

% This frame is just the title.
\begin{frame}
\titlepage
\end{frame}

\section{REU}
\frame{\sectionpage}

\begin{frame}{What is an REU?}
    \begin{itemize}
        \item Research Experience for Undergraduates \pause
        \item Get paid to do research in Math, CS, Engineering, Science, etc., over the summer \pause
        \item See how other schools do things, meet new people \pause
        \item Maybe even get a paper out of it!
    \end{itemize}
\end{frame}

\begin{frame}{How do I find them?}
    Save these slides for later!
    \begin{itemize}
        \item \href{https://www.nsf.gov/crssprgm/reu/reu_search.jsp}{NSF List}
        \item \href{https://www.mathprograms.org/db?joblist-0-0----p--}{Math Programs}
        \item \href{https://docs.google.com/spreadsheets/d/1U-27BeHMSJCWumbNByal2tHyYo9wRVud9WoRE70E47Y/edit\#gid=1262276228}{This spreadsheet}
    \end{itemize}
\end{frame}

\begin{frame}{How do I apply to them?}
    \begin{itemize}
        \item Personal Statement \pause
        \item 2 Letters of Rec \pause
        \item Resume / CV \pause
        \item Most deadlines are early -- mid March
        \begin{itemize}
            \item Start drafting in Winter Break
        \end{itemize}
    \end{itemize}
\end{frame}

\begin{frame}{Tips \& Tricks}
    \begin{itemize}
        \item Most are government funded which means usually US citizens get funding.
        \begin{itemize}
            \item It is possible for non-US citizens to get funding in some special cases.
        \end{itemize} \pause
        \item Get your letter writers to read your personal statement.
        \item There are other options outside of REUs (\href{https://www.microsoft.com/en-us/research/}{MSR}, \href{https://www.ttic.edu/}{TTIC}, \href{https://summer.epfl.ch/}{EPFL}, \href{https://inf.ethz.ch/studies/summer-research-fellowship.html}{ETH Z{\"u}rich}, \href{https://www.cis.mpg.de/internships/}{Max Planck}, \href{https://idaccr.org/scamp}{SCAMP}, Independent Study, etc).
        \item Most people don't apply to enough REUs.
    \end{itemize} \pause
    \textcolor{sigma@alertred}{Any other questions?}
\end{frame}

\begin{frame}{}
      \begin{center}
    {\color{sigma@mainblue} \LARGE Questions?}
  \end{center}
\end{frame}

\section{Introduction to Group Theory}
\frame{\sectionpage}

\begin{frame}{Groups and Group Actions}
    \begin{itemize}
        \item A \emph{group} is an object in the category of groups
        \item A \emph{group action} is a functor from a $1$-groupoid to the category of sets
    \end{itemize}
\end{frame}

\begin{frame}{What is a Group?}
    Groups are one of the most ubiquitous objects in all of math. They generalize structures with some sort of addition/multiplication. \pause

    \begin{defn}
        A \emph{group} is a set $G$ with an operation $\_ \cdot \_ \colon G \times G \to G$ such that
        \begin{itemize}
            \item $\cdot$ is associative: $(x \cdot y) \cdot z = x \cdot (y \cdot z)$ \pause
            \item There exists an identity $e$ such that $g \cdot e = g = e \cdot g$ \pause
            \item Every element $g$ has an inverse $g^{-1}$ such that $g \cdot g^{-1} = e = g^{-1} \cdot g$
        \end{itemize}
    We will usually just write $x \cdot y$ as $xy$
        
    \end{defn}
\end{frame}

\begin{frame}{Important Examples of Groups}
    Consider the set $S_n$ of bijections $\sigma\colon [n] \to [n]$ where $[n] = \set{1, \ldots, n}$. 
    This forms a group with ``multiplication'' using composition\pause
    \begin{itemize}
        \item Composing bijections with each other yields a bijection
        \item Identity: \pause $\id(i) = i$ for all $1 \leq i \leq n$ \pause
        \item Inverses: \pause $\sigma$ has an inverse $\sigma$ such that $\sigma \circ \sigma^{-1} = \id$
    \end{itemize}
\end{frame}

\begin{frame}{Important Examples of Groups}
    Recall that the integers mod $p$ are $ \Z_{p} = \set{1, 2, \ldots, p}$ with addition and multiplication done modulo $p$.
    Consider the set $\GL(n, p)$ of all $n \times n$ matrices with entries in $\Z_p$ with non-zero determinant. 
    This forms a group with matrix multiplication \pause
    \begin{itemize}
        \item Multiplying two matrices with non-zero determinant yields a matrix with non-zero determinant since $\det(AB) = \det(A) \cdot \det(B)$
        \item Identity: \pause $I_n$ with $1$s on the diagonal and $0$s elsewhere
        \item Inverses: \pause Matrices have an inverse if and only if they have non-zero determinant, so each $A$ has an inverse $A^{-1}$ such that $A \times A^{-1} = I_n$
    \end{itemize}
    If you've taken Linear Algebra, these are just invertible linear transformations!
\end{frame}

\begin{frame}{Group Isomorphism}
    Consider the following two groups:
    \[
       G = \Bigg\{  { \id = \begin{pmatrix} 1 & 2 & 3 \\ 1 & 2 & 3 \end{pmatrix}}, \sigma =  {\begin{pmatrix} 1 & 2 & 3 \\ 2 & 3 & 1 \end{pmatrix}},  { \sigma^2 = \begin{pmatrix} 1 & 2 & 3 \\ 3 & 1 & 2 \end{pmatrix}}  \Bigg\}  \subseteq S_3
    \]
    \[
     \hspace{-30pt}  \Z_3 = \{ \hspace{45pt} 0, \hspace{70pt} 1, \hspace{75pt} 2 \hspace{20pt} \}  
    \]
    So $G$ are some permutations with the operation of composition and $\Z_3$ are integers modulo $3$ where the operation is addition but we keep the remainder after division by $3$. So $2 + 2 \equiv 1 \pmod{3}$.
    
    \textcolor{sigma@alertred}{Question:} In what sense are these two groups the same group? \pause

    \textcolor{sigma@mainblue}{Answer:} Mapping $\id \mapsto 0$, $\sigma \mapsto 1$, $\sigma^2 \mapsto 2$ preserve operations! \pause
    
    Notice that $\sigma \circ \sigma = \sigma^2$ and $1 + 1 \equiv 2 \pmod{3}$. \pause
    
    Similarly, $\sigma \circ \sigma \circ \sigma = \id$ and $1 + 1 + 1 \equiv 0 \pmod{3}$.
\end{frame}

\begin{frame}{Group Actions}
    We want to study how a group $G$ interacts with other sets. \pause
    Let $\Omega$ be some set.
    
    \begin{defn}
        Then a \emph{group action} is an operation $\_ \cdot \_ \colon G \times \Omega \to \Omega$ such that \pause
        \begin{itemize}
            \item $e\cdot x = x$, for all $x\in \Omega$ \pause
            \item $g\cdot(h\cdot x) = (gh)\cdot x$, for all $g,h\in G$, and for all $x\in \Omega$
        \end{itemize}
    We write $G \acts \Omega$.
    \end{defn}
    
    To prevent confusion with the group operation in $G$, we will keep the $\cdot$ when talking about actions.
\end{frame}

\begin{frame}{Important Examples of Group Actions}
    Let $G = S_n$ and $\Omega = [n]$. \pause
    \begin{itemize}
        \item Say $\sigma\colon [n] \to [n] \in S_n$ and $i \in [n]$. 
            What would be a good choice of action $\sigma \cdot i$? \pause
        \item $\sigma \cdot i \defeq \sigma(i)$ \pause
    \end{itemize}

    Now let $G$ be a group of invertible linear transformations from a vector space $V \to V$. \pause
    \begin{itemize}
        \item Say $T\colon V \to V \in G$ and $v \in V$. 
            What would be a good choice of action $T \cdot v$? \pause
        \item $T \cdot v \defeq T(v)$
    \end{itemize}
    
\end{frame}

\begin{frame}{Orbits and Stabilizers}
    We want to study the structure of $G \acts \Omega$.

    \begin{defn}
        The \emph{orbit} of $\alpha \in \Omega$ is the set $G \cdot \alpha = \Set{g \cdot \alpha | g \in G}$
    \end{defn} \pause

    Every element of $\Omega$ belongs in some orbit.
    It turns out the orbits partition $\Omega$. \pause

    \begin{defn}
        The \emph{stabilizer} of $\alpha \in \Omega$ is the set $G_\alpha = \Set{g \in G | g \cdot \alpha = \alpha}$
    \end{defn} \pause

    \textcolor{sigma@alertred}{Exercise:} Stabilizers are subgroups of $G$
\end{frame}

\begin{frame}{Example}
    \[
       G = \Bigg\{  { \id = \begin{pmatrix} 1 & 2 & 3 & 4 \\ 1 & 2 & 3 & 4 \end{pmatrix}}, \sigma =  {\begin{pmatrix} 1 & 2 & 3 & 4 \\ 2 & 3 & 1 & 4 \end{pmatrix}},  { \sigma^2 = \begin{pmatrix} 1 & 2 & 3 & 4 \\ 3 & 1 & 2 & 4 \end{pmatrix}}  \Bigg\}  \subseteq S_4
    \]
    This is the same group as before, but now $4$ is a valid input and we don't do anything to it.
    \textcolor{sigma@alertred}{Exercise:} Check that this is a subgroup of $S_4$.

    Consider $G \acts [4]$.

    \begin{itemize}
        \item $G \cdot 1 = $ \pause $\set{1, 2, 3}$ \pause
        \item $G \cdot 4 = $ \pause $\set{4}$       \pause
        \item $G_1 = $       \pause $\set{\id}$     \pause
        \item $G_4 = $       \pause $G$
    \end{itemize}
\end{frame}

\begin{frame}{Group Classification}
    \begin{center}
        The \emph{Group Classification Problem} is the problem of identifying groups satisfying some property ``up to'' isomorphism.
    \end{center} \pause

    \vspace{40pt}

    \begin{itemize}
        \item This is one of the hardest problems in all of group theory. \pause
        \item Even checking if two finite groups are isomorphic is difficult for computer. \pause
        \item The classification of the finite simple groups took tens of thousands of pages written by over 100 authors between 1955 and 2004.
    \end{itemize}
\end{frame}

\begin{frame}{Rank}
    \begin{itemize}
        \item Let $G$ be some group of permutations in $S_n$ and consider $G \acts [n]$ such that $G$ only has one orbit.
        \item Let $G_0$ be the stabilizer of some element of $\Omega$. 
        It turns out it doesn't matter which one.
    \end{itemize} \pause

    \begin{defn}
        The \emph{rank} of $G$ is the number of orbits of $G_0 \acts [n]$.
    \end{defn}

    This is a sort of measurement of the ``reach'' of stabilizer subgroups of $G$.
\end{frame}

\section{Groups, Algorithms, and Programming}
\frame{\sectionpage}

\begin{frame}{Structure}
    \begin{itemize}
        \item Let $G$ be some group of permutations in $S_n$ and consider $G \acts [n]$ such that $G$ only has one orbit.
        \item Let $G_0$ be the stabilizer of some element of $\Omega$.  \pause
        \item \textcolor{sigma@mainblue}{Result 1:} This is the same as considering $G_0 \acts V$ where $G_0$ is now a group of linear transformations and $V$ is a vector space $\F_p^k$. 
        \item \# Orbits of $G_0 \acts [n]$ = \# Orbits of $G_0 \acts \F_p^k$. \pause 
        \item \textcolor{sigma@mainblue}{Result 2:} $G_0$ must contain a certain subgroup $E$ of order $q^{2m + 1}$.
    \end{itemize}
\end{frame}

\begin{frame}{Making Change using Group Theory}
    So now we can consider $G_0 \acts \F_p^k$ and $E \subseteq G_0$ $\abs{E} = q^{2m + 1}$.
    This gives us a nice set of parameters.

    \begin{enumerate}
        \item We find a value $B(p, k, q, m)$ such that $\abs{G_0}$ divides $B$ \pause
        \item A theorem in group theory tells us that the size of orbits of $G_0$ divides $\abs{G_0}$, so they divide $B$
        \begin{itemize}
            \item Let $d_1, \ldots, d_t$ be the divisors of $B$
        \end{itemize}
        \item We know there is one orbit of size $1$ and the sizes of the other orbits must sum up to $p^k - 1$ \pause
    \end{enumerate}

    \textcolor{sigma@mainblue}{Result 3: } We can get a lower bound on rank by solving the \emph{Change Making Problem} with coins $d_1, \ldots, d_t$ and target value $p^k - 1$.    
\end{frame}

\begin{frame}{Making Change using Group Theory}
    In the \emph{Change-making Problem}, we are given coins from some set of denominations $d_1, \ldots, d_t$ and a target value $T$, we want to ``make change'' for $T$ using as few coins as possible

    \vspace{20pt}
    
    \begin{itemize}
        \item We have a fixed set of possible sizes of orbits and a target value $p^k - 1$ \pause
        \item We know the orbits partition this target value \pause
        \item A worst case lower bound is the most efficient packing as possible
        \item Thus we want to solve the Change Making Problem with coins $d_1, \ldots, d_t$ and target value $p^k - 1$.    
    \end{itemize}
\end{frame}

\begin{frame}{Inductively Making Change}
    Let \emph{coins} $= [d_1, \ldots, d_n]$ be a sorted list of denominations of coins.
    Let \textsc{NumCoins}$(i, c)$ be the smallest possible number coins of denomination $[d_1, \ldots, d_c]$ needed make change for $i$
    \begin{itemize}
        \item If \emph{coins} = $[1, 3, 5, 7]$ then \textsc{NumCoins}$(10, 1) = 10$ but \textsc{NumCoins}$(10, 4) = 2$
    \end{itemize}
\end{frame}

\begin{frame}{Inductively Making Change}
    Let \emph{coins} $= [d_1, \ldots, d_n]$ be a sorted list of denominations of coins.
    Let \textsc{NumCoins}$(i, c)$ be the smallest possible number coins of denomination $[d_1, \ldots, d_c]$ needed make change for $i$. 
    \vspace{10pt}
    
    $\textsc{NumCoins}(i, c) = $
    \[
        \begin{cases}
            \infty & c = 0 \\
            \textsc{NumCoins}(i, c - 1) & i < \textit{coins}[c] \\
            1 & i = \textit{coins}[c] \\
            \min\set{\textsc{NumCoins}(i, c - 1), 1 + \textsc{NumCoins}(i - \textit{coins}[c], c)} & \text{otherwise}
        \end{cases}
    \]
\end{frame}



\begin{frame}{A (very high level) Overview of the Whole Paper}
    \begin{enumerate}
        \item Define the parameters $p, k, q, m$ \pause
        \item Do a bunch of pure math to get finite bounds on these parameters
        \item Enumerate all possible sets of parameters and keep the ones that have a lower bound $\leq 6$
    \end{enumerate}
\end{frame}

\begin{frame}{A (very high level) Overview of the Whole Paper}
    For each set of valid parameters $p, k, q, m$.
    Let $N =$ the largest possible $N$ such that $N \acts \F_p^k$ \pause
    \begin{enumerate}
        \item Check if the subgroup $E$ with $\abs{E} = q^{2m + 1}$ is contained in $N$ \textcolor{sigma@alertred}{(HARD!)} \pause
        \item Check if $N$ has rank $\leq 6$ \pause
        \item Enumerate all possible subgroups of $N$ \textcolor{sigma@alertred}{(HARD!)} \pause
        \item Repeat for each subgroup
    \end{enumerate}
\end{frame}

\begin{frame}{How?}
    \begin{itemize}
        \item All of this was done in a programming language called \textcolor{sigma@mainblue}{\textsc{GAP}}: Groups, Algorithms, and Programming \pause
        \item \textsc{GAP} is just one of many computational algebra systems
        \begin{itemize}
            \item SageMath (Built on top of Python
            \item Mathmatica
            \item Magma (popular in Cryptography)
            \item Macauley2 (Created at UIUC!)
        \end{itemize} \pause
        \item Hard computations were done on AWS.
        \item These techniques extend to higher ranks but computational resources are a large issue.
    \end{itemize}
\end{frame}

\begin{frame}{More Details?}
    \begin{itemize}
        \item Check out the paper (linked on my website \href{https://www.anakin-dey.com/}{\ul{anakin-dey.com}})
        \item Come to the Undergraduate Math Seminar (details coming soon)
        \item Ask me in the Discord!
    \end{itemize}
\end{frame}

\begin{frame}{}
      \begin{center}
    {\color{sigma@mainblue} \LARGE Questions?}
  \end{center}
\end{frame}

% Quotes are fun, find some to use!
\font\eightss=cmssq8
\font\eightssi=cmssqi8
\newcommand\quoteAuthorDate[3]{\begingroup
  \baselineskip 10pt
  \parfillskip 0pt
  \interlinepenalty 10000 % not needed in example
  \leftskip 0pt plus 40pc minus \parindent
  \let\rm=\eightss
  \let\sl=\eightssi
  \everypar{\sl}#1\par
  \nobreak\smallskip
  \noindent\rm--- #2\unskip\enspace(#3)\par
  \endgroup}
% If someone can figure out how to horizontally center this and make the text bigger that'd be cool
\begin{frame}
    \begin{center}
        \item \quoteAuthorDate{So long and thanks for all the fish!}{DOUGLAS ADAMS}{\color{sigma@mainblue} 1979}
    \end{center}
\end{frame}

% Remove this slide if you came up with all the material yourself
\begin{frame}{Bibliography}
    \bibliography{refs}
    \bibliographystyle{alpha}
\end{frame}

\end{document}