% Mirror: https://github.com/SIGma-UIUC/presentation-format
% --------------------------------------------------------------------
% This is a simple Beamer document that uses beamerthemesigma.sty
% Reading the comments should help you create a presentation even if
% you've never used Beamer before.
% --------------------------------------------------------------------

% Set our document class to Beamer
\documentclass[aspectratio=169, handout]{beamer}
% \documentclass[aspectratio=169, handout]{beamer}
% Add handout option to ignore pauses



% From Jeff E
\usepackage{algo}
% Some more macros
\usepackage{sigmastyle}
\usepackage{minted}

\usepackage{colortbl}
\definecolor{LightRed}{RGB}{252,160,140}
\newcolumntype{a}{>{\columncolor{LightRed}}c}

% Set a title
\title{Hamming Codes}

% Set a subtitle if you desire
% \subtitle{[TAOCP 5 8.9.10.11]}

% Whoever worked on the presentation:
\author{Sam Ruggerio}

% Date looks ugly, so leave blank
\date{}

% An institute name, if you're so inclined
% \institute{University of Illinois Urbana-Champaign}

% Use the SIGma theme for this Beamer presentation
\usetheme{sigma}
% --------------------------------------------------------------------

\newcommand{\err}[1]{{\color{sigma@alertred}#1}}
\newcommand{\blu}[1]{{\color{sigma@mainblue}#1}}
% Begin document
\begin{document}

% Beamer calls each slide a "frame", defined within the environment:
% \begin{frame}
%   <frame content here>
% \end{frame}

% This frame is just the title.
\begin{frame}
\titlepage
\end{frame}

% A frame with the table of contents.
% This frame's title is "Outline".
\begin{frame}{Outline}
  \tableofcontents
\end{frame}

% Start a section: *sections* (subsections, etc.) are what show up in the TOC.
\section{Review}
% Section pages can be printed thus:
\frame{\sectionpage}
% There's a way to automate this, see:
% https://tex.stackexchange.com/questions/178800/creating-sections-each-with-title-pages-in-beamers-slides/178803
\begin{frame}{Error Correction and Detection}
    \begin{itemize}
        \item Sending messages between parties might be subject to noise and errors. \pause
        \item How do we: \pause
            \begin{itemize}
                \item Detect Errors? \pause
                \item Correct Errors? \pause
            \end{itemize}
        \item The goal is to send a message that minimizes the necessary redundancy to achieve both goals.
    \end{itemize}
\end{frame}

\begin{frame}{Error Correction and Detection}
    \begin{itemize}
        \item Easy solution: send $k$ bits for every bit. Use majority voting to determine true bit. \pause
        \item For example, the message $1011$:
        \begin{itemize}
            \item I send: \ \ \ \ \ \ $11111000001111111111$ (i.e. $k=5$) \pause
            \item You receive $11\err{0}1100\err{11}0\err{0}111111111$ \pause
            \item You determine the correct message by voting $1011$ \pause
        \end{itemize}
        \item Recall the \emph{rate} of a code is $\frac{\text{\textit{len}(original message)}}{\text{\textit{len}(code word)}}$
        \item Good robustness, but our rate is 20\% (4 extra bits for every 1 bit of information).
    \end{itemize}
\end{frame}

\section{Hamming Codes}

\frame{\sectionpage}

\begin{frame}{Better Codes}
    \begin{itemize}
        \item Can we do better? \pause
        \item \emph{Claim:} For 11 bits, we can correct 1 error, and detect 2 errors, using only 5 bits, to make ``nice'' 16 bit blocks  \pause
        \begin{itemize}
            \item 11/16 = 68.75\% rate!
        \end{itemize}
    \end{itemize}
\end{frame}

\begin{frame}{Block Messages}
    \begin{columns}[c]
    \begin{column}{5cm}
    \begin{table}
        \centering
        \begin{tabular}{|c|c|c|c|}
            \hline 
            0 & 0 & 0 & 0 \\ \hline
            0 & 0 & 0 & 0 \\ \hline
            0 & 0 & 0 & 0 \\ \hline
            0 & 0 & 0 & 0 \\ \hline
        \end{tabular}
    \end{table}
    \end{column}
    \hfill
    \begin{column}{5cm}
        \begin{itemize}
            \item Consider a 16-bit block
        \end{itemize}
    \end{column}
    \end{columns}
\end{frame}

\begin{frame}{Block Messages}
    \begin{columns}[c]
    \begin{column}{5cm}
    \begin{table}
        \centering
        \begin{tabular}{|c|c|c|c|}
            \hline 
            0 & 0 & 0 & \blu{0} \\ \hline
            0 & \blu{1} & \blu{1} & \blu{0} \\ \hline
            0 & \blu{1} & \blu{0} & \blu{0} \\ \hline
            \blu{1} & \blu{0} & \blu{1} & \blu{1} \\ \hline
        \end{tabular}
    \end{table}
    \end{column}
    \hfill
    \begin{column}{5cm}
        \begin{itemize}
            \item Consider a 16-bit block
            \item We want to fill it with the message 01101001011
        \end{itemize}
    \end{column}
    \end{columns}
\end{frame}

\begin{frame}{Block Messages}
    \begin{columns}[c]
    \begin{column}{5cm}
    \begin{table}
        \centering
        \begin{tabular}{|c|c|c|c|}
            \hline 
            \err{0} & \err{0} & \err{0} & \blu{0} \\ \hline
            \err{0} & \blu{1} & \blu{1} & \blu{0} \\ \hline
            \err{0} & \blu{1} & \blu{0} & \blu{0} \\ \hline
            \blu{1} & \blu{0} & \blu{1} & \blu{1} \\ \hline
        \end{tabular}
    \end{table}
    \end{column}
    \hfill
    \begin{column}{5cm}
        \begin{itemize}
            \item Consider a 16-bit block
            \item We want to fill it with the message 01101001011
            \item We'll use the remaining 5 bits to mark \err{parity} of certain regions of the block.
        \end{itemize}
    \end{column}
    \end{columns}
\end{frame}

\begin{frame}{Hamming Codes}
    \begin{columns}[c]
    \begin{column}{5cm}
    \begin{table}
        \centering
        \begin{tabular}{|c|a|c|a|}
            \hline 
            0 & \blu{\textbf{1}} & 0 & 0 \\ \hline
            0 & 1 & 1 & 0 \\ \hline
            0 & 1 & 0 & 0 \\ \hline
            1 & 0 & 1 & 1 \\ \hline
        \end{tabular}
    \end{table}
    \end{column}
    \hfill
    \begin{column}{5cm}
        \begin{itemize}
            \item We use the parity bit to keep the marked region even parity (ignoring parity bits themselves).
        \end{itemize}
    \end{column}
    \end{columns}
\end{frame}

\begin{frame}{Hamming Codes}
    \begin{columns}[c]
    \begin{column}{5cm}
    \begin{table}
        \centering
        \begin{tabular}{|c|c|a|a|}
            \hline 
            0 & 1 & \blu{\textbf{1}} & 0 \\ \hline
            0 & 1 & 1 & 0 \\ \hline
            0 & 1 & 0 & 0 \\ \hline
            1 & 0 & 1 & 1 \\ \hline
        \end{tabular}
    \end{table}
    \end{column}
    \hfill
    \begin{column}{5cm}
        \begin{itemize}
            \item We use the parity bit to keep the marked region even parity (ignoring parity bits themselves).
        \end{itemize}
    \end{column}
    \end{columns}
\end{frame}

\begin{frame}{Hamming Codes}
    \begin{columns}[c]
    \begin{column}{5cm}
    \begin{table}
        \centering
        \begin{tabular}{|c|c|c|c|}
            \hline
            0 & 1 & 1 & 0 \\ \hline
            \rowcolor{LightRed}
            \blu{\textbf{1}} & 1 & 1 & 0 \\ \hline
            0 & 1 & 0 & 0 \\ \hline
            \rowcolor{LightRed}
            1 & 0 & 1 & 1 \\ \hline
        \end{tabular}
    \end{table}
    \end{column}
    \hfill
    \begin{column}{5cm}
        \begin{itemize}
            \item We use the parity bit to keep the marked region even parity (ignoring parity bits themselves).
        \end{itemize}
    \end{column}
    \end{columns}
\end{frame}

\begin{frame}{Hamming Codes}
    \begin{columns}[c]
    \begin{column}{5cm}
    \begin{table}
        \centering
        \begin{tabular}{|c|c|c|c|}
            \hline 
            0 & 1 & 1 & 0 \\ \hline
            1 & 1 & 1 & 0 \\ \hline
            \rowcolor{LightRed}
            \blu{\textbf{0}} & 1 & 0 & 0 \\ \hline
            \rowcolor{LightRed}
            1 & 0 & 1 & 1 \\ \hline
        \end{tabular}
    \end{table}
    \end{column}
    \hfill
    \begin{column}{5cm}
        \begin{itemize}
            \item We use the parity bit to keep the marked region even parity (ignoring parity bits themselves).
        \end{itemize}
    \end{column}
    \end{columns}
\end{frame}

\begin{frame}{Hamming Codes}
    \begin{columns}[c]
    \begin{column}{5cm}
    \begin{table}
        \centering
        \begin{tabular}{|c|c|c|c|}
            \hline 
            \blu{\textbf{0}} & 1 & 1 & 0 \\ \hline
            1 & 1 & 1 & 0 \\ \hline
            0 & 1 & 0 & 0 \\ \hline
            1 & 0 & 1 & 1 \\ \hline
        \end{tabular}
    \end{table}
    \end{column}
    \hfill
    \begin{column}{5cm}
        \begin{itemize}
            \item What about the bit in the zero position? \pause
            \item We'll set it to keep the parity of the entire table. \pause
            \item This allows us to detect a second error, should one exist.
        \end{itemize}
    \end{column}
    \end{columns}
\end{frame}


\begin{frame}{Hamming Codes}
    \begin{itemize}
        \item We can detect the error by checking each parity bit, and fix it!
    \end{itemize}
    \begin{columns}[c]
    \begin{column}{2cm}
    \begin{table}
        \centering
        \begin{tabular}{|c|a|c|a|}
            \hline 
            0 & 1 & 1 & 0 \\ \hline
            1 & 1 & \blu{0} & 0 \\ \hline
            0 & 1 & 0 & 0 \\ \hline
            1 & 0 & 1 & 1 \\ \hline
        \end{tabular}
    \end{table}
    \end{column}
    \hfill
    \begin{column}{2cm}
    \begin{table}
        \centering
        \begin{tabular}{|c|c|a|a|}
            \hline 
            0 & 1 & 1 & 0 \\ \hline
            1 & 1 & \blu{0} & 0 \\ \hline
            0 & 1 & 0 & 0 \\ \hline
            1 & 0 & 1 & 1 \\ \hline
        \end{tabular}
    \end{table}
    \end{column}
    \hfill
    \begin{column}{2cm}
    \begin{table}
        \centering
        \begin{tabular}{|c|c|c|c|}
            \hline 
            0 & 1 & 1 & 0 \\ \hline
            \rowcolor{LightRed}
            1 & 1 & \blu{0} & 0 \\ \hline
            0 & 1 & 0 & 0 \\ \hline
            \rowcolor{LightRed}
            1 & 0 & 1 & 1 \\ \hline
        \end{tabular}
    \end{table}
    \end{column}
    \hfill
    \begin{column}{2cm}
    \begin{table}
        \centering
        \begin{tabular}{|c|c|c|c|}
            \hline     
            0 & 1 & 1 & 0 \\ \hline
            1 & 1 & \blu{0} & 0 \\ \hline
            \rowcolor{LightRed}
            0 & 1 & 0 & 0 \\ \hline
            \rowcolor{LightRed}
            1 & 0 & 1 & 1 \\ \hline
        \end{tabular}
    \end{table}
    \end{column}
    \end{columns}
\end{frame}

\begin{frame}{}
      \begin{center}
    {\color{sigma@mainblue} \LARGE Questions?}
  \end{center}
\end{frame}

\begin{frame}{Making More Codes}
    \begin{itemize}
        \item What we described was a (15,11) Extended Hamming Code. (11 bits of message, 4 bit EC parity, 1 bit detection) \pause
        \item You can easily make any $(2^n-1,2^n-(n+1))$ Hamming Code the same way. 
        \begin{itemize}
            \item Place each parity bit in the row/column corresponding to powers of 2 \pause 
        \end{itemize}
        \item In a $2^n$ block, we can efficiently detect 2 errors, and correct 1 error.
    \end{itemize}
\end{frame}

\begin{frame}{In Practice}
    \begin{itemize}
        \item You can implement hamming code processing in hardware, or in software
        \item When sending info, errors tend to happen in \textit{bursts}.
        \item Interleave blocks to spread out the errors that could happen.
    \end{itemize}
\end{frame}

\section{Hamming Bound}

\frame{\sectionpage}

\begin{frame}{Hamming Bound}
    \begin{itemize}
        \item Exactly how efficient can we get with error correcting codes? \pause
        \item We have a lower bound called the {\color{sigma@mainblue}Hamming Bound} \pause
        \item It gives us the efficiency of how well any error correcting code can utilize the space within the entire code word. \pause
        \item Codes that achieve this bound are called \blu{Perfect Codes}
    \end{itemize}
\end{frame}

\begin{frame}{Hamming Bound}
    Let $\Sigma = \{0,1\}$. Let $A_\Sigma(n,d)$ be the maximum possible size of a block code $C$ of length $n$, with minimum hamming distance $d$ between elements of the block code.

    Then,
    $$A_\Sigma(n,d) \leq \frac{|\Sigma|^n}{\sum\limits_{k=0}^{\lfloor (d-1)/2\rfloor} \binom{n}{k}(|\Sigma|-1)^k}$$

    \textit{(Recall that hamming distance is the number of flips you need to reach another string)}
\end{frame}

\begin{frame}{Hamming Bound}
    \begin{itemize}
        \item We're effectively that over all strings of length $n$ $(|\Sigma|^n)$,\pause
        \item If we can make at most $t=\lfloor(d-1)/2\rfloor$ errors, \pause
        \item Then our code $C$ covers the sum over all possible errors up to $k\leq t$, choosing $k$ bits to flip to some other bit. 
    \end{itemize}
\end{frame}

\begin{frame}{Hamming Bound}
    Let $\Sigma = \{0,1\}$. Let $A_\Sigma(n,d)$ be the maximum possible size of a block code $C$ of length $n$, with minimum hamming distance $d$ between elements of the block code.

    Then,
    $$A_2(n,d) \leq \frac{2^n}{\sum\limits_{k=0}^{\lfloor (d-1)/2\rfloor} \binom{n}{k}}$$

    \textit{(Recall that hamming distance is the number of flips you need to reach another string)}
\end{frame}

\begin{frame}{What does $A_\Sigma(n,d)$ \textit{mean?}}
    \begin{itemize}
        \item We have a $q$-ary language ($q = |\Sigma|$), of which we consider some string of length $n$ \pause
        \item \textit{How many strings can we make from at most $d$ changes}?
    \end{itemize}
\end{frame}

\begin{frame}{Computing $A_\Sigma(n,d)$ for Hamming Codes}

    \begin{itemize}
        \item Our basic (15,11) Hamming code (no 0 bit), requires \textit{three} flips to go from one valid code to another \pause
        \item While we can detect errors of 2 flips or less, once we go above 2, distinguishing between one valid message to another is impossible. \pause
        \item This means that the number of strings we can represent with our (15,11) Hamming code, with a minimum hamming distance of $3$ is $2^{11}= 2048$
    \end{itemize}
    
\end{frame}

\begin{frame}{Computing $A_\Sigma(n,d)$ for Hamming Codes}
    \begin{itemize}
        \item So let's compute the Hamming bound for $A_\Sigma(15,3)$ \pause
    \end{itemize}
    $$\frac{2^{15}}{\sum_{k=0}^1 \binom{15}{k}(2-1)^k} = \frac{2^{15}}{16} = 2048$$ \pause
    \begin{itemize}
        \item So our (15,11) Hamming code matches our hamming bound, thus it is considered a \blu{Perfect Code}
    \end{itemize}
\end{frame}

\begin{frame}{Practical, not Perfect}
    \begin{itemize}
        \item Our \textit{extended} Hamming code includes that 0 bit to detect \textit{one more error.} \pause
        \item This means that in a 16 bit code word, we have a minimum hamming distance of 4 \pause
        \item $A_\Sigma(16,4) \leq 3855.06$, but we still can only represent $2048$ messages. \pause
        \item Even though its not \textit{perfect}, this is the mechanism still used in error correction within RAM on your computers. Nice powers of 2 are easy to send around.
    \end{itemize}
\end{frame}

% Quotes are fun, find some to use!
\font\eightss=cmssq8
\font\eightssi=cmssqi8
\newcommand\quoteAuthorDate[3]{\begingroup
  \baselineskip 10pt
  \parfillskip 0pt
  \interlinepenalty 10000 % not needed in example
  \leftskip 0pt plus 40pc minus \parindent
  \let\rm=\eightss
  \let\sl=\eightssi
  \everypar{\sl}#1\par
  \nobreak\smallskip
  \noindent\rm--- #2\unskip\enspace(#3)\par
  \endgroup}
% If someone can figure out how to horizontally center this and make the text bigger that'd be cool
\begin{frame}
    \begin{center}
        \item \quoteAuthorDate{The purpose of computing is insight, not numbers.}{Richard Hamming, PhD UIUC}{1960s}
    \end{center}
\end{frame}

% Remove this slide if you came up with all the material yourself
%im lazy to fix this atm
% \begin{frame}{Bibliography}
%     \bibliography{refs}
%     \bibliographystyle{alpha}
% \end{frame}

\end{document}