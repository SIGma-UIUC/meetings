% Mirror: https://github.com/SIGma-UIUC/presentation-format
% --------------------------------------------------------------------
% This is a simple Beamer document that uses beamerthemesigma.sty
% Reading the comments should help you create a presentation even if
% you've never used Beamer before.
% --------------------------------------------------------------------

% Set our document class to Beamer
\documentclass[aspectratio=169]{beamer}

% From Jeff E
\usepackage{algo}
% Some more macros
\usepackage{sigmastyle}


% Set a title
\title{Welcome to SIGma}

% Set a subtitle if you desire
\subtitle{\cite{zeck, robust}}

% Whoever worked on the presentation:
\author{SIGma}

% Date looks ugly, so leave blank
\date{}

% An institute name, if you're so inclined
% \institute{University of Illinois Urbana-Champaign}

% Use the SIGma theme for this Beamer presentation
\usetheme{sigma}
% --------------------------------------------------------------------

% Begin document
\begin{document}

% Beamer calls each slide a "frame", defined within the environment:
% \begin{frame}
%   <frame content here>
% \end{frame}

% This frame is just the title.
\begin{frame}
\titlepage
\end{frame}

% % A frame with the table of contents.
% % This frame's title is "Outline".
% \begin{frame}{Outline}
%   \tableofcontents
% \end{frame}

\section{Officers in No Particular Order}
\frame{\sectionpage}

\begin{frame}{Anakin} 
        \begin{itemize}
        \item Math Major
        \item Did Computational Group Theory at an REU
        \item Graph Theory / Optimization Research during the year
        \item SIGPwny Crypto\footnotemark\ Gang + Admin team
        \item Coffee Club
        \item CA for CS 173 + CS 374
    \end{itemize}
    \footnotetext[1]{Not that one, the other one}
\end{frame}

\begin{frame}{Aditya}
    \begin{itemize}
        \item ECE/Math double major.
        \item Interned at a satellite internet startup over the summer.
        \item CA for ECE 411, ECE 391 + SIGARCH co-lead.
        \item Other interests: FP, EE, Crypto(graphy).
    \end{itemize} 
\end{frame}

\begin{frame}{Sam}
    \begin{itemize}
    \item CS PhD
    \item Doing Computational Geometry with Sariel Har-Peled
    \item SIGPwny
    \end{itemize}
\end{frame}

\begin{frame}{Hassam}
    \begin{itemize}
        \item Intern at IMC Trading over the summer
        \item CS Major (takes math classes for fun ???)
        \item SIGPwny Crypto Gang + Admin team + Infra lead
        \item CA for CS 341, CS 173
        \item Compiler research
    \end{itemize}
\end{frame}

\begin{frame}{We Need Officers!}
    \begin{itemize}
        \item This list is smaller than last year \pause
        \item Reach out to me if you are interested in improving SIGma and making meetings!
    \end{itemize}
\end{frame}

\section{Fibonacci Codes}
\frame{\sectionpage}

\begin{frame}{But Why?}
    \begin{itemize}
        \item Almost everything you do online involves sending and receiving messages \pause
        \item How can we make these messages ``robust'' to errors?
    \end{itemize}
\end{frame}

\begin{frame}{Starting From The End}
    \begin{itemize}
        \item Suppose we want to uniquely assign the natural numbers a \emph{codeword} \pause
        \item We want this \emph{code} to have a couple properties
        \begin{itemize}
            \item Quick to compute
            \item Variable length
            \item Robust to errors
        \end{itemize} \pause
        \item It turns out \emph{Fibonacci Numbers} do all this for us
    \end{itemize}
\end{frame}

\begin{frame}{Our Favorite Sequence}
    \[
        F_n = \begin{cases}
                0                       & n = 0 \\
                1                       & n = 1 \\
                F_{n - 1} + F_{n - 2}   & n \geq 2
              \end{cases}
    \]
    
    \pause
    \vspace{20pt}
    \begin{table}[]
        \begin{tabular}{|c|c|c|c|c|c|c|c|c|c|c|c|c|c|}
            \hline
            $F_0$ & $F_1$ & $F_2$ & $F_3$ & $F_4$ & $F_5$ & $F_6$ & $F_7$ & $F_8$ & $F_9$ & $F_{10}$ & $F_{11}$ & $F_{12}$ & $F_{13}$ \\ \hline
            0     & 1     & 1     & 2     & 3     & 5     & 8     & 13    & 21    & 34    & 55       & 89       & 144      & 233      \\ \hline
        \end{tabular}
    \end{table}
\end{frame}

\begin{frame}{Zeckendorf's Theorem}
    \begin{thrm}[\cite{zeck}]
        Every natural number $n \geq 1$ can be represented as a unique sum of non-consecutive Fibonacci numbers $F_i$ where $i \geq 2$. \pause
        \textit{If we allow $F_0$ and $F_1$ we lose uniqueness}
        
        We call this sum a \emph{Zeckendorf sum}.
    \end{thrm} \pause
    \begin{itemize}
        \item $4 = F_4 + F_2 = 3 + 1$
        \item $64 = F_{10} + F_6 + F_2 = 55 + 8 + 1$
    \end{itemize}
\end{frame}

\begin{frame}{Recursion is Induction is Recursion is Induction is $\ldots$}
    We prove existence by \emph{induction} on our natural number $n$.
    Suppose that for all natural numbers strictly smaller than $n$, such a Zeckendorf sum exists.
    There are two cases. \pause

    If $n \leq 4\colon$
    \begin{align*}
        1 &= F_2 & 2 &= F_3 \\
        3 &= F_4 & 4 &= F_4 + F_2 
    \end{align*} \pause

    If $n > 4$ itself is a Fibonacci number, we are done.
\end{frame}

\begin{frame}{Recursion is Induction is Recursion is Induction is $\ldots$}
    If $n > 4$ is not a Fibonacci number:
    \begin{itemize}
        \item Since $n > 4$, it is strictly between two consecutive Fibonacci numbers $F_i < n < F_{i + 1}$ for some $i \geq 3$ \pause
        \item $n - F_i < n$ so, by \emph{induction}, $n - F_i$ has some Zeckendorf sum \pause
        \item Note that
        \begin{align*}
                    & n - F_i + F_i = n < F_{i + 1} = F_{i - 1} + F_i \\
           \implies & n - F_i < F_{i - 1}
        \end{align*} 
        and thus the Zeckendorf sum of $n - F_i$ does not contain $F_{i - 1}$ \pause
        \item Combine the Zeckendorf sum of $n - F_i$ with $F_i$ to obtain a Zeckendorf sum for $n$ \qed
    \end{itemize}
\end{frame}

\begin{frame}{Deadly Sins $\implies$ Fast Algorithms}
    The statement and proof of the theorem helps design a \emph{greedy} algorithm \pause
    \begin{itemize}
        \item The inductive proof implies we should find the largest $F_i \leq n$ \pause
        \item The statement implies that if we picked $F_i$, we should skip $F_{i - 1}$ \pause
        \item Our goal is to encode text, so we can precompute an array of Fibonacci numbers ahead of time up to some maximum
    \end{itemize}
    \begin{nalgo}
      \label{}  \textit{maximum} $\gets 1114111$ \Comment{largest Unicode value U+10FFFF}
    \\\label{}  \textsc{F} $\gets [0, 1]$
    \\\label{}  $i \gets 2$
    \\\label{}  while $\textsc{F}[i - 1] \leq \textit{maximum}$:\+
    \\\label{}      \textsc{F}.append($\textsc{F}[i - 2] + \textsc{F}[i - 1]$)
    \\\label{}      $i \pluseq 1$
    \end{nalgo}
\end{frame}

\begin{frame}{Deadly Sins $\implies$ Fast Algorithms}
   \begin{nalgo}[0.925]
       \underline{\textbf{\textsc{Zeckendorf}}$(x)$:}\+
    \\\label{}      $i \gets$ max $i$ such that $\textsc{F}[i] \leq x$
    \\\label{}      \textit{rep} $\gets $ ``~''
    \\\label{}      \textit{rem} $\gets x$
    \\\label{}      while $i \geq 2:$\+
    \\\label{}          if $\textsc{F}[i] \leq \textit{rem}:$\+
    \\\label{}              $\textit{rem} \minuseq \textsc{F}[i]$
    \\\label{}              $\textit{rep} \pluseq \texttt{1}$
    \\\label{}              if $\textit{rem} > 0:$\+
    \\\label{}                  $\textit{rep} \pluseq \texttt{1}$
    \\\label{}                  $i \minuseq 1$\-\-
    \\\label{}          else:\+
    \\\label{}              $\textit{rep} \pluseq \texttt{0}$\-
    \\\label{}          $i \minuseq 1$\-
    \\\label{}      return \textit{rep}
   \end{nalgo}
\end{frame}

\begin{frame}{Deadly Sins $\implies$ Fast Algorithms}
\vspace{-5pt}
\begin{minipage}[t]{0.45\textwidth}
   \begin{nalgo}[0.925]
       \underline{\textbf{\textsc{Zeckendorf}}$(x)$:}
    \\\label{}      $i \gets$ max $i$ such that $\textsc{F}[i] \leq i$
    \\\label{}      {\color{lightgray}\textit{rep} $\gets $ ``~''}
    \\\label{}      {\color{lightgray}\textit{rem} $\gets x$}
    \\\label{}      {\color{lightgray}while $i \geq 2:$\+}
    \\\label{}      {\color{lightgray}    if $\textsc{F}[i] \leq \textit{rem}:$\+}
    \\\label{}      {\color{lightgray}        $\textit{rem} \minuseq \textsc{F}[i]$}
    \\\label{}      {\color{lightgray}        $\textit{rep} \pluseq \texttt{1}$}
    \\\label{}      {\color{lightgray}        if $\textit{rem} > 0:$\+}
    \\\label{}      {\color{lightgray}            $\textit{rep} \pluseq \texttt{1}$}
    \\\label{}      {\color{lightgray}            $i \minuseq 1$\-\-}
    \\\label{}      {\color{lightgray}    else:\+}
    \\\label{}      {\color{lightgray}        $\textit{rep} \pluseq \texttt{0}$\-}
    \\\label{}      {\color{lightgray}    $i \minuseq 1$\-}
    \\\label{}      {\color{lightgray}return \textit{rep}}
   \end{nalgo}
\end{minipage}
\hspace{1pt}
\begin{minipage}[t]{0.5\textwidth}
    \begin{align*}
          & \# \text{ of } i \text{ such that } F_i \leq x \\
          =& \floor{\log_\phi\pqty{x\sqrt{5}}} = O(\log x)
    \end{align*}
    \begin{itemize}
        \item {\color{lightgray}The while loop does $i = O(\log x)$ iterations}
        \item {\color{lightgray}The work inside the while loop takes $O(1)$ time}
        \item {\color{lightgray}So \textbf{\textsc{Zeckendorf}} takes $O(\log x)$ time}
        \item {\color{lightgray}Each iteration we add at most 2 characters $\implies \abs{\textbf{\textsc{Zeckendorf}}(x)} = O(\log x)$}
    \end{itemize}
\end{minipage}
\end{frame}

\begin{frame}{Deadly Sins $\implies$ Fast Algorithms}
\vspace{-5pt}
\begin{minipage}[t]{0.45\textwidth}
   \begin{nalgo}[0.925]
       \underline{\textbf{\textsc{Zeckendorf}}$(x)$:}
    \\\label{}      {\color{lightgray}$i \gets$ max $i$ such that $\textsc{F}[i] \leq i$}
    \\\label{}      {\color{lightgray}\textit{rep} $\gets $ ``~''}
    \\\label{}      {\color{lightgray}\textit{rem} $\gets x$}
    \\\label{}      while $i \geq 2:$\+
    \\\label{}          if $\textsc{F}[i] \leq \textit{rem}:$\+
    \\\label{}              $\textit{rem} \minuseq \textsc{F}[i]$
    \\\label{}              $\textit{rep} \pluseq \texttt{1}$
    \\\label{}              if $\textit{rem} > 0:$\+
    \\\label{}                  $\textit{rep} \pluseq \texttt{1}$
    \\\label{}                  $i \minuseq 1$\-\-
    \\\label{}          else:\+
    \\\label{}              $\textit{rep} \pluseq \texttt{0}$\-
    \\\label{}          $i \minuseq 1$\-
    \\\label{}      {\color{lightgray}return \textit{rep}}
   \end{nalgo}
\end{minipage}
\hspace{1pt}
\begin{minipage}[t]{0.5\textwidth}
    {\color{lightgray}\begin{align*}
          & \# \text{ of } i \text{ such that } F_i \leq x \\
          =& \floor{\log_\phi\pqty{x\sqrt{5}}} = O(\log x)
    \end{align*}}
    \begin{itemize}
        \item The while loop does $i = O(\log x)$ iterations 
        \item {\color{lightgray}The work inside the while loop takes $O(1)$ time}
        \item {\color{lightgray}So \textbf{\textsc{Zeckendorf}} takes $O(\log x)$ time}
        \item {\color{lightgray}Each iteration we add at most 2 characters $\implies \abs{\textbf{\textsc{Zeckendorf}}(x)} = O(\log x)$}
    \end{itemize}
\end{minipage}
\end{frame}

\begin{frame}{Deadly Sins $\implies$ Fast Algorithms}
\vspace{-5pt}
\begin{minipage}[t]{0.45\textwidth}
   \begin{nalgo}[0.925]
       \underline{\textbf{\textsc{Zeckendorf}}$(x)$:}
    \\\label{}      {\color{lightgray}$i \gets$ max $i$ such that $\textsc{F}[i] \leq i$}
    \\\label{}      {\color{lightgray}\textit{rep} $\gets $ ``~''}
    \\\label{}      {\color{lightgray}\textit{rem} $\gets x$}
    \\\label{}      {\color{lightgray}while $i \geq 2:$\+}
    \\\label{}          if $\textsc{F}[i] \leq \textit{rem}:$\+
    \\\label{}              $\textit{rem} \minuseq \textsc{F}[i]$
    \\\label{}              $\textit{rep} \pluseq \texttt{1}$
    \\\label{}              if $\textit{rem} > 0:$\+
    \\\label{}                  $\textit{rep} \pluseq \texttt{1}$
    \\\label{}                  $i \minuseq 1$\-\-
    \\\label{}          else:\+
    \\\label{}              $\textit{rep} \pluseq \texttt{0}$\-
    \\\label{}          $i \minuseq 1$\-
    \\\label{}      {\color{lightgray}return \textit{rep}}
   \end{nalgo}
\end{minipage}
\hspace{1pt}
\begin{minipage}[t]{0.5\textwidth}
    {\color{lightgray}\begin{align*}
          & \# \text{ of } i \text{ such that } F_i \leq x \\
          =& \floor{\log_\phi\pqty{x\sqrt{5}}} = O(\log x)
    \end{align*}}
    \begin{itemize}
        \item {\color{lightgray}The while loop does $i = O(\log x)$ iterations}
        \item The work inside the while loop takes $O(1)$ time
        \item {\color{lightgray}So \textbf{\textsc{Zeckendorf}} takes $O(\log x)$ time}
        \item {\color{lightgray}Each iteration we add at most 2 characters $\implies \abs{\textbf{\textsc{Zeckendorf}}(x)} = O(\log x)$}
    \end{itemize}
\end{minipage}
\end{frame}



\begin{frame}{Deadly Sins $\implies$ Fast Algorithms}
\vspace{-5pt}
\begin{minipage}[t]{0.45\textwidth}
   \begin{nalgo}[0.925]
       \underline{\textbf{\textsc{Zeckendorf}}$(x)$:}
    \\\label{}      $i \gets$ max $i$ such that $\textsc{F}[i] \leq i$
    \\\label{}      \textit{rep} $\gets $ ``~''
    \\\label{}      \textit{rem} $\gets x$
    \\\label{}      while $i \geq 2:$\+
    \\\label{}          if $\textsc{F}[i] \leq \textit{rem}:$\+
    \\\label{}              $\textit{rem} \minuseq \textsc{F}[i]$
    \\\label{}              $\textit{rep} \pluseq \texttt{1}$
    \\\label{}              if $\textit{rem} > 0:$\+
    \\\label{}                  $\textit{rep} \pluseq \texttt{1}$
    \\\label{}                  $i \minuseq 1$\-\-
    \\\label{}          else:\+
    \\\label{}              $\textit{rep} \pluseq \texttt{0}$\-
    \\\label{}          $i \minuseq 1$\-
    \\\label{}      return \textit{rep}
   \end{nalgo}
\end{minipage}
\hspace{1pt}
\begin{minipage}[t]{0.5\textwidth}
    {\color{lightgray}\begin{align*}
          & \# \text{ of } i \text{ such that } F_i \leq x \\
          =& \floor{\log_\phi\pqty{x\sqrt{5}}} = O(\log x)
    \end{align*}}
    \begin{itemize}
        \item {\color{lightgray}The while loop does $i = O(\log x)$ iterations}
        \item {\color{lightgray}The work inside the while loop takes $O(1)$ time}
        \item So \textbf{\textsc{Zeckendorf}} takes $O(\log x)$ time 
        \item Each iteration we add at most 2 characters $\implies \abs{\textbf{\textsc{Zeckendorf}}(x)} = O(\log x)$
    \end{itemize}
\end{minipage}
\end{frame}

\begin{frame}{Undo}
    \begin{nalgo}
       \underline{\textbf{\textsc{Frodnekcez}}$(\textit{rep}[0..n])$:}
    \\\label{}      $\textit{res} \gets 0$
    \\\label{}      for $i \gets 0..n$:\+
    \\\label{}          if $\textit{rep}[i] = \texttt{1}$:\+
    \\\label{}              $\textit{res} \pluseq \textsc{F}[i + 2]$ \Comment{Remember we don't use $F_0$ or $F_1$}\-\-
    \\\label{}      return $\textit{res}$
    \end{nalgo}\pause

    Runtime analysis:
    \begin{itemize}
        \item Work is constant for each iteration of the for loop $\implies O(n)$ \pause
        \item If $\textit{rep}[0..n] = \textbf{Zeckendorf}(x)$ then $O(n) = O(\log x)$
    \end{itemize}
\end{frame}

\begin{frame}{A Fibonacci Code}
    We now show how to assign natural numbers a code word using the Zeckendorf Decomposition \cite{robust}
    \begin{itemize}
        \item The length of the Zeckendorf Representation for numbers can vary
        \begin{itemize}
            \item $\textbf{\textsc{Zeckendorf}}(2) = \texttt{01}$, $\textbf{\textsc{Zeckendorf}}(7) = \texttt{0101}$
        \end{itemize}\pause
        \item We want to be able to send this bit strings and tell when a character begins and ends
        \begin{itemize}
            \item Does $\texttt{0101}$ correspond to $\bqty{2, 2}$ or $\bqty{7}$?
        \end{itemize} \pause
        \item Solution: Add a \emph{``comma''} using an extra \texttt{1}
        \begin{itemize}
            \item $\textbf{\textsc{ENC}}\pqty{\bqty{2, 2}} = \texttt{011011}$, $\textbf{\textsc{ENC}}\pqty{\bqty{7}} = \texttt{01011}$
        \end{itemize}
    \end{itemize}
\end{frame}

\begin{frame}{Heavy Lifting Has Already Been Done}
    \begin{nalgo}
        \underline{\textbf{\textsc{enc}}$(x)$:}
    \\\label{}      return \textbf{\textsc{Zeckendorf}}($x$) + \texttt{1} \Comment{add comma} \hspace{41.25pt}
    \end{nalgo} \pause
    
    \begin{nalgo}
        \underline{\textbf{\textsc{dec}}$(\textit{rep}[0..n])$:}
    \\\label{}      return \textbf{\textsc{Frodnekcez}}($\textit{rep}[0..n - 1]$) \Comment{remove comma}
    \end{nalgo} \pause

    Runtime analysis:
    \begin{itemize}
        \item Same as $\textbf{\textsc{Zeckendorf}}$ and $\textbf{\textsc{Frodnekcez}}$
    \end{itemize}
\end{frame}

\begin{frame}{Heavy Lifting Has Already Been Done}
    \begin{nalgo}
        \underline{\textbf{\textsc{encode}}$(m[0..n])$:}
    \\\label{}      $\textit{code} \gets$ ``~''
    \\\label{}      for $i \gets 0..n$:\+
    \\\label{}          $\textit{val} \gets \textsc{ord}(m[i])$
    \\\label{}          $\textit{code} \pluseq \textbf{\textsc{enc}}(\textit{val})$\-
    \\\label{}      return \textit{code}
    \end{nalgo} \pause
    Runtime analysis:
    \begin{itemize}
        \item To simplify our life, since $\textsc{ord}(m[i])$ is some Unicode value which has a set maximum, $\textbf{\textsc{enc}}$ runs in constant time \pause
        \begin{itemize}
            \item More precise analysis would require knowledge of the distribution of characters in whatever language being used. Ask your nearest linguist.
        \end{itemize} \pause
        \item Thus, $\textbf{\textsc{encode}}(m[0..n])$ runs in $O(n)$ time
    \end{itemize}
\end{frame}

\begin{frame}{Heavy Lifting Has Already Been Done}
    \begin{nalgo}
        \underline{\textbf{\textsc{decode}}$(code[0..n])$:}
    \\      $m \gets$ ``~''
    \\      $i \gets 0$
    \\      while $i \leq n$:\+
    \\          $j \gets $ smallest $j > i$ such that 
    \\          $\textit{code}[j] = \textit{code}[j + 1] = \texttt{1}$
    \\          $\textit{rep} = \textit{code}[i..j + 1]$
    \\          $n \gets \textbf{\textsc{dec}}(\textit{rep})$
    \\          $m \pluseq \textsc{chr}(n)$
    \\          $i \gets j + 1$\-
    \\      return $m$
    \end{nalgo} \pause
    Runtime analysis:
    \begin{itemize}
        \item By similar logic, $\textbf{\textsc{decode}}(code[0..n])$ runs in $O(n)$ time
    \end{itemize}
\end{frame}

\begin{frame}{An Example}
   \begin{tabular}{ccccc}
       \texttt{S} & \texttt{I} & \texttt{G} & \texttt{m} & \texttt{a} \\
       \hline
       83         & 73         & 71         & 109        & 97         \\ 
       \hline
       \texttt{0101001011} & \texttt{0001010011} & \texttt{0010010011}  & \texttt{01010100011}  & \texttt{00001000011} \\ 
   \end{tabular}
\end{frame}

\begin{frame}{Containment of Errors}
    \emph{Claim:} When a single error occurs, at most 3 codewords are lost \pause
    \begin{itemize}
        \item We know that $\textbf{\textsc{enc}}(x)$ ends with \texttt{011} for all $x > 1$ \pause
        \item For such $x$, if an error occurs outside of these last 3 bits, only one codeword is lost:
        \begin{itemize}
            \item If a \texttt{01} gets turned into a \texttt{11}, if a \texttt{0} is deleted, or if a \texttt{1} is inserted in some specific spot, then one codeword may turn into two 
            \item Consider \texttt{0101011} $\leadsto$ \texttt{0111011} / \texttt{011011} / \texttt{01101011} \pause
            \item Otherwise, we just misconvert that single codeword
        \end{itemize}
    \end{itemize} \pause
\end{frame}


% Asking questions is fun but we should answer some first
\begin{frame}{}
      \begin{center}
    {\color{sigma@mainblue} \LARGE Questions?}
  \end{center}
\end{frame}

% Quotes are fun, find some to use!
\font\eightss=cmssq8
\font\eightssi=cmssqi8
\newcommand\quoteAuthorDate[3]{\begingroup
  \baselineskip 10pt
  \parfillskip 0pt
  \interlinepenalty 10000 % not needed in example
  \leftskip 0pt plus 40pc minus \parindent
  \let\rm=\eightss
  \let\sl=\eightssi
  \everypar{\sl}#1\par
  \nobreak\smallskip
  \noindent\rm--- #2\unskip\enspace(#3)\par
  \endgroup}
% If someone can figure out how to horizontally center this and make the text bigger that'd be cool
\begin{frame}
    \begin{center}
        \item \quoteAuthorDate{Abstract is a word people use when they haven't gotten used to something}{EUGENE LERMAN}{\color{sigma@mainblue} 8/28/2023}
    \end{center}
\end{frame}

\begin{frame}{Question: Containment of Errors}
    \emph{Claim:} When a single error occurs, at most 3 codewords are lost 
    \begin{itemize}
        \item We know that $\textbf{\textsc{enc}}(x)$ ends with \texttt{011} for all $x > 1$ 
        \item For such $x$, if an error occurs outside of these last 3 bits, only one codeword is lost:
        \begin{itemize}
            \item If a \texttt{01} gets turned into a \texttt{11}, if a \texttt{0} is deleted, or if a \texttt{1} is inserted in some specific spot, then one codeword may turn into two 
            \item Consider \texttt{0101011} $\leadsto$ \texttt{0111011} / \texttt{011011} / \texttt{01101011} 
            \item Otherwise, we just misconvert that single codeword
        \end{itemize}
    \end{itemize} 
    \textcolor{sigma@alertred}{\textit{Exercise:}} Consider what may happen in the cases of insertion, deletion, and bitflipping for each of the last three bits of $\textbf{\textsc{enc}}(x)$ for $x > 1$
\end{frame}

% Remove this slide if you came up with all the material yourself
\begin{frame}{Bibliography}
    \bibliography{refs}
    \bibliographystyle{alpha}
\end{frame}

\end{document}