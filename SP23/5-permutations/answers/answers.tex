\documentclass[letterpaper]{article}

\usepackage{style}  % If you feel like procrastinating, mess with this file
\usepackage{algo}   % Thank you Jeff, very cool!
\usepackage{quiver} % https://q.uiver.app/

\addbibresource{refs.bib}

% Required reading
% https://jmlr.csail.mit.edu/reviewing-papers/knuth_mathematical_writing.pdf
% https://faculty.math.illinois.edu/~west/grammar.html

% % % % % % % % % %
%     Cursor      %
%     Parking     %
%     Lot         %
% % % % % % % % % %

\title{Permutations Exercise Answers}
\author{Sam Ruggerio}

\begin{document}
\maketitle

\section*{Delay between permutations in \textsc{AlgorithmL}}

Between visiting each permutation, we run at least one \textbf{while} loop.
These loops take $O(n)$ time in the worst case.
Thus, there is an $O(n)$ time delay between visiting permutations

\section*{Runtime of \textsc{AlgorithmL}}

For $n$ items, there are $O(n!)$ possible permutations to visit.
Then we have $O(n)$ delay between visiting each permutation.
Thus, the overall runtime is $O(n \cdot n!)$

\section*{Number of decrements of {\boldmath $j$} in \textsc{AlgorithmL}}

Suppose all elements in $S$ are distint.
Exactly half of the possible permutations of $S$ have that $s_{n - 1} < s_{n}$.
Thus, the loop will not run for $\frac{n!}{2}$ permutations of $S$.
\textbf{Note:} Exercise 1 of \cite[Chapter~7.2.1.2]{TAOCP4A} talks about how to modify \textsc{AlgorithmL} to take advantage of this fact and obtain a slightly faster algorithm.

\printbibliography
\end{document}
