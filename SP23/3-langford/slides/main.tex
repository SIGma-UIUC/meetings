% Mirror: https://github.com/SIGma-UIUC/presentation-format
% --------------------------------------------------------------------
% This is a simple Beamer document that uses beamerthemesigma.sty
% Reading the comments should help you create a presentation even if
% you've never used Beamer before.
% --------------------------------------------------------------------

% Set our document class to Beamer
\documentclass[aspectratio=169, handout]{beamer}

% Some packages for nice font encodings in the final PDF
\usepackage[utf8]{inputenc}
\usepackage[T1]{fontenc}

% From Jeff E
\usepackage{algo}

% Some more macros
\usepackage{sigmastyle}

% Citations
\usepackage{cite}

% To insert images
\usepackage{graphicx}

% Useful packages from the AMS
\usepackage{amsmath,amssymb,amsthm}

% Package for code highlighting
\usepackage{minted}
\setminted{linenos=true, breaklines=true, breakanywhere=true, style=default}
\usemintedstyle{monokai}

% Set a title
\title{Langford Pairings and Exact Covers}

% Set a subtitle if you desire
\subtitle{\cite[Chapter~7]{TAOCP4A} and \cite[Chapter~7.2.2.1]{TAOCP4B}}

% Whoever worked on the presentation:
\author{Anakin}

% A date, if you'd like.
\date{}

% An institute name, if you're so inclined
% \institute{University of Illinois Urbana-Champaign}

% Use the SIGma theme for this Beamer presentation
\usetheme{sigma}
% --------------------------------------------------------------------

% Begin document
\begin{document}

% Beamer calls each slide a "frame", defined within the environment:
% \begin{frame}
%   <frame content here>
% \end{frame}

% This frame is just the title.
\begin{frame}
\titlepage
\end{frame}

% A frame with the table of contents.
% This frame's title is "Outline".
\begin{frame}{Outline}
  \tableofcontents
\end{frame}

\begin{frame}{Updates!}
  % Let's put some real content in this frame:
  Weekly updates:
    \begin{itemize}
        \item \textbf{TODO}
    \end{itemize}
\end{frame}

\section{Langford Pairings}
\frame{\sectionpage}

\begin{frame}{Langford Pairings}
    Consider the following list, called a ``Langford pairing''
    \begin{equation}\label{Langford3}
        \bqty{2, 3, 1, 2, 1, 3}
    \end{equation}
    It has a very peculiar property. Each pair of the same digits $k$ has exactly $k$ numbers between them \pause
    \begin{itemize}
        \item There is exactly \textbf{1} number between both \textbf{1}'s
        \item There is exactly \textbf{2} numbers between both \textbf{2}'s
        \item There is exactly \textbf{3} numbers between both \textbf{3}'s
    \end{itemize} \pause
    \textcolor{sigma@alertred}{\textbf{Exercise:}} Consider the list of digits $\bqty{1, 1, \ldots, n, n}$. Creating such a number as in Equation~\ref{Langford3} is impossible for $n = 1$ or $2$. We just saw it's possible for $n = 3$. Come up with a pairing for $n = 4$.\pause 
    
    \textcolor{sigma@mainblue}{\textbf{Answer:}} $\bqty{4, 1, 3, 1, 2, 4, 3, 2}$ or $\bqty{2, 3, 4, 2, 1, 3, 1, 4}$.
\end{frame}

\begin{frame}{Existence of Langford Pairs}
    \begin{itemize}
        \item So these Langford pairs for $\bqty{1, 1, \ldots, n, n}$ exist sometimes
        \begin{itemize}
            \item Trivially\footnote{or perhaps stupidly, depending on your perspective}, it exists for $n = 0$ \pause
            \item No such pairing exists for $n = 1$ or $n = 2$ \textcolor{sigma@alertred}{(try it yourself)} \pause
            \item We just saw pairings exist for $n = 3$ and $n = 4$ \pause
            \item Can we characterize for exactly which $n$ we can find pairings?
        \end{itemize}
    \end{itemize}
\end{frame}

\section{Characterization and Existance}
\frame{\sectionpage}

\begin{frame}{A characterization of $n$}
    We are going to characterize the set of $n$ that have at least one Langford pairing. In doing so, we will find a formula to \textcolor{sigma@mainblue}{construct} these pairings. \pause
    
    \vspace{30pt}
    
    \textbf{Theorem \cite{Langford4m}: }{\itshape The numbers $\bqty{1, 1, \ldots, n, n}$ can be arranged in a Langford pairing if and only if $n$ is a multiple of $4$ or one less than a multiple of $4$}
\end{frame}

\begin{frame}{Proof of the Theorem}
    \begin{itemize}
        \item Suppose $\bqty{1, 1, \ldots, n, n}$ can be arranged into some sort of Langford pairing. \pause
        \item Consider the numbers in such a pairing. Let $a_r$ be equal to the index of the first time $r$ appears in the sequence
        \begin{itemize}
            \item Then note that $a_r + r + 1$ is the index of the second time $r$ appears
        \end{itemize} \pause
        \item These $a_r$ and $a_r + r + 1$ are just some arrangement of the indices $1$ through $2n$
    \end{itemize}
\end{frame}

\begin{frame}{Proof of the Theorem}
    Since the $a_r$ and $a_r + r + 1$ are just some arrangement of the indices $1$ through $2n$
    \begin{align*}
        \sum_{r = 1}^n a_r + \sum_{r = 1}^n (a_r + r + 1) &= 2 \sum_{r = 1}^n a_r + \sum_{r = 1}^n r + \sum_{r = 1}^n 1 \\
        &= 2 \sum_{r = 1}^n a_r + \frac{n(n + 1)}{2} + n
    \end{align*}
\end{frame}

\begin{frame}{Proof of the Theorem}
    But the indices in total must sum to
    \[
        \sum_{i = 1}^{2n} i = \frac{2n(2n + 1)}{2} = 2n^2 + n
    \] \pause
    This implies that
    \[
        2 \sum_{r = 1}^n a_r + \frac{n(n + 1)}{2} + n = 2n^2 + n
    \] \pause
    which in turn implies that
    \[
        \sum_{r = 1}^n a_r = \frac{3n^2 - n}{4}
    \]
\end{frame}

\begin{frame}{Proof of the Theorem}
    All the $a_r$ are integers which means that $\sum_{r = 1}^n a_r$ is an integer. Thus $\frac{3n^2 - n}{4}$ must be an integer
    \vspace{20pt}
    \pause
    
    If $n$ is an integer, than $n$ is either $4m$, $4m + 1$, $4m + 2$, or $4m + 3$
    \vspace{20pt}
    \pause
    
    Plugging in all possible options into $\frac{3n^2 - n}{4}$ yields that $n = 4m$ or $4m + 3 = 4(m + 1) - 1$. Thus $n$ is a multiple of 4 or one less than a multiple of 4
\end{frame}

\begin{frame}{Formula for general $n$}
    These formulas are from \cite{Langford4m}. The terms hidden by $\ldots$'s are consecutive even / odd terms. Ex: $(2, 4, 8, \ldots),~ (1, 3, 5, \ldots)$ \pause
    \begin{equation*}
    \begin{split}
        & \textbf{The case }{\boldsymbol n \boldsymbol = \boldsymbol 4 \boldsymbol m \boldsymbol \colon } 4m - 4, \ldots, 2m, 4m - 2, 2m - 3, \ldots, 1, 4m - 1, \\
        & 1, \ldots, 2m - 3, 2m, \ldots, 4m - 4, 4m, 4m - 3, \ldots, 2m + 1, 4m - 2, \\
        & 2m - 2, \ldots, 2, 2m - 1, 4m - 1, 2, \ldots, 2m - 2, 2m + 1, \ldots, 4m - 3, 2m - 1, 4m
    \end{split}    
    \end{equation*} \pause
    \begin{equation*}
    \begin{split}
        & \textbf{The case }{\boldsymbol n \boldsymbol = \boldsymbol 4 \boldsymbol m \boldsymbol - \boldsymbol 1 \boldsymbol \colon } 4m - 4, \ldots, 2m, 4m - 2, 2m - 3, \ldots, 1, 4m - 1, \\
        & 1, \ldots, 2m - 3, 2m, \ldots, 4m - 4, 2m - 1, 4m - 3, \ldots, 2m + 1, 4m - 2, \\
        & 2m - 2, \ldots, 2, 2m - 1, 4m - 1, 2, \ldots, 2m - 2, 2m + 1, \ldots, 4m - 3
    \end{split}
    \end{equation*} \pause
    
    \textcolor{sigma@alertred}{\textbf{Exercise:}} Convince yourself these formulas work by writing a program that generates Langford pairings using these formulas
\end{frame}

\section{Enumeration}
\frame{\sectionpage}

\begin{frame}{Enumeration}
    \begin{itemize}
        \item For $n = 4m$ or $n = 4m - 1$, Langford pairings exist
        \item For $n = 0, 3, 4$ the solution is unique. What about larger $n$? \pause
        \item There are many pairings for larger $n$
        \begin{itemize}
            \item Can we \textcolor{sigma@mainblue}{enumerate} them?
        \end{itemize} \pause
        \item Let $L_n$ denote the number of Langford pairings. We will count a pairing and it's reverse as the same. 
        \item The state of the matter is that it is quite hard to compute $L_n$ \pause
                \item \href{http://dialectrix.com/langford.html}{\textcolor{sigma@mainblue}{John Miller}} has a wonderful online history on enumerating Langford pairings for various $n$
    \end{itemize}
\end{frame}

\begin{frame}{Some Formulas}
    \href{http://dialectrix.com/langford/godfrey/method.html}{\textcolor{sigma@mainblue}{Mike Godfrey}}\footnote{\href{http://dialectrix.com/langford/godfrey/method.html}{\textcolor{sigma@mainblue}{http://dialectrix.com/langford/godfrey/method.html}}} in 2002 came up with the following formula. For a derivation, see \textcolor{sigma@alertred}{Exercise 6a of \cite[Chapter~7]{TAOCP4A}}\pause
    \begin{equation*}
        \text{Let } f(x_1, \ldots, x_{2n}) = \prod_{k = 1}^n \pqty{x_k x_{n + k} \sum_{j = 1}^{2n - k - 1} x_j x_{j + k + 1}}
    \end{equation*}
    \begin{equation*}
        \text{Then } \sum_{x_1, \ldots, x_{2n} \in \set{-1, 1}} f(x_1, \ldots, x_{2n}) = 2^{2n + 1} \cdot L_n
    \end{equation*}\pause
    \cite{LangfordAsymptotic} conjectures some asymptotic approximations for $L_n$
\end{frame}

\section{Exact Covers}
\frame{\sectionpage}

\begin{frame}{Exact Cover Problems}
    \begin{itemize}
        \item Langford Pairings are a special case of a type of problem called \emph{Exact Cover} \pause
        \item In 1972, Richard Karp proved that Exact Cover, among 20 other problems, is \textcolor{sigma@mainblue}{NP-Complete}
        \begin{itemize}
            \item Easy to verify solutions in polynomial time
            \item Hard to solve, best known solutions run in exponential time 
            \item Can simulate (or reduce) other problems in NP using Exact Cover
        \end{itemize} \pause
        \item The goal of Exact Cover is to ``cover'' a list of items using different given subsets, and select each item exactly one time
    \end{itemize}
\end{frame}

\begin{frame}{An Example of Exact Cover}
    \[
        \begin{pmatrix}
            0 & 0 & 1 & 0 & 1 & 0 & 0 \\
            1 & 0 & 0 & 1 & 0 & 0 & 1 \\
            0 & 1 & 1 & 0 & 0 & 1 & 0 \\
            1 & 0 & 0 & 1 & 0 & 1 & 0 \\
            0 & 1 & 0 & 0 & 0 & 0 & 1 \\
            0 & 0 & 0 & 1 & 1 & 0 & 1 \\
        \end{pmatrix}
    \] \pause 
    We can abstract this to \emph{option} containing \emph{items}
    \begin{table}[]
        \begin{tabular}{ l l l }
            $1\colon \bqty{3, 5}$& $2\colon \bqty{1, 4, 7}$ & $3\colon \bqty{2, 3, 6}$ \\ 
            $4\colon \bqty{1, 4, 6}$ & $5\colon \bqty{2, 7}$ & $6\colon \bqty{4, 5, 7}$ \\ 
        \end{tabular}
    \end{table} \pause
    \textcolor{sigma@mainblue}{Answer:} Select items 1, 4, and 5
\end{frame}

\begin{frame}{Solving Exact Cover Problems}
    In trying to solve the previous problem, you may have naturally found a recursive algorithm to find a solution \pause
    \begin{nalgo}
    \underline{\textsc{FindCover}(\emph{Options, Cover}, $i$)}:
    \\\label{}  \textbf{if} \emph{Cover} is a cover:\+
    \\\label{}      \textbf{terminate} successfully\-
    \\\label{}  \textbf{if} no option in \emph{Options} contains $i$:\+
    \\\label{}      \textbf{terminate} unsuccessfully\-
    \\\label{}
    \\\label{}  $I \gets$ options in \emph{Options} that contain $i$
    \\\label{}  \emph{Options} $\gets$ \emph{Options} $\setminus I$
    \\\label{}  \textbf{for} each $O$ in $I$:\+
    \\\label{}      $j \gets$ an item still not covered
    \\\label{}      \textsc{FindCover}(\emph{Options}, \emph{Cover} $\cup \set{O}$, $j$)
    \end{nalgo}
\end{frame}

\begin{frame}{Non-recursive Algorithms}
    \begin{itemize}
        \item In \cite[Chapter~7.2.1.1]{TAOCP4B}, Knuth talks about algorithms which solve exact cover problems \pause
        \item He does so using method involving \textcolor{sigma@mainblue}{doubly linked lists}
        \begin{itemize}
            \item He colorfully calls these \emph{dancing links}
        \end{itemize} \pause
        \item His \textsc{AlgorithmX} uses dancing links to solve exact cover problems
    \end{itemize}
\end{frame}

\begin{frame}{Langford Pairings as an Exact Cover}
    \begin{itemize}
        \item Let's model finding a Langford Pairing as an exact cover problem
        \item Suppose $n = 4$, then we want to place $\bqty{1, 1, \ldots, 4, 4}$ in a list of size 8
        \item Our items can be slots in the list: $l_1, l_2, \ldots, l_8$
        \item Our options can be modeled as such
    \end{itemize}
    \begin{table}[]
        \begin{tabular}{llllll}
            $1\colon \bqty{l_1, l_3}$ & $1\colon \bqty{l_2, l_4}$ & $1\colon \bqty{l_3, l_5}$ & $1\colon \bqty{l_4, l_6}$ & $1\colon \bqty{l_5, l_7}$ & $1\colon \bqty{l_6, l_8}$ \\[10pt]
            $2\colon \bqty{l_1, l_4}$ & $2\colon \bqty{l_2, l_5}$ & $2\colon \bqty{l_3, l_6}$ & $2\colon \bqty{l_4, l_7}$ & $2\colon \bqty{l_5, l_8}$ &                           \\[10pt]
            $3\colon \bqty{l_1, l_5}$ & $3\colon \bqty{l_2, l_6}$ & $3\colon \bqty{l_3, l_7}$ & $3\colon \bqty{l_4, l_8}$ &                           &                           \\[10pt]
            $4\colon \bqty{l_1, l_6}$ & $4\colon \bqty{l_2, l_7}$ & $4\colon \bqty{l_3, l_8}$ &                           &                           &                          
        \end{tabular}
    \end{table}
\end{frame}

\begin{frame}{Langford Pairings as an Exact Cover}
    \begin{itemize}
        \item We can generalize this
        \item For general $n$, what items do we have? \pause
        \begin{itemize}
            \item $l_1, \ldots, l_{2n}$
        \end{itemize}
        \item For some $1 \leq i \leq n$, what $j, k$ work to form an option $i\colon \bqty{l_j, l_k}$? Say $j < k$ to avoid duplicates \pause
        \begin{itemize}
            \item $1 \leq j < k \leq 2n$ \pause
            \item $k = j + i + 1$ 
        \end{itemize}
        \item So all of our options take the form
    \end{itemize}
    \begin{align*}
        i\colon \bqty{l_j, l_k}, & & \text{for } 1 \leq j < k \leq 2n, & & k = j + i + 1, & & 1 \leq i \leq n.
    \end{align*}
    \begin{itemize}
        \item We can use our algorithm \textsc{FindCover} to (perhaps slowly) find all solutions for general $n$
    \end{itemize}
\end{frame}

\begin{frame}{}
      \begin{center}
    {\color{sigma@mainblue} \LARGE Questions?}
  \end{center}
\end{frame}

\font\eightss=cmssq8
\font\eightssi=cmssqi8
\newcommand\quoteAuthorDate[3]{\begingroup
  \baselineskip 10pt
  \parfillskip 0pt
  \interlinepenalty 10000 % not needed in example
  \leftskip 0pt plus 40pc minus \parindent
  \let\rm=\eightss
  \let\sl=\eightssi
  \everypar{\sl}#1\par
  \nobreak\smallskip
  \noindent\rm--- #2\unskip\enspace(#3)\par
  \endgroup}
% If someone can figure out how to horizontally center this and make the text bigger that'd be cool
\begin{frame}
    \begin{center}
        \item \quoteAuthorDate{Combinatorics is special. Most mathematical topics which can be covered in a lecture course build towards a single, well-defined goal, such as the Prime Number Theorem. Even if such a clear goal doesn’t exist, there is a sharp focus (e.g. finite groups). By contrast, combinatorics appears to be a collection of unrelated puzzles chosen at random. Two factors contribute to this. First, combinatorics is broad rather than deep. Second, it is about techniques rather than results.}{PETER J. CAMERON}{\color{sigma@mainblue}1995}
    \end{center}
\end{frame}

\begin{frame}{Questions!}
    \begin{align*}
        i\colon \bqty{l_j, l_k}, & & \text{for } 1 \leq j < k \leq 2n, & & k = j + i + 1, & & 1 \leq i \leq n.
    \end{align*} \vspace{-25pt}
    \begin{itemize}
        \item \textcolor{sigma@alertred}{Exercise 15 of \cite[Chapter~7.2.2.1]{TAOCP4B}:} Recall our formulation of finding Langford Pairings as an exact cover. Running \textsc{FindCover} on this will produce a pairing and it's reverse. Modify our formulation to only produce half of the Langford Pairings for $n$, where the missing half is the reversals of the solutions we find. \vspace{10pt}
        \item Use the formulation of Langford Pairings stated before, or the one you find in the previous exercise, to write a program that finds all Langford Pairings for a given $n$. Try your algorithm out for $n = 7$ (there are $26$, not including reversals).
    \end{itemize}
    
\end{frame}

% Remove this slide if you came up with all the material yourself
\begin{frame}{Bibliography}
    \bibliographystyle{alpha}
    {\scriptsize \bibliography{refs}}
\end{frame}

\end{document}