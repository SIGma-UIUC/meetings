\documentclass[letterpaper]{article}

\usepackage{style}  % If you feel like procrastinating, mess with this file
\usepackage{algo}   % Thank you Jeff, very cool!
\usepackage{quiver} % https://q.uiver.app/

\addbibresource{refs.bib}

% Good writing advice from the man himself
% https://www-cs-faculty.stanford.edu/~knuth/papers/cs1193.pdf

% % % % % % % % % %
%     Cursor      %
%     Parking     %
%     Lot         %
% % % % % % % % % %

\title{Langford Pairs Exercise Answers}
\author{Anakin Dey}

\begin{document}
\maketitle

\section*{Line 2 of \textsc{Algorithm-M}}

This is essentially adding a leading digit of 0. This digit $a[n + 1]$ will act as a flag. While it is $0$, we haven't finished enumerating through all numbers $a[n] \cdots a[1]$. For sake of example, suppose we are enumerating base 10, so $m[j] = 10$ for all $1 \leq j \leq n$, and we just printed our last number $999 \cdots 99$. We then enter the \textbf{while} loop to begin rolling over the digits. Since we have all $9$'s, we will always satisfy $a[j] = m[j] - 1$. Thus the loop will end when $j = n + 1$ since we will have $a[n + 1] = 0$ but $m[n + 1] = 2$ so $a[n + 1] \neq m[j + 1] - 1$. So then we exit the \textbf{while} loop, $j = n + 1$, and so we \textbf{return} and are done. If we did not have these auxiliary variables $a[n + 1]$ and $m[n + 1]$, we would not be able to so cleanly check when we are done enumerating numbers.

\section*{\boldmath Prove that $\Gamma_n$ generates all binary strings $0$ to $2^{n} - 1$}

\quest{TODO:\@ Proof by induction}

\section*{Line 2 of \textsc{Algorithm-G}}

$a[0]$ acts as a parity bit. This bit is mainly there to clean the code up the check when to flip the last bit $a[1]$. This bit flips every time we make the loop (line $5$). It turns out that whenever $a[0] = 1$, we have that $a[1] = 0$ and vice versa. So the \textbf{minimum} $j \geq 1$ such that $a[j - 1] = 1$ is $j = 1$. So adding $a[0]$ and flipping the bit every iteration of the \textbf{while} loop allows us to cleanly check when to flip the bit $a[1]$.

\section{Generating the modular sequence (\cite{TAOCP4A} Chapter 7.2.1.1 Exercise 77)}

\quest{TODO}

\printbibliography
\end{document}
