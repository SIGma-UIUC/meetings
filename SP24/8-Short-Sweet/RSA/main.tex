% Mirror: https://github.com/SIGma-UIUC/presentation-format
% --------------------------------------------------------------------
% This is a simple Beamer document that uses beamerthemesigma.sty
% Reading the comments should help you create a presentation even if
% you've never used Beamer before.
% --------------------------------------------------------------------

% Set our document class to Beamer
\documentclass[handout, aspectratio=169]{beamer}
% \documentclass[aspectratio=169, handout]{beamer}
% Add handout option to ignore pauses

% From Jeff E
\usepackage{algo}
% Some more macros
\usepackage{sigmastyle}


% Set a title
\title{RSA}

% Set a subtitle if you desire
% \subtitle{[TAOCP 5 8.9.10.11]}

% Whoever worked on the presentation:
\author{Sasha Levinshteyn}

% Date looks ugly, so leave blank
\date{}

% An institute name, if you're so inclined
% \institute{University of Illinois Urbana-Champaign}

% Use the SIGma theme for this Beamer presentation
\usetheme{sigma}
% --------------------------------------------------------------------

% Begin document
\begin{document}

% Beamer calls each slide a "frame", defined within the environment:
% \begin{frame}
%   <frame content here>
% \end{frame}

% This frame is just the title.
\begin{frame}
\titlepage
\end{frame}

% A frame with the table of contents.
% This frame's title is "Outline".
\begin{frame}{Outline}
  \tableofcontents
\end{frame}

% \begin{frame}{Updates!}
%   % Let's put some real content in this frame:
%   Weekly updates:
%   \begin{itemize}
%     \item SIGma is an excellent SIG.
%     \item I'm out of ideas for updates.
%   \end{itemize}
% \end{frame}

% Start a section: *sections* (subsections, etc.) are what show up in the TOC.
\section{Public-Key Cryptosystems}
\frame{\sectionpage}

\begin{frame}{General Idea}
    \begin{itemize}
        \item Each user publicizes their encryption procedure $E$
        \item The user determines their corresponding decryption procedure $D$
        \item The user does not reveal $D$
    \end{itemize}
\end{frame}

\begin{frame}{Goal}
    \begin{itemize}
        \item Alice has a message $M$ to send to Bob
        \item Need an encryption method $E$ and a decryption method $D$ such that
        \begin{itemize}
            \item $D(E(M)) = M = E(D(M))$
            \item Both $E$ and $D$ are easy to compute
            \item Publicly revealing $E$ doesn't reveal $D$
        \end{itemize}
        \item We want an encryption key $(e, n)$ and a decryption key $(d, n)$
    \end{itemize}
\end{frame}



\section{RSA Algorithm}
\frame{\sectionpage}

\begin{frame}{Key Distribution}
    \begin{itemize}
        \item If Bob sends a message $M$ to Alice...
        \item Alice sends her public encryption key $(e, n)$ to Bob
        \item She keeps her private decryption key $d$
    \end{itemize}
\end{frame}

\begin{frame}{Encryption}
    \begin{itemize}
        \item Bob turns his message $M$ into some number $m < n$
        \item Long messages can be split up into chunks
        \item He computes the ciphertext $c$ using the encryption algorithm $E$:
        \begin{equation*}
            c \equiv E(m) \equiv m^e \mod n
        \end{equation*}
        \item Computers would use exponentiation by repeating squaring and multiplication
    \end{itemize}
\end{frame}


\begin{frame}{Decryption}
    \begin{itemize}
        \item Alice receive the ciphertext $c$
        \item She decrypts it and finds $m$ using the decryption algorithm $D$:
        \begin{equation*}
            m \equiv D(c) \equiv c^d \mod n
        \end{equation*}
        \item How do we find a valid $e$, $d$, and $n$?
    \end{itemize}
\end{frame}



\begin{frame}{Key Generation}
    \begin{enumerate}
        \item Choose 2 prime numbers $p$ and $q$ \pause
        \item Compute $n = p \cdot q$ \pause
        \item Compute $\phi(n) = (p - 1) \cdot (q - 1)$
        \begin{itemize}
            \item We actually use the Carmichael function now instead \\
            $\implies \lambda(n) = \lcm(p - 1, q - 1)$
        \end{itemize}\pause
        \item Choose $d$ relatively prime to $\phi(n)$ \\ $\implies \gcd(d, \phi(n)) = 1$ \pause
        \item Choose $e$ to be the multiplicative inverse of $d \mod \phi(n)$ \\ $\implies e \cdot d = 1 \mod \phi(n)$ 
        \begin{itemize}
            \item Computers would use Euclid's algorithm \\
        \end{itemize}
    \end{enumerate}
\end{frame}

\begin{frame}{Example}
    \begin{enumerate}
        \item Choose $p = 47$ and $q = 59$ \pause
        \item Compute $n = p \cdot q = 47 \cdot 59 = 2773$ \pause
        \item Compute $\phi(n) = (p - 1) \cdot (q - 1) = 46 \cdot 58 = 2668$ \pause
        \item Choose $d = 157$, which is relatively prime to $\phi(n) = 2668$ \pause
        \item We find that $e = 17$ as $17 \cdot 157 \equiv 1 \mod 2773$ \pause
        \item We release $(n, e) = (2773, 17)$ as our public key and keep $d = 157$ as our private key
    \end{enumerate}
\end{frame}

\begin{frame}{Example}
\begin{enumerate}
\item Convert the message to numbers and encrypt
    \begin{equation*}
    \begin{split}
        &\text{ITS ALL GREEK TO ME} \\
        &\implies 0920 1900 0112 1200 0718 0505 1100 2015 0013 0500 \\
        &\implies 0948 2342 1084 1444 2663 2390 0778 0774 0219 1655
    \end{split}
    \end{equation*}
    \item We write the first block ($m = 920$) as
    \begin{equation*}
        m^{17} \equiv 920^{17} \equiv 948 \mod 2773
    \end{equation*}
\end{enumerate}
\end{frame}

% % Asking questions is fun but we should answer some first
\section{Proof of Correctness}
% Section pages can be printed thus:
\frame{\sectionpage}

\begin{frame}{Proof}
    \begin{itemize}
        \item $\phi(n)$ is the Euler totient function returning the number of integers $k$ less than $n$ relatively prime to $n$ \\
        $\implies \gcd(k, n) = 1, k < n$
        \item Note that for a prime number $p$,
        \begin{equation*}
            \phi(p) = p - 1
        \end{equation*}
        \item Then,
        \begin{equation*}
        \begin{split}
            \phi(n) &= \phi(p) \cdot \phi(q) \\
            &= (p - 1) \cdot (q - 1)
        \end{split}
        \end{equation*}
    \end{itemize}
\end{frame}

\begin{frame}{More Proof}
    \begin{itemize}
    \item $d$ is relatively prime to $\phi(n) \implies d$ has a multiplicative inverse $\mod \phi(n)$
    
    \item Consider $D(E(m))$ and $E(D(m))$
    \begin{equation*}
    \begin{split}
        &D(E(m)) \equiv (E(m))^d \equiv (m^e)^d \equiv m^{e \cdot d} \mod n \\
        &E(D(m)) \equiv (D(m))^e \equiv (m^d)^e \equiv m^{e \cdot d} \mod n
    \end{split}
    \end{equation*}
    \item Then,
    \begin{equation*}
        e \cdot d \equiv 1 \mod \phi(n) \implies m^{e \cdot d} \equiv m^{k \cdot \phi(n) + 1}  \mod n
    \end{equation*}

        % \item We use Fermat's little theorem: for any integer $M$ which is relatively prime to $n$,
        % \begin{equation*}
        %     M^{\phi(n)} \equiv 1 \mod n            
        % \end{equation*}
    \end{itemize}
\end{frame}

\begin{frame}{Even More Proof}
    \begin{itemize}
    \item For any integer $a$ which is relatively prime to $b$,
        \begin{equation*}
            a^{\phi(b)} \equiv 1 \mod b.           
        \end{equation*}
    \item So, as $p - 1$ and $q - 1$ divide $\phi(n)$
    \begin{equation*}
    \begin{split}
            &m^{p - 1} \equiv 1 \mod p \implies m^{k \cdot \phi(n) + 1} = m \mod p \\
            &m^{q - 1} \equiv 1 \mod q \implies m^{k \cdot \phi(n) + 1} = m \mod q
    \end{split}
    \end{equation*}
    \item These equations yield that for all $m$ (as they are trivially true for $m \equiv 0 \mod p$),
    \begin{equation*}
        m^{e \cdot d} \equiv m^{k \cdot \phi(n) + 1} = m \mod n
    \end{equation*}
    \end{itemize}
\end{frame}

\begin{frame}{Security}
    \begin{itemize}
        \item Security relies on the difficulty of factoring $n$
        \begin{itemize}
            \item About 200 digits long and $3.8 \times 10^9$ years to factor by 1977 standards
            \item Typically a few hundred digits now
             \item If we were to know $p$ and $q$, we could perhaps find $d$ from $e$
        \end{itemize}
        \item Computing $\phi(n)$ would allow us to find $d$ as the multiplicative inverse of $e \mod \phi(n)$
        \item Finding $\phi(n)$ or determining $d$ otherwise is at least as hard as factoring $n$
        \item If only there was some way to factor $n$ in polynomial time...
    \end{itemize}
\end{frame}

\section{Conclusion}
% Section pages can be printed thus:
\frame{\sectionpage}

\begin{frame}{Conclusion}
    \begin{itemize}
        \item RSA is very cool
        \item Average Passover with too much wine moment (this is supposedly how Rivest came up with this idea)
    \end{itemize}
\end{frame}


\begin{frame}{}
      \begin{center}
    {\color{sigma@mainblue} \LARGE Questions?}
  \end{center}
\end{frame}



% Quotes are fun, find some to use!
\font\eightss=cmssq8
\font\eightssi=cmssqi8
\newcommand\quoteAuthorDate[3]{\begingroup
  \baselineskip 10pt
  \parfillskip 0pt
  \interlinepenalty 10000 % not needed in example
  \leftskip 0pt plus 40pc minus \parindent
  \let\rm=\eightss
  \let\sl=\eightssi
  \everypar{\sl}#1\par
  \nobreak\smallskip
  \noindent\rm--- #2\unskip\enspace(#3)\par
  \endgroup}
% If someone can figure out how to horizontally center this and make the text bigger that'd be cool
\begin{frame}
    \begin{center}
        \item \quoteAuthorDate{The era of electronic mail may soon be upon us.}{Rivest, Shamir, and Adleman}{\textcolor{sigma@mainblue}{1977}}
        \cite{site:csail}
    \end{center}
\end{frame}

% Remove this slide if you came up with all the material yourself
\begin{frame}[allowframebreaks]{Bibliography}
    \tiny
    \bibliography{refs}
    \bibliographystyle{alpha}
\end{frame}

\end{document}