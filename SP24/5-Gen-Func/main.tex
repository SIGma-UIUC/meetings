% Mirror: https://github.com/SIGma-UIUC/presentation-format
% --------------------------------------------------------------------
% This is a simple Beamer document that uses beamerthemesigma.sty
% Reading the comments should help you create a presentation even if
% you've never used Beamer before.
% --------------------------------------------------------------------

% Set our document class to Beamer
% \documentclass[aspectratio=169]{beamer}
\documentclass[aspectratio=169]{beamer}
% Add handout option to ignore pauses

% From Jeff E
\usepackage{algo}
% Some more macros
\usepackage{sigmastyle}
\usepackage{amssymb}
\usepackage{subcaption}
\usepackage{wrapfig}

\usepackage{forest}
\forestset{
  default preamble={
    for tree={
        grow=south,
        circle, draw, minimum size=3ex, inner sep=1pt,
        s sep=3mm,
    }
  }
}
\usepackage{adjustbox}

\newcommand{\TT}{\mathcal{T}}

% Set a title
\title{Generating Functions}

% Set a subtitle if you desire
% \subtitle{[TAOCP 5 8.9.10.11]}

% Whoever worked on the presentation:
\author{Franklin}

% Date looks ugly, so leave blank
\date{}

% An institute name, if you're so inclined
% \institute{University of Illinois Urbana-Champaign}

% Use the SIGma theme for this Beamer presentation
\usetheme{sigma}
% --------------------------------------------------------------------

% Begin document
\begin{document}

% Beamer calls each slide a "frame", defined within the environment:
% \begin{frame}
%   <frame content here>
% \end{frame}

% This frame is just the title.
\begin{frame}
\titlepage
\end{frame}

% A frame with the table of contents.
% This frame's title is "Outline".
\begin{frame}{Outline}
  \tableofcontents
\end{frame}

% \begin{frame}{Updates!}
%   % Let's put some real content in this frame:
%   Weekly updates:
%   \begin{itemize}
%     \item SIGma is an excellent SIG.
%     \item I'm out of ideas for updates.
%   \end{itemize}
% \end{frame}

% Start a section: *sections* (subsections, etc.) are what show up in the TOC.
\section{Overview}
% Section pages can be printed thus:
\frame{\sectionpage}
% There's a way to automate this, see:
% https://tex.stackexchange.com/questions/178800/creating-sections-each-with-title-pages-in-beamers-slides/178803

\begin{frame}{Introduction}
  Suppose we have a sequence $a_0, a_1, a_2, \dots$. \pause
  
  The \textbf{generating function} for this sequence is
  \[
    f(x) = a_0 + a_1x + a_2 x^2 + \cdots + a_k x^k + \cdots,
  \]
  a polynomial or power series whose coefficients are the elements of the sequence. \pause

  Generating functions enable us to use manipulations to learn more about the sequence.
  
\end{frame}

\begin{frame}{Examples}
  \begin{center}
  \begin{tabular}{c|c}
    \hline \\
    $1, 6, 15, 20, 15, 6, 1$ & $1 + 6x + 15x^2 + 20x^3 + 15x^4 + 6x^5 + x^6 = (1+x)^6$ \\ 
    \\ \hline \\
    $1, 1, 1, 1, \dots$ & $\displaystyle 1 + x + x^2 + x^3 + \cdots = \frac{1}{1-x}$ \\
    \\ \hline \\
    $0, 1, 1, 2, 3, 5, 8, 13, \dots$ & $x + x^2 + 2x^3 + 3x^4 + 5x^5 + 8x^6 + 13x^7 + \cdots$ \\
    \\ \hline
  \end{tabular}
  \end{center}
  
\end{frame}

\begin{frame}{Size}
  Often, the terms $x^k$ of a generating function correspond to something of \textit{size} $k$. \pause

  Consider the generating function from the Binomial Theorem:
  \[\binom{n}{0} + \binom{n}{1}x + \binom{n}{2}x^2 + \cdots + \binom{n}{n}x^n = (1+x)^n.\] \pause
  The $x^k$ term has coefficient $\displaystyle \binom{n}{k}$. Given a set with $n$ elements, this coefficient is the number of ways to choose a subset of size $k$.
\end{frame}

\section{Counting}
\frame{\sectionpage}

\begin{frame}{Coin Tosses} 
  \textbf{\textcolor{sigma@mainblue}{Q}}:
  Flip 4 coins. How many ways are there to get 2 heads? \pause

  The answer is $\binom{4}{2} = 6$, but let's see how generating functions can be used. \pause

  Consider a single coin -- it either gives us 1 tail (no heads), or 1 head. \pause
  There's only one way each of these can occur,
  so the generating function for a single coin is $1 \cdot x^0 + 1 \cdot x^1 = 1 + x$. \pause

  Here $x$ represents a head -- the coefficient of $x^1$ is the number of ways to get 1 head with a single coin.
  The "size" in this problem is the total number of heads -- the $x^k$ term will correspond to getting $k$ heads.
\end{frame}

\begin{frame}{Combining Generating Functions}
  Now let's combine 2 coins by multiplying their generating functions: \pause
  \[(\textcolor{red}{1}+\textcolor{blue}{x}) (\textcolor{red}{1}+\textcolor{blue}{x})
  = \textcolor{red}{1} \cdot \textcolor{red}{1}
  + \textcolor{red}{1} \cdot \textcolor{blue}{x}
  + \textcolor{blue}{x} \cdot \textcolor{red}{1}
  + \textcolor{blue}{x} \cdot \textcolor{blue}{x} = 1 + 2x + x^2.\]
  What does multiplying mean here? \pause
  \begin{itemize}
    \item $\textcolor{red}{1} \cdot \textcolor{red}{1} = x^0$ corresponds to TT (no heads) \pause
    \item $\textcolor{red}{1} \cdot \textcolor{blue}{x} = x^1$ corresponds to TH (1 head) \pause
    \item $\textcolor{blue}{x} \cdot \textcolor{red}{1} = x^1$ corresponds to HT (1 head) \pause
    \item $\textcolor{blue}{x} \cdot \textcolor{blue}{x} = x^2$ corresponds to HH (2 heads) \pause
  \end{itemize}
  Each way to get $k$ heads contributes a term of $x^k$ to the product.
  So the coefficient of $x^k$ in the product is the total number of ways to get $k$ heads!
  \pause

  For 2 coins, there are 2 ways to get 1 head (coefficient of $x$ in $1 + 2x + x^2$).
\end{frame}

\begin{frame}{More Coins}
  If we have 4 coins, simply multiply the generating functions for each:
  \[ (1+x) (1+x) (1+x) (1+x) = (1+x)^4 = 1 + 4x + 6x^2 + 4x^3 + 1.\]
  \pause
  The coefficient of $x^2$ (6) is our answer -- the number of ways to get 2 heads.
  \pause

  More generally, if we flip $n$ coins, the number of ways to get $k$ heads is
  the coefficient of $x^k$ in $(1+x)^n$, which is $\binom{n}{k}$.
\end{frame}

\begin{frame}{Changemaking}
  \textbf{\textcolor{sigma@mainblue}{Q}}: How many ways can we make a dollar with (unlimited) pennies, nickels, and dimes?
  \pause

  The generating function for the amounts we can make in pennies is
  \[1 + x + x^2 + x^3 + \cdots.\]
  Here our "size" is the total money -- the coefficient of $x^k$ is the number of ways to make $k$ cents.

  For pennies, the coefficients are all 1, because we can make $k$ cents in exactly one way: use $k$ pennies.

\end{frame}

\begin{frame}{Changemaking}
  For nickels and dimes, the generating functions are
  \[1 + x^5 + x^{10} + \cdots, \qquad 1 + x^{10} + x^{20} + \cdots.\]
  \pause
  Putting it all together, our generating function using pennies, nickels, and dimes is
  \begin{align*}
    F(x) &= (1 + x + x^2 + \cdots)(1 + x^5 + x^{10} + \cdots)(1 + x^{10} + x^{20} + \cdots) \\
    &= \frac{1}{1-x} \cdot \frac{1}{1-x^5} \cdot \frac{1}{1-x^{10}}.
  \end{align*}
  \pause
  The answer for making a dollar is the coefficient of $x^{100}$.
  This is nontrivial to find, but there are ways \cite{gasarch2014ways}, which we won't go into here.
\end{frame}

\begin{frame}{Biased Coin Tosses}
  \textbf{\textcolor{sigma@mainblue}{Q}}: There are $n$ coins $C_1, C_2, \dots, C_n$.
  For each $k$, coin $C_k$ is biased so that, when tossed, it has probability $\tfrac{1}{(2k+1)}$ of falling heads.
  If the $n$ coins are tossed, what is the probability that the number of heads is odd? (Putnam 2001)
  \pause
  
  Consider coin $C_1$.
  It has a $\frac{1}{3}$ chance of being heads, and a $\frac{2}{3}$ chance of being tails.
  \pause

  The "size" in this problem is the total number of heads.
  For coin $C_1$, we can write the \textit{probability} generating function
  $f_1(x) = \frac{1}{3}x + \frac{2}{3}$.

  In other words, $C_1$ has a $\frac{1}{3}$ chance of contributing 1 to the total number of heads, and a $\frac{2}{3}$ chance of contributing 0.
\end{frame}

\begin{frame}{Combining the Coins}
  The generating function for coin $C_k$ is $f_k(x) = \frac{1}{2k+1} x + \frac{2k}{2k+1}$.
  \pause

  Then the probability generating function for all coins is the product of the individual generating functions for each coin:
  \begin{align*}
  F_n(x)
  % &= f_1(x) f_2(x) \cdots f_n(x) \\
  &= \left( \frac{1}{3}x + \frac{2}{3} \right) \left( \frac{1}{5}x + \frac{4}{5} \right)
  \cdots \left( \frac{1}{2n+1}x + \frac{2n}{2n+1} \right).
  \end{align*}
  \pause
  Given $k$, this product produces an $x^k$ term for every possible way to choose $k$ factors to take an $x$ term from, corresponding to choosing $k$ coins to land heads.
  The probabilities for each coin are multiplied, producing the probability that $k$ specific coins land on heads.

  The overall coefficient of $x^k$ is then the sum of all these probabilities over all combinations of $k$ heads. This is the probability of getting $k$ heads!
\end{frame}

\begin{frame}{The Generating Function}
  We have
  \begin{align*}
    F_n(x) &= \left( \frac{1}{3}x + \frac{2}{3} \right) \left( \frac{1}{5}x + \frac{4}{5} \right)
    \cdots \left( \frac{1}{2n+1}x + \frac{2n}{2n+1} \right) \\
    &= \frac{(x+2) (x+4) \cdots (x+2n)}{3 \cdot 5 \cdots (2n+1)} \\
    &= p_0 + p_1 x + p_2 x^2 + \cdots + p_n x^n,
  \end{align*}
  where $p_k$ is the probability of getting $k$ heads.
  \pause
  
  Expanding the generating function and finding each coefficient is difficult.
  But we actually only want the probability that the total number of heads is odd:
  \[ p_1 + p_3 + p_5 + \cdots .\]
\end{frame}

\begin{frame}{The Generating Function}
  A generating function is still a function of a variable. So what happens if we plug things in?
  \pause
  
  If we plug in $x = 1$, we get
  \[ F_n(1) = p_0 + p_1 + p_2 + \cdots + p_n
  = \frac{(1+2)(1+4) \cdots (1+2n)}{3 \cdot 5 \cdots (2n+1)}
  = 1.\]
  \pause
  That seems useful, but we only want the sum of every other coefficient.
  Let's try $x = -1$:
  \begin{align*} F_n(-1) &= p_0 - p_1 + p_2 - \cdots \pm p_n \\
  &= \frac{(-1+2)(-1+4) \cdots (-1+2n)}{3 \cdot 5 \cdots (2n-1) (2n+1)}
  = \frac{1}{2n+1}.
  \end{align*}
\end{frame}

\begin{frame}{Filtering Oddities}
  We have the sum and alternating sum of the coefficients.
  Subtracting them will cancel the terms we don't want:
  \begin{align*}
    F_n(1) - F_n(-1)
    &= 2p_1 + 2p_3 + 2p_5 + \cdots \\
    &= 1 - \frac{1}{2n+1} = \frac{2n}{2n+1}.
  \end{align*}
  \pause
  So $p_1 + p_3 + p_5 + \cdots = \frac{n}{2n+1}$.
  That's our answer -- the probability of getting an odd number of heads upon rolling the coins $C_1, \dots, C_n$.
\end{frame}

\begin{frame}{Roots of Unity Filter}
  Plugging in 1 and $-1$ is a common way to filter out every 2nd term.
  \pause

  This technique can be extended to filter out every $k$th term, by plugging in the $k$th roots of unity (complex solutions to $x^k = 1$).
  \pause

  If you're interested, I highly recommend this 3Blue1Brown video:
  \href{https://youtu.be/bOXCLR3Wric}{https://youtu.be/bOXCLR3Wric}.

  It nicely presents a solution using generating functions and the roots of unity filter to the following problem: 
  \newline

  \begin{quote}
    Find the number of subsets of $\{1, 2, 3, 4, 5, \dots, 2000\}$,
    the sum of whose elements is divisible by 5.
  \end{quote}
\end{frame}

\section{Binary Trees}
\frame{\sectionpage}

\subsection{Enumeration}
\frame{\subsectionpage}

\begin{frame}{Enumerating Binary Trees}
  \textbf{\textcolor{sigma@mainblue}{Q}}: How many different binary trees have $n$ nodes? \cite{algoanalysis} \pause

  \begin{itemize}
    \item For $n = 0$, there is 1 tree -- the empty tree \pause
    \item For $n = 1$, there is 1 tree -- a single node \pause
    \item For $n = 2$, there are 2 trees -- a node with a left child, or a node with a right child \pause
    \item For $n = 3$, there are 5 trees:
  \end{itemize}


  \begin{adjustbox}{valign=t}
    \begin{forest} [ [] [] ] \end{forest}
  \end{adjustbox} \hspace{1cm}
  \begin{adjustbox}{valign=t}
    \begin{forest} [ [[] [, phantom]] [, phantom]] \end{forest}
  \end{adjustbox} \hspace{1cm}
  \begin{adjustbox}{valign=t}
    \begin{forest} [ [[, phantom] []] [, phantom]] \end{forest}
  \end{adjustbox} \hspace{1cm}
  \begin{adjustbox}{valign=t}
    \begin{forest} [ [, phantom] [[] [, phantom]] ] \end{forest}
  \end{adjustbox} \hspace{1cm}
  \begin{adjustbox}{valign=t}
    \begin{forest} [ [, phantom] [[, phantom] []] ] \end{forest}
  \end{adjustbox}
  
\end{frame}

\begin{frame}{The Generating Function}
  Let $\TT$ be the set of all binary trees.
  Given a tree $t \in \TT$, let $\abs{t}$ be the number of nodes in $t$.
  We want the number of trees in $\TT$ with $n$ nodes (trees of "size" $n$). \pause

  Consider the generating function
  \[T(x) = \sum_{t \in \TT} x^{\abs{t}}.\]
  In other words, each tree contributes a single term to the sum, which is determined by the number of nodes in that tree.
  \pause

  For example, the 3-node trees on the previous slide would each contribute an $x^3$ term to $T(x)$.
  Since there are exactly 5 trees with 3 nodes, the coefficient of $x^3$ in $T(x)$ is $5x^3$.
\end{frame}

\begin{frame}{The Generating Function}
  So we have the generating function
  \[T(x) = \sum_{t \in \TT} x^{\abs{t}} = 1 + x + 2x^2 + 5x^3 + \cdots = \sum_{n \ge 0} T_n x^n.\] \pause
  The coefficients $T_n$ represent the number of binary trees with $n$ nodes.
  We want to find an expression for $T_n$.
  \pause

  To use this generating function, we exploit the recursive structure of binary trees.
\end{frame}

\begin{frame}{Recursive Structure}
  A binary tree $t$ is either:
  \begin{itemize}
    \item Empty with 0 nodes, or
    \item A single node with a left subtree $t_L$ and a right subtree $t_R$ (binary trees) -- with $1 + \abs{t_L} + \abs{t_R}$ nodes total \pause
    \begin{itemize}
      \item In this case, each tree $t$ uniquely corresponds to a pair of trees $(t_L, t_R)$
    \end{itemize}
  \end{itemize} \pause

  Using this, we can write our generating function as follows:
  \begin{align*}
    T(x) &= \sum_{t \in \TT} x^{\abs{t}}
    = 1 + \sum_{t_L \in \TT} \sum_{t_R \in \TT} x^{\abs{t_L} + \abs{t_R} + 1} \\
  \end{align*}
\end{frame}

\begin{frame}{A Functional Equation}
  Then,
  \begin{align*}
    T(x) &= 1 + \sum_{t_L \in \TT} \sum_{t_R \in \TT} x^{\abs{t_L} + \abs{t_R} + 1} \\
    \onslide<+->{&= 1 + x \sum_{t_L \in \TT} x^{\abs{t_L}} \sum_{t_R \in \TT} x^{\abs{t_R}} \\}
    \onslide<+->{&= 1 + xT(x)^2.}
  \end{align*} \pause
  Solving for $T(x)$ using the quadratic formula,
  \[T(x) = \frac{1 \pm \sqrt{1 - 4x}}{2x}.\]
  Since $T(0) = 1$, we take the minus sign.
\end{frame}

\begin{frame}{Binomial Expansion}
  We need to expand our expression for $T(x)$ to extract coefficients.
  \pause

  According to an extended form of the binomial theorem,
  \begin{align*}
    (1+y)^{1/2} = 1 + \binom{1/2}{1} y + \binom{1/2}{2} y^2 + \binom{1/2}{3} y^3 + \cdots
    = \sum_{k=0}^{\infty} \binom{1/2}{k} y^k.
  \end{align*}
  where we can generalize the definition of $\displaystyle \binom{n}{k}$ to real $n$ as follows:
  \[\binom{n}{k} = \frac{n(n-1) \cdots (n-k+1)}{k!}.\]
\end{frame}

\begin{frame}{Extracting Coefficients}
  Using this expansion with $y = -4x$,
  \begin{align*}
    T(x) &= \frac{1 - \sqrt{1 - 4x}}{2x} \\
    &= \frac{1 - \left( 1 + \binom{1/2}{1}(-4x) + \binom{1/2}{2} (-4x)^2 + \binom{1/2}{3} (-4x)^3 + \cdots \right)}{2x}
  \end{align*} \pause
  After some simplification, we find that the coefficient of $x^n$ is
  \[T_n = \frac{1}{n+1} \binom{2n}{n}.\]
  This is the number of binary trees with $n$ nodes!
  \pause
  Note: The numbers $T_n$ are called the \textit{Catalan numbers}.
\end{frame}

\subsection{Path Length}
\frame{\subsectionpage}

\begin{frame}{Path Length}
  Given a binary tree $t$, its \textbf{path length} $\pi(t)$ is the sum of the lengths of the paths from the root to each node in the tree.
  \pause
  \begin{minipage}{0.7 \textwidth}
    For example, this binary tree has path length
    
    $0 + 1 + 1 + 2 + 3 = 7$.
  \end{minipage}%
  \begin{minipage}{0.3 \textwidth}
    \centering
    \begin{forest} [ 0 [1 [2 [, phantom] [3]] [, phantom]] [1] ] \end{forest}
  \end{minipage}

  \hspace{1cm} \pause

  If we divide the path length by the number of nodes, we get the average distance from the root to a node in the tree.
  This is useful for analyzing algorithms that involve searching for nodes in a tree.
\end{frame}

\begin{frame}{Path Length in Random Trees}
  \textbf{\textcolor{sigma@mainblue}{Q}}: What is the expected path length of a random binary tree with $n$ nodes? \cite{algoanalysis}
  
  Here, "random" means that each of the $T_n$ binary trees with $n$ nodes has an equal probability of being selected (\textit{random binary Catalan trees}). \pause
  \begin{itemize}
    \item This is often used in compiler design for parse trees for expression evaluation.
    However, other models of random trees exist.
  \end{itemize} \pause
  In other words, over all binary trees with $n$ nodes, what is the average path length?

  We'll answer this question using a similar approach to enumerating binary trees.
  % But this time, we'll need an extra variable in the generating function,
  % because not only do we have to keep track of the tree size, we also have to keep track of the path length.
\end{frame}

% Change to cumulative approach
% \begin{frame}{The Bivariate Generating Function}
%   Recall that $\TT$ is the set of all binary trees.

%   Define the \textit{bivariate} generating function
%   \[P(u, x) = \sum_{t \in \TT} u^{\pi(t)} x^{\abs{t}}
%   = \sum_{n \ge 0} \sum_{k \ge 0} p_{nk} u^k x^n.\]

%   In the middle is the generating function written summing tree-by-tree.
%   On the right, all the like terms are collected, and we sum over each term.

%   The coefficients $p_{nk}$ tell us the number of trees with $n$ nodes and path length $k$.
%   If we know how many trees there are of each path length, we can find the average path length.
% \end{frame}

\begin{frame}{The Generating Function}
  Recall that $\TT$ is the set of all binary trees.

  Define the \textit{cumulative} generating function
  \[C_T(x) = \sum_{t \in \TT} \pi(t) x^{\abs{t}}
  = \sum_{n \ge 0} C_{n} x^n.\] \pause
  In the middle is the generating function written summing tree-by-tree.
  On the right, all the like terms are collected, and we sum over each term.

  The coefficients $C_{n}$ tell us the total path length over all trees with $n$ nodes.
\end{frame}

\begin{frame}{Example}
  Here are the five 3-node trees from earlier, and their contribution to $C_T(x)$:
  \begin{figure}
    \begin{subfigure}[b]{0.18\linewidth}
      \begin{forest} [ 0 [1 [, phantom] [, phantom]] [1] ] \end{forest}
      \centering
      \caption*{$2 x^3$}
    \end{subfigure}
    \begin{subfigure}[b]{0.18\linewidth}
      \begin{forest} [ 0 [1 [2] [, phantom]] [, phantom]] \end{forest}
      \centering
      \caption*{$3 x^3$}
    \end{subfigure}
    \begin{subfigure}[b]{0.18\linewidth}
      \begin{forest} [0 [1 [, phantom] [2]] [, phantom]] \end{forest}
      \centering
      \caption*{$3 x^3$}
    \end{subfigure}
    \begin{subfigure}[b]{0.18\linewidth}
      \begin{forest} [0 [, phantom] [1 [2] [, phantom]] ] \end{forest}
      \centering
      \caption*{$3 x^3$}
    \end{subfigure}
    \begin{subfigure}[b]{0.18\linewidth}
      \begin{forest} [0 [, phantom] [1 [, phantom] [2] ]] \end{forest}
      \centering
      \caption*{$3 x^3$}
    \end{subfigure}
  \end{figure}
  The coefficient corresponds to path length;
  the exponent of $x$ corresponds to the number of nodes.
  
\end{frame}

\begin{frame}{Recursive Definition of Path Length}
  % \begin{figure}
  % \begin{subfigure}[t]{0.2 \textwidth}
  %   \centering
  %   \begin{forest} [ 0 [1, fill=green [2, fill=green [, phantom] [3, fill=green]] [, phantom]] [1, fill=yellow] ] \end{forest}
  % \end{subfigure}
  % \begin{subfigure}[t]{0.1 \textwidth}
  %   $\implies$
  % \end{subfigure}
  % \begin{subfigure}[t]{0.2 \textwidth}
  %   \centering
  %   \begin{forest}
  %     [0, fill=green [1, fill=green [, phantom] [2, fill=green]] [, phantom]]
  %   \end{forest}
  %   \caption*{$\pi(t_L) = 3$}
  % \end{subfigure}
  % \begin{subfigure}[t]{0.2 \textwidth}
  %   \centering
  %   \begin{forest}
  %     [0, fill=yellow]
  %   \end{forest}
  % \end{subfigure}
  % \end{figure}

  Consider the tree $t$ from earlier; its subtrees $t_L$ and $t_R$ are colored.
  % Can we write $\pi(t)$ in terms of $t_L$ and $t_R$?
  \vspace{0.2cm}

  \begin{minipage}{0.5 \textwidth}
  \begin{forest} [ 0 [1, fill=green, baseline [2, fill=green [, phantom] [3, fill=green]] [, phantom]] [1, fill=yellow] ] \end{forest}
  \quad
  $\implies$
  \quad
  \begin{forest}
    [0, fill=green, baseline [1, fill=green [, phantom] [2, fill=green]] [, phantom]]
  \end{forest}
  \quad
  \begin{forest}
    [0, fill=yellow, baseline]
  \end{forest}
  \quad
  \end{minipage}\pause%
  \begin{minipage}{0.5 \textwidth}
  When $t_L$ is split off from $t$, the "new root" is one level lower.
  \\

  So the depths of the nodes in $t_L$ are one less than the depths of the same nodes of in $t$.
  \\

  The same is true for $t_R$.
  \end{minipage}
  \vspace{0.3cm} \pause

  So $\pi(t) = 0 + \pi(t_L) + \abs{t_L} + \pi(t_R) + \abs{t_R}$.
\end{frame}

\begin{frame}{Using Recursion}
  Now,
  \begin{align*}
    C_T(x) &= \sum_{t \in \TT} \pi(t) x^{\abs{t}} \\
    &= \sum_{t_L \in \TT} \sum_{t_R \in \TT} (\pi(t_L) + \pi(t_R) + \abs{t_L} + \abs{t_R}) x^{\abs{t_L} + \abs{t_R} + 1}
    % &= x \sum_{t_L \in \TT} \sum_{t_R \in \TT} \pi(t_L) x^{\abs{t_L}} x^{\abs{t_R}}
    % + x \sum_{t_L \in \TT} \sum_{t_R \in \TT} \pi(t_R) x^{\abs{t_L}} x^{\abs{t_R}} \\
    % & \quad + x \sum_{t_L \in \TT} \sum_{t_R \in \TT} \abs{t_L} x^{\abs{t_L}} x^{\abs{t_R}}
    % + x \sum_{t_L \in \TT} \sum_{t_R \in \TT} \abs{t_R} x^{\abs{t_L}} x^{\abs{t_R}}
  \end{align*}
  Note that there's no constant term ($1 + \cdots$) since the only tree with 0 nodes (the empty tree) has path length 0, so the coefficient of $x^0$ is 0. \pause

  This expands into 4 parts. The first is
  \begin{align*}
    \sum_{t_L \in \TT} \sum_{t_R \in \TT} \pi(t_L) x^{\abs{t_L} + \abs{t_R} + 1}
    = x \sum_{t_L \in \TT} \pi(t_L) x^{\abs{t_L}} \sum_{t_R \in \TT} x^{\abs{t_R}}
    = x C_T(x) T(x).
  \end{align*}
\end{frame}

\begin{frame}{Expansion}
  The second part is similarly
  \begin{align*}
    \sum_{t_L \in \TT} \sum_{t_R \in \TT} \pi(t_R) x^{\abs{t_L} + \abs{t_R} + 1}
    = x \sum_{t_L \in \TT} x^{\abs{t_L}} \sum_{t_R \in \TT} \pi(t_R) x^{\abs{t_R}}
    = x T(x) C_T(x).
  \end{align*} \pause
  The third part is (recall $T(x) = \displaystyle \sum_{t \in \TT} x^{\abs{t}}$, so $T'(x) = \displaystyle \sum_{t \in \TT} \abs{t} x^{\abs{t}-1}$)
  \begin{align*}
    \sum_{t_L \in \TT} \sum_{t_R \in \TT} \abs{t_L} x^{\abs{t_L} + \abs{t_R} + 1} 
    = x^2 \sum_{t_L \in \TT} \abs{t_L} x^{\abs{t_L} - 1} \sum_{t_R \in \TT} x^{\abs{t_R}}
    = x^2 T'(x) T(x).
  \end{align*} \pause
  Similarly, the fourth part is
  \begin{align*}
    \sum_{t_L \in \TT} \sum_{t_R \in \TT} \abs{t_R} x^{\abs{t_L} + \abs{t_R} + 1} 
    = x^2 \sum_{t_L \in \TT} x^{\abs{t_L}} \sum_{t_R \in \TT} \abs{t_R} x^{\abs{t_R} - 1}
    = x^2 T(x) T'(x).
  \end{align*}
\end{frame}

\begin{frame}{The Functional Equation}
  Putting it all together,
  \begin{align*}
    C_T(x) &= \sum_{t \in \TT} \pi(t) x^{\abs{t}} \\
    &= \sum_{t_L \in \TT} \sum_{t_R \in \TT} (\pi(t_L) + \pi(t_R) + \abs{t_L} + \abs{t_R}) x^{\abs{t_L} + \abs{t_R} + 1} \\
    &= 2x C_T(x) T(x) + 2x^2 T(x) T'(x).
  \end{align*} \pause
  Earlier we found that $T(x) = \displaystyle \frac{1 - \sqrt{1-4x}}{2x}$.
  Solving for $C_T(x)$,
  \begin{align*}
    C_T(x) = \frac{2x^2 T(x) T'(x)}{1 - 2x T(x)}
    = \frac{1}{x} \left( \frac{x}{1 - 4x} - \frac{1 - x}{\sqrt{1 - 4x}} + 1 \right).
  \end{align*}
\end{frame}

\begin{frame}{Coefficients}
  The coefficient of $x^n$ is the total path length over all binary trees with $n$ nodes.

  We want the average path length, so we divide the coefficient by the number of trees with $n$ nodes: $T_n$. \pause

  After some messy expansions, the average path length comes out as
  \begin{align*}
    \frac{(n+1) 4^n}{\binom{2n}{n}} - 3n - 1 = n \sqrt{\pi n} - 3n + O(\sqrt{n}).
  \end{align*}

\end{frame}

\begin{frame}{Other Applications of Generating Functions}
  \begin{itemize}
    \item General trees, permutations, strings/regular expressions, many other combinatorial structures
    for analyzing algorithms \pause
    \item Symbolic method (analytic combinatorics) \pause
    \begin{itemize}
      \item To get the generating function $T(x)$ for binary trees, we can actually just write this, since a tree is either empty or a node and two subtrees:
      \[
      \TT = \{ \square \} + \{ \bullet \} \times \TT \times \TT
      \implies T(x) = 1 + xT(x)^2.\]
    \end{itemize} \pause
    \item Snake Oil method: evaluate terrible sums like
    $f_n = \displaystyle \sum_{k} \binom{n+k}{2k} 2^{n-k}$
    by throwing them in as coefficients of a generating function and swapping the order of summation \pause
    \item Solving recurrences (e.g. finding a formula for Fibonacci numbers)
  \end{itemize}
\end{frame}

% \begin{frame}{Some Text}
%     \begin{itemixe}
%         \item You may want some stuff to appear in a sequence \pause
%         \item Use \textbackslash pause for this \pause
%         \item \textcolor{sigma@mainblue}{colors} \textcolor{sigma@highlightpink}{are} \textcolor{sigma@alertred}{cool}
%     \end{itemize}
% \end{frame}

% \begin{frame}
%   \frametitle{Some Math Mode Testing}
%   % Some fun with LaTeX Math
%   $$\frac{x^2+3}{y^2+7}$$

%   \[
%     \mathcal L_{\mathcal T}(\vec{\lambda})
%     = \sum_{(\mathbf{x},\mathbf{s})\in \mathcal T}
%        \log P(\mathbf{s}\mid\mathbf{x}) - \sum_{i=1}^m
%        \frac{\lambda_i^2}{2\sigma^2}
%   \]

%   $$\int_0^8 f(x) dx$$
% \end{frame}

% % Use \pause to make stuff readable
% % Large walls of text scare the audience, we don't want that
% % Introducing stuff sequentially allows for questions
% \begin{frame}
%   % Alternate syntax for frame titles
%   \frametitle{There Is No Largest Prime Number}
%   % Frames can have subtitles:
%   \framesubtitle{The proof uses \textit{reductio ad absurdum}.}
%   % Some frame content:
%   \begin{thrm}
%     There is no largest prime number.
%   \end{thrm}
%   \begin{pf}
%     \begin{enumerate}
%         \item Suppose $p$ were the largest prime number \pause
%         \item Let $q$ be the product of the first $p$ primes \pause
%         \item Then $q+1$ is not divisible by any of them \pause
%         \item But $q + 1$ is greater than $1$, thus divisible by some prime number not in the first $p$ numbers. \pause
%         \item Thus, there exists a prime larger than $p$.
%     \end{enumerate}
%   \end{pf}
% \end{frame}

% % However, this doesn't work in math mode. It is quite annoying to figure out
% % So just copy this as reference
% % This works for \onslide<> and \item<>
% % Really good read on this: 
% %   https://www.texdev.net/2014/01/17/the-beamer-slide-overlay-concept/
% \begin{frame}{Sequential Math Frames}
%     Here is a sentence \pause
    
%     I shall now carry out some calculations \pause
%     \begin{align*}
%         \onslide<+->{\zeta(s) &= \sum_{n = 1}^\infty \frac{1}{n^s} \\}
%         \onslide<+->{&= \prod_{p \in \text{primes}} \frac{1}{1 - p^{-s}} \\}
%         \onslide<.->{&= \frac{1}{1 - 2^-s} \cdot \frac{1}{1 - 3^-s} \cdots \\}
%         \onslide<+->{&= \frac{1}{\Gamma(s)} \int_0^\infty \frac{x^{s - 1}}{e^x - 1} ~\textrm{d}x\\}
%     \end{align*}
% \end{frame}

% \section{Some Template Slides}
% \frame{\sectionpage}

% % Similar for subsections:
% \subsection{A subsection, Wow}
% % And their pages:
% \frame{\subsectionpage}

% \begin{frame}{Image}
%   % This is how you'd include an image, centered.
%   \begin{center}
%     \includegraphics[width=0.25\textwidth]{sigma.png}
%   \end{center}
% \end{frame}

% \begin{frame}{Side by Side}
%     \includegraphics[width=0.25\textwidth]{sigma.png}\hspace{0.4\textwidth}
%     \includegraphics[width=0.25\textwidth]{sigma.png}
% \end{frame}

% \begin{frame}{Demonstration of algo and nalgo env}
%     \begin{algo}
%     \underline{\textsc{GetRandomNumber}():}\+
%     \\      return $4$   \Comment{chosen by fair dice roll.}
%     \\      \hspace{42.75pt}\Comment{guaranteed to be random.}
%     \end{algo}
    
%     % nalgo has line numbers
%     % only lines with \label{} are numbered
%     \begin{nalgo}[1.3]
%         \underline{\textsc{GetRandomNumber}():}\+
%     \\\label{}  return $4$   \Comment{chosen by fair dice roll.}
%     \\\label{}  \hspace{42.75pt}\Comment{guaranteed to be random.}
%     \end{nalgo}
    
%     Random number generation from~\cite{site:xkcd}. 
%     \quest{
%     \textbackslash cref for line numbers does not work. 
%     If you want to refer to specific line numbers, do it manually
%     }
    
% \end{frame}

% \begin{frame}{}
% \begin{minipage}[c]{0.6\textwidth}
% \begin{nalgo}
% \textul{\textbf{\textsc{AlgorithmP}}$(S[a_0, \dots, a_n])$}:
% \\\label{}  $C[1..n] \gets 0, O[1..n] \gets 1$
% \\\label{}  \textbf{while} \textsc{True}:\+
% \\\label{}      \textsc{print}$(S)$
% \\\label{}      {\color{lightgray}$j \gets n, s \gets 0$}
% \\\label{}      {\color{lightgray}\textbf{A:}~~$q \gets C[j] + O[j]$\+}
% \\\label{}          {\color{lightgray}\textbf{if} $q < 0$: goto \textbf{D}}
% \\\label{}          {\color{lightgray}\textbf{if} $q = j$: goto \textbf{B}}
% \\\label{}          {\color{lightgray}\textsc{swap}$(S, j-C[j]+s, j-q+s)$}
% \\\label{}          {\color{lightgray}$C[j] \gets q$} 
% \\\label{}          {\color{lightgray}continue\-}
% \\\label{}      {\color{lightgray}\textbf{B:}~~\textbf{if} $j=1$:\+\+}
% \\\label{}              {\color{lightgray}break\-}
% \\\label{}          {\color{lightgray}$s \gets s+1$\-}
% \\\label{}    {\color{lightgray}\textbf{D:}~~$O[j] \gets -O[j], j \gets j-1$\+}
% \\\label{}      {\color{lightgray}goto \textbf{A}}
% \end{nalgo}
% \end{minipage}
% \begin{minipage}[c]{0.35\textwidth}
% \begin{itemize}
%     \item Here is an example of annotating an algorithms \pause
%     \item We have grayed out text to highlight what we want to discuss 
% \end{itemize}
% \end{minipage}
% \end{frame}

% % Of course, not everything is in pseudocode
% % You MUST have this [containsverbatim] option
% % https://www.overleaf.com/learn/latex/Code_Highlighting_with_minted
% \begin{frame}[containsverbatim]{Source Code}
%     \begin{minted}
%     [
%     framesep=2mm,
%     baselinestretch=1.2,
%     bgcolor=black,
%     fontsize=\footnotesize,
%     ]
%     {python}
%     def algorithm_g(n):
%         a = [0 for _ in range(n + 1)]
%         while True:
%             curr = a[1 : n + 1][::-1]
%             yield "".join([str(i) for i in curr])
    
%             a[0] = 1 - a[0]
%             j = 1
%             while a[j - 1] != 1:
%                 j += 1
%             if j == n + 1:
%                 return
%             a[j] = 1 - a[j]
%     \end{minted}
% \end{frame}


% \begin{frame}{Theorems and Lemmas}
%     \begin{lem}
%         The map $\C \times \pqty{\C \setminus \set{0}} \to \C$, $(z, w) \mapsto z / w$ is $C^{\infty}$.
%     \end{lem}
%     \begin{pf}
%         To see this, we identify $\C$ with $\R^{2}$ where $a + bi = (a, b)$.
%         Thus our map is now
%         \begin{align*}
%             \pqty{\R^{2}} \times \pqty{\R^{2} \setminus \set{(0, 0)}} &\to \R^{2} \\
%                             ((a, b), (c, d)) &\mapsto \pqty{\frac{ac + bd}{c^{2} + d^{2}}, \frac{bc - ad}{c^{2} + d^{2}}}
%         \end{align*}
%         which is well defined since at least one of $c$ or $d$ is nonzero.
%         It is simple to verify that this map is equivalent to the original one.
%         Since this new map is $C^{\infty}$ in each component, we have that it is $C^{\infty}$ overall.
%     \end{pf}
% \end{frame}

% \section{Conclusion}
% \frame{\sectionpage}

% Asking questions is fun but we should answer some first
\begin{frame}{}
      \begin{center}
    {\color{sigma@mainblue} \LARGE Questions?}
  \end{center}
\end{frame}

% \begin{frame}{Questions!}
%     \begin{itemize}
%         \item How many zeros of $\zeta(s) = \sum\limits_{i = 1}^\infty \frac{1}{n^s} = \prod\limits_{p \text{ prime}} \frac{1}{1 - p^{-s}}$ have real part equal to $\frac{1}{2}$?
%         \item Find a closed form to the following recurrence:
%         $$
%             f(n) = \begin{cases} 
%                 f\pqty{\frac{n}{2}} & n \text{ even} \\
%                 f{3n + 1} & \text{otherwise}                
%             \end{cases}
%         $$
%     \end{itemize}
% \end{frame}

% Quotes are fun, find some to use!
\font\eightss=cmssq8
\font\eightssi=cmssqi8
\newcommand\quoteAuthorDate[3]{\begingroup
  \baselineskip 10pt
  \parfillskip 0pt
  \interlinepenalty 10000 % not needed in example
  \leftskip 0pt plus 40pc minus \parindent
  \let\rm=\eightss
  \let\sl=\eightssi
  \everypar{\sl}#1\par
  \nobreak\smallskip
  \noindent\rm--- #2\unskip\enspace(#3)\par
  \endgroup}
% If someone can figure out how to horizontally center this and make the text bigger that'd be cool
\begin{frame}
    \begin{center}
        \item \quoteAuthorDate{A generating function is a device somewhat similar to a bag. Instead of carrying many little objects detachedly, which could be embarrassing, we put them all in a bag, and then we have only one object to carry, the bag.}{George Polya}{\textcolor{sigma@mainblue}{1954}}
    \end{center}
\end{frame}

% Remove this slide if you came up with all the material yourself
\begin{frame}[allowframebreaks]{Bibliography}
    \tiny
    \bibliography{refs}
    \bibliographystyle{alpha}
    
    \nocite{algoanalysis, ac, gfology, MITNotes, gasarch2014ways}
    
\end{frame}

% @misc{site:xkcd,
%   author={Randall Munroe},
%   title={RFC 1149.5 specifies 4 as the standard IEEE-vetted random number.},
%   howpublished={\url{https://xkcd.com/221/}},
%   year={2007},
%   note={Accessed: 01-09-2023},
% }
% Moved from refs.bib

\end{document}