% Mirror: https://github.com/SIGma-UIUC/presentation-format
% --------------------------------------------------------------------
% This is a simple Beamer document that uses beamerthemesigma.sty
% Reading the comments should help you create a presentation even if
% you've never used Beamer before.
% --------------------------------------------------------------------

% Set our document class to Beamer
\documentclass[aspectratio=169]{beamer}
% \documentclass[aspectratio=169, handout]{beamer}
% Add handout option to ignore pauses

% From Jeff E
\usepackage{algo}
% Some more macros
\usepackage{sigmastyle}


% Set a title
\title{Some Basic Turing-Complete Systems}

% Whoever worked on the presentation:
\author{Parth Deshmukh}

% Date looks ugly, so leave blank
\date{}

% An institute name, if you're so inclined
% \institute{University of Illinois Urbana-Champaign}

% Use the SIGma theme for this Beamer presentation
\usetheme{sigma}
% --------------------------------------------------------------------

% Begin document
\begin{document}

% Beamer calls each slide a "frame", defined within the environment:
% \begin{frame}
%   <frame content here>
% \end{frame}

% This frame is just the title.
\begin{frame}
\titlepage
\end{frame}

% A frame with the table of contents.
% This frame's title is "Outline".
\begin{frame}{Outline}
  \tableofcontents
\end{frame}

% Start a section: *sections* (subsections, etc.) are what show up in the TOC.
\section{Turing-completeness}
% Section pages can be printed thus:
\frame{\sectionpage}
% There's a way to automate this, see:
% https://tex.stackexchange.com/questions/178800/creating-sections-each-with-title-pages-in-beamers-slides/178803

\begin{frame}{What are some Turing-complete systems?}
    \begin{itemize}
        \pause
        \item Turing machines \pause
        \item Lambda calculus, formal grammars and languages \pause
        \item Most programming languages (all of the useful ones) \pause
        \item Electrical circuits with NAND gates (or NOT and AND/OR gates) \pause
        \item TeX, SQL (more on that later!) \pause
        \item (kind of) Minecraft, Cities: Skylines, Dwarf Fortress, Magic: The Gathering, you could probably add to this list
        % \item \textcolor{sigma@mainblue}{colors} \textcolor{sigma@highlightpink}{are} \textcolor{sigma@alertred}{cool}
    \end{itemize}
\end{frame}

\begin{frame}{What does it mean to be Turing-complete?}{It runs DOOM (eventually)}
    \begin{itemize}
        \item A system of computation is Turing-complete if it can simulate any Turing machine \pause
        \item Turing machines are \textit{abstract computers} \pause
        \begin{itemize}
            \item Memory is an infinite tape \pause
            \item Computation is performed by following a table of states \pause
            \item Importantly, Turing machines have \textit{infinite time and space} \pause
        \end{itemize}
        \item To show something is Turing-complete, we can show that it simulates an arbitrary Turing machine
    \end{itemize}
\end{frame}

\begin{frame}{Why Turing machines?}{Some philosophy}
    \begin{itemize}
        \item We want to define what it means to compute some function $f: \mathbb{N} \rightarrow \mathbb{N}$ for any input $x$ \pause
        \item Turns out, two important ways to define this are equivalent: \pause
        \begin{enumerate}
            \item Turing-computable functions can be run on a Turing machine \pause
            \item $\lambda$-computable functions can be written (and solved) in the language of lambda calculus: composing building block functions \pause
        \end{enumerate}
        \item The \textcolor{sigma@mainblue}{Church-Turing thesis}\cite{sep-church-turing} posits that these definitions are enough to define all computable functions - there is no challenge to this \pause
        \item The efficient Church-Turing thesis, in comparison, posits these define all \textit{realistically computable} functions - this is up to debate
    \end{itemize}
\end{frame}

% However, this doesn't work in math mode. It is quite annoying to figure out
% So just copy this as reference
% This works for \onslide<> and \item<>
% Really good read on this: 
%   https://www.texdev.net/2014/01/17/the-beamer-slide-overlay-concept/
% \begin{frame}{Sequential Math Frames}
%     Here is a sentence \pause
%     
%     I shall now carry out some calculations \pause
%     \begin{align*}
%         \onslide<+->{\zeta(s) &= \sum_{n = 1}^\infty \frac{1}{n^s} \\}
%         \onslide<+->{&= \prod_{p \in \text{primes}} \frac{1}{1 - p^{-s}} \\}
%         \onslide<.->{&= \frac{1}{1 - 2^-s} \cdot \frac{1}{1 - 3^-s} \cdots \\}
%         \onslide<+->{&= \frac{1}{\Gamma(s)} \int_0^\infty \frac{x^{s - 1}}{e^x - 1} ~\textrm{d}x\\}
%     \end{align*}
% \end{frame}

\section{Tag systems}
\frame{\sectionpage}

\begin{frame}{Motivations}
    \begin{itemize}
        \item It can be useful to find simple Turing-complete systems to study \pause
        \item Most frequently, it's easiest to show a system of interest is Turing-complete through one or more intermediary steps \pause
        \item We will demonstrate this through \textit{cyclic tag systems}
    \end{itemize}
\end{frame}

\begin{frame}{Tag systems}
\begin{defn}
    A \textcolor{sigma@mainblue}{$m$-tag system} is specified via a triplet $(m, A, P):$
    \begin{enumerate}
        \item $m:$ the deletion number, $m \geq 1$
        \item $A:$ the alphabet, a set of symbols
        \item $P:$ a set of production rules, essentially a function $P: A \rightarrow A^*$
    \end{enumerate}
    \pause
    \vspace{8pt}
    A tag system takes as input a word $W$ in $A^*$. Let $w$ be the first letter in $W$. At each step, it removes the first $m$ letters and concatenates $P(w)$ to the end of $W$, re-reading $w$ as it goes.
    \vspace{4pt}
    As the tag system iterates, we denote $W$ to be the \textcolor{sigma@mainblue}{register.}
\end{defn}
\end{frame}

\begin{frame}{Halt states}
    Unlike Turing machines, tag systems can halt in two ways: \pause
    \begin{enumerate}
        \item Either the machine halts if the register goes below a set size... \pause
        \item ...or we include a halting symbol $h \in A$ that halts the machine when it reaches the head (when $P(h)$ is called).
    \end{enumerate}
\end{frame}

\begin{frame}{A worked example}
    This example is from \cite{DEMOL200892} and calculates the Collatz/hailstone sequence of the given number.
\begin{gather*}
    m = 2 \\
    A = \{a, c, y\} \\
    P = \begin{cases}
        a &\rightarrow cy \\
        c &\rightarrow a \\
        y &\rightarrow aaa
    \end{cases}
\end{gather*}
A number $n$ is represented with repeating $a$ $n$ times. We'll start with $n = 5$, so the starting word is $aaaaa.$ We halt when our register has only one symbol in it.
\end{frame}

\begin{frame}{Doing the worked example}
\begin{itemize}
    \item First letter is $a$: we append $cy,$ and remove 2 letters from the front, so $aaaaa \rightarrow aaacy$ \pause
    \item Similarly, $aaacy \rightarrow acycy \rightarrow ycycy$ \pause
    \item Now, the first letter is $y$: append $aaa$, so $ycycy \rightarrow ycyaaa \rightarrow yaaaaaa \rightarrow aaaaaaaa = 8$ \pause
    \item We've computed two steps in one: $5 \rightarrow 3(5) + 1 \rightarrow \frac{3(5) + 1}{2} = 8$ \pause
    \item If you continue from here, you'll see the production rule $c \rightarrow a$ is used to divide our number by 2 \pause
    \item We halt when our register has one symbol, which ends up being $a \implies$ we halt iff the Collatz sequence terminates
\end{itemize}
\end{frame}

\begin{frame}{Tag systems are Turing-complete}{(trust me bro)}
\begin{thrm}
    2-tag systems are Turing-complete\cite{10.1145/321203.321206}.
\end{thrm}
\begin{pf}
    Cocke and Minsky show a construction to make a tag system simulating a given Turing machine. The construction is too complicated to cover here; it uses 17 symbols per state in the Turing machine it wants to simulate, and works by representing Turing machine states with special words.
\end{pf}
\end{frame}

\section{Cyclic tag systems}
\frame{\sectionpage}

\begin{frame}{Can we go simpler?}{Tag systems are tough to work with!}
\pause
    \begin{itemize}
        \item The answer is yes, actually \pause
        \item \textcolor{sigma@mainblue}{Cyclic tag systems} can simulate any tag system and have much simpler specifications, making them way easier to work with \pause
        \item The downside is that $P(x)$ has a different function signature \pause
        \item Halting is also more confusing
    \end{itemize}
\end{frame}

\begin{frame}{Cyclic tag systems}
    \begin{defn}
        A \textcolor{sigma@mainblue}{cyclic tag system} is a modified tag system where $m=1$ and $A=\{0, 1\}.$ The set of production rules is a function $P: \mathbb{Z}_n \rightarrow A^*$ where $n$ is the number of production rules.
    \end{defn}
    \pause
    A cyclic tag system takes as input a word $W$ in $A^*$ and additionally tracks its step number $k \geq 1$. At each step, it removes the first letter $w,$  concatenating $P(k)$ to the end of $W$ only if $w=1.$
\end{frame}

\begin{frame}{Some notes on the definition}
    \begin{itemize}
        \item This is a non-standard but equivalent definition; usually $P$ is taken to be an ordered list of words instead of a function \pause
        \item Why this definition? Since choice of $P$ \textit{exactly specifies a cyclic tag system!}\pause
        \begin{itemize}
            \item Therefore, there are as many cyclic tag systems with $n$ rules as there are functions $\mathbb{Z}_n \rightarrow A^*$ \pause
            \item Also, this definition is easier to code: removes one pointer
        \end{itemize} \pause
        \item We can rewrite the updating rule at each step compactly: $$(W, k) \rightarrow (Wq, k+1)$$
        $$q = \begin{cases}
            \epsilon &\text{if } w = 0 \\ P(k) &\text{if } w = 1
        \end{cases}$$
    \end{itemize}
\end{frame}

\begin{frame}{Halt states}
    Cyclic tag systems also have two halts: \pause
    \begin{enumerate}
        \item Either the register clears out entirely, or... \pause
        \item ...the system enters an infinitely repeating loop of states
    \end{enumerate} \pause
    What an infinite loop usually would be is actually represented by the register growing to unbounded length! \pause
    \vspace{4pt}
    
    Side note: These two are equivalent; see \url{https://cs.stackexchange.com/a/44931} for a sketch of why.
\end{frame}

\begin{frame}{Are cyclic tag systems Turing-complete?}
    \pause
    \begin{itemize}
        \item Remember how we saw that tag systems are Turing-complete? \pause
        \item The heavy lifting's already done! Just show that we can simulate tag systems with cyclic tag systems \pause
        \item Thankfully, it's a simple construction: just one-hot encode the alphabet of a tag system \pause
        \begin{itemize}
            \item \textcolor{sigma@mainblue}{One-hot encoding} is a common trick: if you have $n$ categories, you map each category to one of the standard basis vectors in $\mathbb{R}^n$ to keep them both equally weighted and orthogonal to each other
        \end{itemize}
    \end{itemize}
\end{frame}

\begin{frame}{Proving cyclic tag systems are Turing-complete}
\begin{itemize}
    \item Suppose we're given a tag system $(m, A = \{a_1, a_2, \dots a_n\}, P).$ We assume $A$ is ordered; order it if not \pause
    \item First, one-hot encode the letters: $a_1 = 100\dots, a_2 = 010\dots,$ so on until $a_n = \dots001$ \pause
    \item Now create the new production rules $\Hat{P}: \mathbb{Z}_{(mn)} \rightarrow \{0, 1\}^*$ as follows: \pause
    \begin{itemize}
        \item For $1 \leq k \leq n,$ $\Hat{P}(k) = P(a_k)$ in the one-hot encoded form. This removes the first letter $w$ and concatenates $P(w)$ onto the register. \pause
        \item For $n < k \leq mn,$ $\Hat{P}(k) = \epsilon.$ This deletes the remaining $m-1$ symbols from the front of the register.
    \end{itemize} \pause
    \item Apply this to Cocke and Minsky's construction to simulate any Turing machine!
\end{itemize}
\end{frame}

\section{Conclusion}
\frame{\sectionpage}

\begin{frame}{So what was the point?}
    \begin{itemize}
        \item I promised a proof that SQL is Turing-complete \pause
        \item We've shown that all we need to do is implement a cyclic tag system within SQL: that's a proof that SQL is Turing-complete! \pause
        \item \url{https://wiki.postgresql.org/wiki/Cyclic_Tag_System} implements a cyclic tag system in \textit{only 28 lines}
    \end{itemize}
\end{frame}

\begin{frame}{So what was the point?}
    \begin{itemize}
        \item Cyclic tag systems are far easier to reason with and implement than directly implementing Turing machines \pause
        \item I've written a cyclic tag system simulator in Haskell: \url{https://github.com/papermango/cyts} \pause
        \begin{itemize}
            \item This is a proof that Haskell is Turing-complete! \pause
            \item It also lets you run any cyclic tag system of your choice on the command line - check it out if that sounds interesting
        \end{itemize} \pause
        \item For fun, go implement a cyclic tag system in any language of your choice - it won't take too long, but you'll have written something that can run DOOM!
    \end{itemize}
\end{frame}

% Asking questions is fun but we should answer some first
\begin{frame}{}
      \begin{center}
    {\color{sigma@mainblue} \LARGE Questions?}
  \end{center}
\end{frame}

% Quotes are fun, find some to use!
\font\eightss=cmssq8
\font\eightssi=cmssqi8
\newcommand\quoteAuthorDate[3]{\begingroup
  \baselineskip 10pt
  \parfillskip 0pt
  \interlinepenalty 10000 % not needed in example
  \leftskip 0pt plus 40pc minus \parindent
  \let\rm=\eightss
  \let\sl=\eightssi
  \everypar{\sl}#1\par
  \nobreak\smallskip
  \noindent\rm--- #2\unskip\enspace(#3)\par
  \endgroup}
% If someone can figure out how to horizontally center this and make the text bigger that'd be cool
\begin{frame}
    \begin{center}
        \item \quoteAuthorDate{Google "hypercomputation."}{SCOTT AARONSON}{\textcolor{sigma@mainblue}{2016}}
    \end{center}
\end{frame}

% Remove this slide if you came up with all the material yourself
\begin{frame}[allowframebreaks]{Bibliography}
    \tiny
    \bibliography{refs}
    \bibliographystyle{alpha}
\end{frame}

\end{document}