% Mirror: https://github.com/SIGma-UIUC/presentation-format
% --------------------------------------------------------------------
% This is a simple Beamer document that uses beamerthemesigma.sty
% Reading the comments should help you create a presentation even if
% you've never used Beamer before.
% --------------------------------------------------------------------

% Set our document class to Beamer
\documentclass[aspectratio=169, handout]{beamer}
% \documentclass[aspectratio=169, handout]{beamer}
% Add handout option to ignore pauses

% From Jeff E
\usepackage{algo}
% Some more macros
\usepackage{sigmastyle}


% Set a title
\title{Estimates in Analytical Number Theory}

% Set a subtitle if you desire
\author{Saurav Chittal}

% Date looks ugly, so leave blank
\date{}

% An institute name, if you're so inclined
% \institute{University of Illinois Urbana-Champaign}

% Use the SIGma theme for this Beamer presentation
\usetheme{sigma}
% --------------------------------------------------------------------

% Begin document
\begin{document}

% Beamer calls each slide a "frame", defined within the environment:
% \begin{frame}
%   <frame content here>
% \end{frame}

% This frame is just the title.
\begin{frame}
\titlepage
\end{frame}

\begin{frame}{The use of bigO notation in Analytic number theory}
    \begin{itemize}
        \item Presumably, You've seen the BigO notation in your CS classes. They are generally used to notify the speed or the size of a program \pause
        \item For example, BFS takes O(V + E) time and hashing algorithms like maps generally take O(n) space.\pause
        \item Similar ideas are used in Analytical Number theory as well. The BigO is used to usually indicate the size of the error. 
        \begin{itemize}
            \item It is very useful too, since you can pull out the bigO out of integrals and limits, more useful that other estimates such as smallO.
            \int O(\log{x}) dx = O(\int \log{x} dx)
        \end{itemize}
    \end{itemize}
\end{frame}

% \begin{frame}{Some Text}
%     \begin{itemize}
%         \item You may want some stuff to appear in a sequence \pause
%         \item Use \textbackslash pause for this \pause
%         \item \textcolor{sigma@mainblue}{colors} \textcolor{sigma@highlightpink}{are} \textcolor{sigma@alertred}{cool}
%     \end{itemize}
% \end{frame}

\begin{frame}
  \frametitle{Some examples}
  % Some fun with LaTeX Math
  The Prime Number Theorem is equivalent to the statement:
  $$\pi(x) = \frac{x}{\log(x)} + o(\frac{x}{\log(x)}))$$
  Where $\pi(x)$ is the number of prime numbers less than or equal to x.\pause

  Knowing $$Li(x) = \int_2^x \frac{1}{\log(t)} dt$$ the current best known estimate is $$\pi(x) = Li(x) + O(x\exp{-(\log(x))^\alpha}); \alpha \in \mathbf{R}, \alpha < \frac{3}{5}$$. \pause

  Finally, the Riemann Hypothesis is equivalent to:
  $$\pi(x) = Li(x) + O(x^{\frac{1}{2} + \epsilon}); \epsilon > 0$$
\end{frame}

% Use \pause to make stuff readable
% Large walls of text scare the audience, we don't want that
% Introducing stuff sequentially allows for questions
\begin{frame}
  % Alternate syntax for frame titles
  \frametitle{Manipulation with BigO notation}
  % Frames can have subtitles:
  \framesubtitle{Proof of Sterling Formula}.
  % Some frame content:
  \begin{thrm}
    S(N) = \sum_{n \le N} \log(n) = N(\log(N) - 1) + \frac{\log(N)}{2} + c + O(\frac{1}{N})
  \end{thrm}
  \begin{pf}
    \begin{enumerate}
        \item We will use special case of Euler's Summation Formula here
        $$\sum_{n \le x} f(n) = \int_1^x f(t) dt + \int_1^x \{t\} f^\prime(t) dt - \{x\}f(x) + f(1)$$
        Here $\{x\}$ indicates the fractional part of x, and $f^\prime(x)$ indicates the derivative of f(x). There isn't enough time for a proof, so assume this formula exists.\pause
        \item Use Euler's Summation formula but with $f(x) = \log(x)$, we note that $\{N\} = 0, f(1) = 0$. Hence $S(N) = I_1(N) + I_2(N)$
    \end{enumerate}
  \end{pf}
\end{frame}

\begin{frame}{Manipulation with BigO notation}
    \begin{enumerate}
      \setcounter{enumi}{2}
      \item $I_1(N) = \int_1^N \log(t) dt = N\log(N) - N + 1$ \pause
      \item $I_2(N) = \int_1^N \frac{\{t\}}{t} dt = \int_1^N \frac{1}{2t} dt + \int_1^N \frac{\{t\} - \frac{1}{2}}{t} dt = \frac{\log(N)}{2} + I_3(N)$\pause
    \end{enumerate}
    Combining these formulas gives us: $$S(N) = N\log(N) - N + 1 - \frac{\log(N)}{2} + I_3(N)$$
\pause
    Now, we use integration by parts on $I_3(N)$
    $$I_3(N) = \left. \frac{R(t)}{t} \right\rvert_{1}^N + \int_1^N \frac{R(t)}{t^2} dt$$
    where $R(N) = \int_1^N \{t\} - \frac{1}{2} dt$
\end{frame}

% However, this doesn't work in math mode. It is quite annoying to figure out
% So just copy this as reference
% This works for \onslide<> and \item<>
% Really good read on this: 
%   https://www.texdev.net/2014/01/17/the-beamer-slide-overlay-concept/
\begin{frame}{Manipulation with BigO notation}
    We note the following:
    \begin{itemize}
        \item $\rho(t) = \{t\} - \frac{1}{2}$ is periodic with period 1 and $\abs{\rho(t)} \le \frac{1}{2}$\pause
        \item $\int_k^{k+1} \rho(t) dt = 0$ for any integer k, because of the way it's structured. \pause 
        \item Finally, $\abs{R(t)} \le \frac{1}{2}$ for any t \pause
    \end{itemize}
    Hence we have that $I_3(N) = \int_1^N \frac{R(t)}{t^2} dt$.
    This specific integral now converges as $N \to \infty$, since $\frac{\abs{R(t)}}{t^2} \le \frac{(\frac{1}{2})}{t^2}$ \pause

    Therefore $$I_3(N) = I - \int_N^\infty \frac{R(t)}{t^2} dt = I - O(\int_N^\infty \frac{1}{t^2} dt) = I - O(\frac{1}{N})$$
    where $I = \int_1^\infty \frac{R(t)}{t^2} dt$, a constant. 
    \qed
\end{frame}

\begin{frame}{Manipulation with BigO notation}
    \framesubtitle{The end is near}
    Recapping what we've done so far, we have shown that:
    $$S(N) = \sum_{n \le N} \log(n) = N\log(N) - N + C - \frac{\log(N)}{2} - O(\frac{1}{N})$$
\pause
    However, we note that $n! = \exp{\sum_{k \le n} \log(k)}$, exponentiating the formula that we just derived we get:

    $$n! = n^ne^{-n}\sqrt{n}e^C(1 + O(\frac{1}{N}))$$ due to the taylor series of exponentials.
    \qed

    Note: We can accurately get the constant value to be $\sqrt{2\pi}$, but that is much more difficult, and will require a course in number theory
\end{frame}

\begin{frame}{}
      \begin{center}
    {\color{sigma@mainblue} \LARGE Questions?}
  \end{center}
\end{frame}

\end{document}