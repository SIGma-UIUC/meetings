% Mirror: https://github.com/SIGma-UIUC/presentation-format
% --------------------------------------------------------------------
% This is a simple Beamer document that uses beamerthemesigma.sty
% Reading the comments should help you create a presentation even if
% you've never used Beamer before.
% --------------------------------------------------------------------

% Set our document class to Beamer
\documentclass[aspectratio=169]{beamer}
%\documentclass[aspectratio=169, handout]{beamer}
% Add handout option to ignore pauses

% From Jeff E
\usepackage{algo}
% Some more macros
\usepackage{sigmastyle}
\usepackage{amssymb}
\usepackage{subcaption}
\usepackage{wrapfig}
\usepackage{tcolorbox}

\usepackage{forest}
\forestset{
  default preamble={
    for tree={
        grow=south,
        circle, draw, minimum size=3ex, inner sep=1pt,
        s sep=3mm,
    }
  }
}
\usepackage{adjustbox}

% Quotes are fun, find some to use!
\font\eightss=cmssq8
\font\eightssi=cmssqi8
\newcommand\quoteAuthor[2]{\begingroup
  \baselineskip 10pt
  \parfillskip 0pt
  \interlinepenalty 10000 % not needed in example
  \leftskip 0pt plus 40pc minus \parindent
  \let\rm=\eightss
  \let\sl=\eightssi
  \everypar{\sl}#1\par
  \nobreak\smallskip
  \noindent\rm--- #2\unskip\par
  \endgroup}
\newcommand\quoteAuthorDate[3]{\begingroup
  \baselineskip 10pt
  \parfillskip 0pt
  \interlinepenalty 10000 % not needed in example
  \leftskip 0pt plus 40pc minus \parindent
  \let\rm=\eightss
  \let\sl=\eightssi
  \everypar{\sl}#1\par
  \nobreak\smallskip
  \noindent\rm--- #2\unskip\enspace(#3)\par
  \endgroup}

\newcommand{\TT}{\mathcal{T}}

% Set a title
\title{Factoring Integers}

% Set a subtitle if you desire
% \subtitle{[TAOCP 5 8.9.10.11]}

% Whoever worked on the presentation:
\author{Franklin}

% Date looks ugly, so leave blank
\date{}

% An institute name, if you're so inclined
% \institute{University of Illinois Urbana-Champaign}

% Use the SIGma theme for this Beamer presentation
\usetheme{sigma}
% --------------------------------------------------------------------

% Begin document
\begin{document}

% Beamer calls each slide a "frame", defined within the environment:
% \begin{frame}
%   <frame content here>
% \end{frame}

% This frame is just the title.
\begin{frame}
\titlepage
\end{frame}

% A frame with the table of contents.
% This frame's title is "Outline".
\begin{frame}{Outline}
  \tableofcontents
\end{frame}

% \begin{frame}{Updates!}
%   % Let's put some real content in this frame:
%   Weekly updates:
%   \begin{itemize}
%     \item SIGma is an excellent SIG.
%     \item I'm out of ideas for updates.
%   \end{itemize}
% \end{frame}

% Start a section: *sections* (subsections, etc.) are what show up in the TOC.
\section{Introduction}
% Section pages can be printed thus:
% \frame{\sectionpage}
% There's a way to automate this, see:
% https://tex.stackexchange.com/questions/178800/creating-sections-each-with-title-pages-in-beamers-slides/178803

\begin{frame}{Introduction}
  
  \begin{tcolorbox}[title=Problem,colbacktitle=sigma@mainblue]
    
  Given a positive integer $n$, find a nontrivial factor.
  \end{tcolorbox}

  A \textit{nontrivial factor} is a factor other than $1$ or $n$. \pause
  
  If the input size is the number of bits $b$,
  there is no known algorithm that runs in polynomial time. \pause

  For simplicity, we'll often assume that $n$ is composite.
  \begin{itemize}
    \item Checking whether $n$ is prime is easier, and can be done in polynomial time
    (AKS primality test).
  \end{itemize} \pause

  Important for RSA encryption, whose security depends on the
  difficulty of factoring the product of two primes (a \textit{semiprime}).

\end{frame}

\begin{frame}{Trial Division}
  The brute force approach: Simply try all integers less than $n$, checking if each is a factor.

  We only need to check up to $\sqrt{n}$;
  if we haven't found a factor by then, $n$ is prime. \pause

  Unfortunately, $O(\sqrt{n}) = O(\sqrt{2^b}) = O(2^{b/2})$, so the time complexity
  is still exponential in the number of bits. \pause

  Considering that there's no known polynomial algorithm, this is not all that bad.
  But can we do better?
\end{frame}

\begin{frame}{Fermat's Algorithm}
  Suppose $n = a \cdot b$. Let's assume $n$ is odd, so $a$ and $b$ are odd. \pause

  Then write $a = x+y$, $b = x-y$ for some integers $x, y$, so
  $$n = ab = x^2 - y^2 \implies x^2 - n = y^2.$$ \pause
  Now try values for $x$, starting from $x = \ceil{\sqrt{n}}$.
  For each $x$, compute $x^2 - n$, and see if the result is a perfect square.

  % Taking mod $n$,
  % $$x^2 - y^2 \equiv 0 \pmod{n}
  % \implies x^2 \equiv y^2 \pmod{n}.$$
\end{frame}

\begin{frame}{Example}
  Consider $n = 1081$. Then $\ceil{\sqrt{1081}} = 33$, so we try: \pause
  \begin{center}
  \begin{tabular}{c|c|c}
    $x$ & $x^2$ & $y^2 = x^2 - n$ \\
    \hline
    33 & 1089 & 8 \\
    34 & 1156 & 75 \\
    35 & 1225 & 144 \\
  \end{tabular}
  \end{center} \pause
  We find $144$ is a perfect square, so $x = 35, y = 12$ works. \pause

  Then we can recover $a, b$:
  $$a = x + y = 47, \quad b = x - y = 23.$$ \pause
  As expected, $47 \cdot 23 = 1081$.
\end{frame}

\begin{frame}{Fermat's Algorithm: Analysis}
  If a factor is near $\sqrt{n}$, the algorithm is fast -- $x$ will not be far from $\sqrt{n}$.
  \pause

  But if the factors are far apart (or worse, if $n$ is prime), we could be looking
  from $\sqrt{n}$ to near $n/2$.
  In the worst case, that takes $O(n)$ time -- even worse than trial division!
  \pause

  We can make some optimizations to skip $x$ that won't work,
  guess approximately where the primes are and start looking for $x$
  around there, or use Fermat's algorithm to narrow the search space
  for trial division.
  \pause

  But there's much better.
\end{frame}

\section{Dixon's Algorithm}
\frame{\sectionpage}

\begin{frame}{Modulo}
  Let's loosen our conditions a bit, and instead look for $x, y$ such that
  $x^2 - y^2$ is a multiple of $n$. In other words,
  $$x^2 - y^2 \equiv 0 \pmod{n} \implies x^2 \equiv y^2 \pmod{n}.$$
  \pause
  We can try to just look for pairs $(x, y)$ satisfying this,
  but there are a lot of different possible values of $x^2$ mod $n$
  (these remainders are called \textit{quadratic residues}). Specifically,
  if $n$ is the product of just two primes, there are around $n/4$
  quadratic residues mod $n$.
  \pause

  We can also try various $x$, reduce $x^2$ to its remainder mod $n$,
  and see if this result ($<n$) is a perfect square,
  like in Fermat's algorithm.
  Unfortunately, there are only $\sqrt{n}$ perfect squares less than $n$,
  so it will take a long time to land on one.
  \pause
\end{frame}

\begin{frame}{Combinations}
  Landing the quadratic residue $x^2$ mod $n$ on a perfect square is quite hard. But we don't need to.
  \pause

  Consider $n = 84923$.
  Then, here are some $x$ we might try at some point, with the remainders
  factorized into primes: 
  \begin{align*}
    513^2 \equiv 8400 &= 2^4 \cdot 3 \cdot 5^2 \cdot 7 \pmod{84923} \\
    537^2 \equiv 33600 &= 2^6 \cdot 3 \cdot 5^2 \cdot 7 \pmod{84923}
  \end{align*} \pause
  Neither of the residues are perfect squares. However, when multiplied:
  \begin{align*}
    8400 \cdot 33600 = 2^{10} \cdot 3^2 \cdot 5^4 \cdot 7^2
    = (2^5 \cdot 3 \cdot 5^2 \cdot 7)^2 = 16800^2.
  \end{align*} \pause
  So we can multiply remainders to get a perfect square!
\end{frame}

\begin{frame}{Combinations}
  We now have $(513 \cdot 537)^2 \equiv 16800^2 \pmod{84923}$.
  To recover the original primes, note that
  $513 \cdot 537 = 275481 \equiv 20712 \pmod{84923}$, so \pause
  \begin{align*}
    (513 \cdot 537)^2 - 16800^2 &\equiv 20712^2 - 16800^2 \\
    &\equiv (20712 - 16800)(20712 + 16800) \\
    &\equiv 0 \pmod{84923}.
  \end{align*}
  So some divisor of $20712-16800$ is a factor of $84923$,
  and same for $20712+16800$.
  \pause

  Our factors are then
  \begin{align*}
    \gcd(20712-16800, 84923) = 163, \quad
    \gcd(20712+16800, 84923) = 521.
  \end{align*}
\end{frame}

\begin{frame}{Taking this Further}
  The key idea is that we don't need to hit a quadratic residue that's a perfect square --
  instead, we can find several residues whose product is a perfect square.
  \pause

  We could start just squaring numbers above $\sqrt{n}$ and tracking the quadratic residues... \pause
  \begin{itemize}
    \item 
    ...but it's unrealistic to systematically check all the combinations of the residues.
  \end{itemize}
\end{frame}

\begin{frame}{Prime Factorization}
  Let's return to our earlier example of $n = 84923$. We had:
  \begin{align*}
    513^2 \equiv 8400 &= 2^4 \cdot 3 \cdot 5^2 \cdot 7 \pmod{84923} \\
    537^2 \equiv 33600 &= 2^6 \cdot 3 \cdot 5^2 \cdot 7 \pmod{84923}
  \end{align*} \pause
  From the prime factorizations, we can tell the product of these residues will give a
  square because the resulting exponents will be even. \pause

  So we can try:
  \begin{enumerate}
    \item Square a bunch of numbers to get some residues.
    \item Find the prime factorization of the residues. \pause
    \item Find some way to multiply them so that the sum of the exponents
    of each prime in the result is even.
  \end{enumerate}
\end{frame}

\begin{frame}{Factor Base}
  Unfortunately, factoring all the residues and then dealing
  with all the different primes that can appear in the factorizations is too much. \pause

  Instead:
  \begin{enumerate}
    \item Choose a few small primes up to some bound $B$ -- a \textbf{factor base}. \pause
    \item Square a bunch of numbers to get some residues.
    \item Attempt to factor the residues over the factor base,
    throwing out any that contain other prime factors.
  \end{enumerate} \pause
  The prime factors of the remaining residues are less than $B$ -- we call these residues \textbf{$B$-smooth}.
  We can then try to combine the residues so that the resulting exponents in the prime factorization are even,
  which would give us a perfect square.
\end{frame}

\begin{frame}{Example}
  Suppose $n = 4183$. We'll choose $B = 11$, so our factor base is $\{2, 3, 5, 7, 11\}$.
  Trying some random numbers not less than $\ceil{\sqrt{n}} = 65$,
  generate 6 $B$-smooth numbers: \pause

  \begin{minipage}[c]{0.8 \textwidth}
  \begin{tabular}{c|c|c|c}
    $x$ & $x^2$ & $y = x^2 \pmod{n}$ & Prime factorization \\
    \hline
    65 & 4225 & 42 & $2 \cdot 3 \cdot 7$ \\
    82 & 6724 & 2541 & $3 \cdot 7 \cdot 11$ \\
    92 & 8464 & 98 & $2 \cdot 7^2$ \\
    104 & 10816 & 2450 & $2 \cdot 5^2 \cdot 7^2$ \\
    113 & 12769 & 220 & $2^2 \cdot 5 \cdot 11$ \\
    118 & 13924 & 1375 & $5^3 \cdot 11$ \\
  \end{tabular} \pause
  \end{minipage}
  \begin{minipage}[c]{0.1 \textwidth}
  \[\begin{pmatrix}
    1 \\ 0 \\ 2 \\ 2 \\ 0
  \end{pmatrix}\]
  \end{minipage}
  
  Write the prime factorizations as column vectors, storing the exponent
  for each prime in the factor base. The vector above corresponds to $y = 2450$.
\end{frame}

\begin{frame}{Example}
  Multiplying residues is equivalent to adding the exponents in their prime factorizations.
  So given the 6 column vectors,
  \[
  \begin{pmatrix} 1 \\ 1 \\ 0 \\ 1 \\ 0 \end{pmatrix} \quad
  \begin{pmatrix} 0 \\ 1 \\ 0 \\ 1 \\ 1 \end{pmatrix} \quad
  \begin{pmatrix} 1 \\ 0 \\ 2 \\ 0 \\ 0 \end{pmatrix} \quad
  \begin{pmatrix} 1 \\ 0 \\ 2 \\ 2 \\ 0 \end{pmatrix} \quad
  \begin{pmatrix} 2 \\ 1 \\ 0 \\ 0 \\ 1 \end{pmatrix} \quad
  \begin{pmatrix} 0 \\ 0 \\ 3 \\ 0 \\ 1 \end{pmatrix}
  \] \pause
  our goal is to sum some of them to get a vector where every entry is even
  -- in other words, $0 \pmod{2}$.
\end{frame}

\begin{frame}{Linear System}
  Reduce the entries of the vectors mod 2.
  Then, we want to find $x_1, \dots, x_6 \in \{0, 1\}$ (not all zero) such that, mod 2,
  \[
  x_1 \begin{pmatrix} 1 \\ 1 \\ 0 \\ 1 \\ 0 \end{pmatrix} +
  x_2 \begin{pmatrix} 0 \\ 1 \\ 0 \\ 1 \\ 1 \end{pmatrix} +
  x_3 \begin{pmatrix} 1 \\ 0 \\ 0 \\ 0 \\ 0 \end{pmatrix} +
  x_4 \begin{pmatrix} 1 \\ 0 \\ 0 \\ 0 \\ 0 \end{pmatrix} +
  x_5 \begin{pmatrix} 0 \\ 1 \\ 0 \\ 0 \\ 1 \end{pmatrix} +
  x_6 \begin{pmatrix} 0 \\ 0 \\ 1 \\ 0 \\ 1 \end{pmatrix}
  \equiv \begin{pmatrix} 0 \\ 0 \\ 0 \\ 0 \\ 0 \end{pmatrix}.
  \] \pause
  From linear algebra, this always has a solution. There are 5 equations and 6 variables --
  the vectors are \textit{linearly dependent}, meaning there is some solution
  $(x_1, x_2, x_3, x_4, x_5, x_6) \ne (0, 0, 0, 0, 0, 0)$.
  \pause

  In this case, $(0, 0, 1, 1, 0, 0)$ works.
\end{frame}

\begin{frame}{Combining the Residues}
  Our solution $(0, 0, 1, 1, 0, 0)$ corresponds to the 3rd and 4th rows of our table:

  %\begin{minipage}[c]{0.8 \textwidth}
  \begin{tabular}{c|c|c|c}
    $x$ & $x^2$ & $y = x^2 \pmod{n}$ & Prime factorization \\
    \hline
    92 & 8464 & 98 & $2 \cdot 7^2$ \\
    104 & 10816 & 2450 & $2 \cdot 5^2 \cdot 7^2$ \\
  \end{tabular} \pause
  %\end{minipage}

  Let's multiply these.
  We have $92 \cdot 104 \equiv 1202 \pmod{4183}$, and
  %Then we multiply these (recall $n = 4183$ is the number to be factored):
  \begin{align*}
  92^2 \cdot 104^2 \equiv 98 \cdot 2450
  \equiv (2 \cdot 5 \cdot 7^2)^2 \equiv 490^2 \pmod{4183} \\
    0 \equiv (92 \cdot 104)^2 - 490^2
    \equiv (1202-490)(1202+490) \pmod{4183}
  \end{align*}
  Our factors are then
  $$a = \gcd(1202-490, 4183) = 89,\quad b = \gcd(1202+490, 4183) = 41.$$
\end{frame}

\begin{frame}{Dixon's Algorithm}
  Given an input $n$:
  \begin{enumerate}
    \item Choose a factor base of size $k$, containing all primes up to some bound $B$. \pause
    \item Generate residues (at least $k+1$) that factor over the factor base.
    \begin{itemize}
      \item Repeatedly choose (random) numbers between $\sqrt{n}$ and $n$ and square them.
    \end{itemize} \pause
    \item From the prime factorizations, use the exponents
    to create a system of linear equations mod 2.
    \item Find a solution to the system. (e.g., with Gaussian elimination)
    \item Multiply the corresponding residues and recover the factors of $n$.
  \end{enumerate}
\end{frame}

\begin{frame}{Analysis}
  Dixon's algorithm is dominated mainly by:
  \begin{itemize}
    \item Generation of residues (repeated squaring), most of which will be discarded
    due to not factoring over the factor base
    \item Attempting to factor the residues over the factor base
  \end{itemize} \pause
  Choosing the correct upper bound $B$ is important -- it turns out to be about
  \[\exp\left(\sqrt{\log n} \sqrt{\log \log n} / 2\right).\]
  \pause
  The (expected) runtime of Dixon's algorithm ends up as roughly
  $$O\left(\exp\left(2\sqrt{2} \sqrt{\log n}\sqrt{\log \log n}\right)\right).$$ \pause
  This is actually sub-exponential in $b = \log n$, thanks to the square roots.
\end{frame}

\section{Quadratic Sieve}
\frame{\sectionpage}

\begin{frame}{Improving on Dixon's}
  Ideas:
  \begin{enumerate}
    \item When generating residues, square values near $\sqrt{n}$.
    This way, the residue will be small:
    \[r(x) = x^2 - n \ll n.\] \pause
    \item Choose the primes in the factor base wisely.
    In particular, choose $p$ such that
    $$r(x) = x^2 - n \equiv 0 \pmod{p}$$
    has solutions. We'll see why shortly.
  \end{enumerate}
\end{frame}

\begin{frame}{The Sieve}
  Suppose we have one solution $x$ to $x^2 - n \equiv 0 \pmod{p}$. \pause
  Then note
  \begin{align*}
    r(x + kp) &\equiv (x+kp)^2 - n \\
    &\equiv x^2 + 2kpx + (kp)^2 - n \\
    &\equiv x^2 - n \\
    &\equiv 0 \pmod{p}.
  \end{align*} \pause
  Therefore, if $x$ is a solution, $x+kp, x+2kp, \dots$ are solutions.
  Each of these give residues that are multiples of $p$. \pause

  So by solving $x^2 - n \equiv 0 \pmod{p}$,
  we get many $x$ that give residues that are multiples of $p$ at once!

  %(This holds as long as the residue is actually $x^2 - n < n$.
  %As long as $x$ doesn't stray too far from $\sqrt{n}$, this is true.)
\end{frame}

\begin{frame}{The Sieve}
  Let $x_0 = \ceil{\sqrt{n}}$. Say we generate an array of residues of some size:
  \[A = 
  \begin{bmatrix}
    r(x_0) & r(x_0+1) & r(x_0+2) & r(x_0+3) & r(x_0+4) & r(x_0+5) & \cdots
  \end{bmatrix}
  \] \pause
  % Now take the small prime $p$ (starting with $p = 2$).
  Now consider $p = 2$.
  Find the solution of $r(x) = x^2 - n \equiv 0 \pmod{2}$ closest to $\ceil{\sqrt{n}}$.
  Let's say it's $x_0$ (it's either that or $x_0 + 1$).
  Then we know 2 divides all of the following:
  \[r(x_0), r(x_0+2), r(x_0+4), \dots\] \pause
  So we can update our array as follows:
  \[A = 
  \begin{bmatrix}
    \frac{r(x_0)}{2} & r(x_0+1) & \frac{r(x_0+2)}{2} & r(x_0+3) & \frac{r(x_0+4)}{2} & r(x_0+5) & \cdots
  \end{bmatrix}
  \]
\end{frame}

\begin{frame}{The Sieve}
  Next, repeat for other primes $p$:
  \begin{itemize}
    \item Solve $x^2 - n \equiv 0 \pmod{p}$ to get a sequence of solutions
    $x, x+p, x+2p, x+3p, \dots$.
    \item Divide the corresponding entries in the array $A$ by $p$.
  \end{itemize} \pause
  This is the "sieve" in the quadratic sieve -- since we divide
  every 2nd element by $p$, then every 3rd element, then every 5th element, etc.
  (reminiscent of the Sieve of Eratosthenes). \pause

  Once an entry in $A$ reaches 1, that means the
  corresponding residue fully factors over all primes $p$ used in the sieving. \pause

  A solution to $x^2 - n \equiv 0 \pmod{p}$ doesn't actually exist for all $p$,
  so we'll have to skip some primes $p$.
  The primes that we keep constitute our factor base.
\end{frame}

\begin{frame}{Combining Residues}
  Once we have enough fully-factoring residues (likely one more than
  the size of the factor base), we can create the system of equations
  as in Dixon's algorithm to find the combination of residues
  that will result in a perfect square. \pause

  Then we can apply difference of squares to get the factors of $n$, as seen earlier.
\end{frame}

\begin{frame}{Quadratic Sieve Algorithm}
  Given an input $n$:
  \begin{enumerate}
    \item Initialize a large array $A$ to hold information about
    residues for $x$ near $\sqrt{n}$. \pause
    \item Generate the factor base and sieve: \pause
    \begin{itemize}
      \item Solve $x^2 - n \equiv 0 \pmod{p}$ for increasing values of $p$.
      \item Add the $p$ to the factor base.
      \item Update the entries in $A$ corresponding to the solution.
    \end{itemize} \pause
    \item From the prime factorizations, use the exponents
    to create a system of linear equations mod 2.
    \item Find a solution to the system. (e.g., with Gaussian elimination)
    \item Multiply the corresponding residues and recover the factors of $n$.
  \end{enumerate}
\end{frame}

\begin{frame}{Analysis}
  With proper selection of the size of the factor base
  (too small and few residues factor over it; too large and we'll need a lot
  of residues to solve the system of equations),
  the expected runtime of the quadratic sieve is
  $$O\left(\exp\left(\sqrt{9/8} \sqrt{\log n}\sqrt{\log \log n}\right)\right).$$ \pause
  In particular, $\sqrt{9/8}$ is a smaller value than what appeared there in Dixon's algorithm.
\end{frame}

\begin{frame}{In Practice}
  Currently, the quadratic sieve is the second-fastest known factoring algorithm,
  beaten only by the general number field sieve.
  For numbers under $\approx 100$ digits, the quadratic sieve
  is still the fastest. \pause

  The quadratic sieve was the first to factor
  RSA-129, a 129-digit semiprime (in 1994). The factor base used 524339 primes.
  The data collection (sieving) took over 5000 MIPS-years, distributed over 1600 computers.
  The data processing (solving the system) took another 45 hours on a supercomputer. \pause
  \begin{itemize}
    \item We've come a long way since:

    \quoteAuthor{In 2015, RSA-129 was factored in about one day,
    with the CADO-NFS open source implementation of number field sieve,
    using a commercial cloud computing service for about \$30.}{Wikipedia}
    % \begin{quote}
    %   In 2015, RSA-129 was factored in about one day,
    %   with the CADO-NFS open source implementation of number field sieve,
    %   using a commercial cloud computing service for about \$30.
    % \end{quote}
  \end{itemize}
\end{frame}

% \begin{frame}{Some Text}
%     \begin{itemixe}
%         \item You may want some stuff to appear in a sequence \pause
%         \item Use \textbackslash pause for this \pause
%         \item \textcolor{sigma@mainblue}{colors} \textcolor{sigma@highlightpink}{are} \textcolor{sigma@alertred}{cool}
%     \end{itemize}
% \end{frame}

% \begin{frame}
%   \frametitle{Some Math Mode Testing}
%   % Some fun with LaTeX Math
%   $$\frac{x^2+3}{y^2+7}$$

%   \[
%     \mathcal L_{\mathcal T}(\vec{\lambda})
%     = \sum_{(\mathbf{x},\mathbf{s})\in \mathcal T}
%        \log P(\mathbf{s}\mid\mathbf{x}) - \sum_{i=1}^m
%        \frac{\lambda_i^2}{2\sigma^2}
%   \]

%   $$\int_0^8 f(x) dx$$
% \end{frame}

% % Use \pause to make stuff readable
% % Large walls of text scare the audience, we don't want that
% % Introducing stuff sequentially allows for questions
% \begin{frame}
%   % Alternate syntax for frame titles
%   \frametitle{There Is No Largest Prime Number}
%   % Frames can have subtitles:
%   \framesubtitle{The proof uses \textit{reductio ad absurdum}.}
%   % Some frame content:
%   \begin{thrm}
%     There is no largest prime number.
%   \end{thrm}
%   \begin{pf}
%     \begin{enumerate}
%         \item Suppose $p$ were the largest prime number \pause
%         \item Let $q$ be the product of the first $p$ primes \pause
%         \item Then $q+1$ is not divisible by any of them \pause
%         \item But $q + 1$ is greater than $1$, thus divisible by some prime number not in the first $p$ numbers. \pause
%         \item Thus, there exists a prime larger than $p$.
%     \end{enumerate}
%   \end{pf}
% \end{frame}

% % However, this doesn't work in math mode. It is quite annoying to figure out
% % So just copy this as reference
% % This works for \onslide<> and \item<>
% % Really good read on this: 
% %   https://www.texdev.net/2014/01/17/the-beamer-slide-overlay-concept/
% \begin{frame}{Sequential Math Frames}
%     Here is a sentence \pause
    
%     I shall now carry out some calculations \pause
%     \begin{align*}
%         \onslide<+->{\zeta(s) &= \sum_{n = 1}^\infty \frac{1}{n^s} \\}
%         \onslide<+->{&= \prod_{p \in \text{primes}} \frac{1}{1 - p^{-s}} \\}
%         \onslide<.->{&= \frac{1}{1 - 2^-s} \cdot \frac{1}{1 - 3^-s} \cdots \\}
%         \onslide<+->{&= \frac{1}{\Gamma(s)} \int_0^\infty \frac{x^{s - 1}}{e^x - 1} ~\textrm{d}x\\}
%     \end{align*}
% \end{frame}

% \section{Some Template Slides}
% \frame{\sectionpage}

% % Similar for subsections:
% \subsection{A subsection, Wow}
% % And their pages:
% \frame{\subsectionpage}

% \begin{frame}{Image}
%   % This is how you'd include an image, centered.
%   \begin{center}
%     \includegraphics[width=0.25\textwidth]{sigma.png}
%   \end{center}
% \end{frame}

% \begin{frame}{Side by Side}
%     \includegraphics[width=0.25\textwidth]{sigma.png}\hspace{0.4\textwidth}
%     \includegraphics[width=0.25\textwidth]{sigma.png}
% \end{frame}

% \begin{frame}{Demonstration of algo and nalgo env}
%     \begin{algo}
%     \underline{\textsc{GetRandomNumber}():}\+
%     \\      return $4$   \Comment{chosen by fair dice roll.}
%     \\      \hspace{42.75pt}\Comment{guaranteed to be random.}
%     \end{algo}
    
%     % nalgo has line numbers
%     % only lines with \label{} are numbered
%     \begin{nalgo}[1.3]
%         \underline{\textsc{GetRandomNumber}():}\+
%     \\\label{}  return $4$   \Comment{chosen by fair dice roll.}
%     \\\label{}  \hspace{42.75pt}\Comment{guaranteed to be random.}
%     \end{nalgo}
    
%     Random number generation from~\cite{site:xkcd}. 
%     \quest{
%     \textbackslash cref for line numbers does not work. 
%     If you want to refer to specific line numbers, do it manually
%     }
    
% \end{frame}

% \begin{frame}{}
% \begin{minipage}[c]{0.6\textwidth}
% \begin{nalgo}
% \textul{\textbf{\textsc{AlgorithmP}}$(S[a_0, \dots, a_n])$}:
% \\\label{}  $C[1..n] \gets 0, O[1..n] \gets 1$
% \\\label{}  \textbf{while} \textsc{True}:\+
% \\\label{}      \textsc{print}$(S)$
% \\\label{}      {\color{lightgray}$j \gets n, s \gets 0$}
% \\\label{}      {\color{lightgray}\textbf{A:}~~$q \gets C[j] + O[j]$\+}
% \\\label{}          {\color{lightgray}\textbf{if} $q < 0$: goto \textbf{D}}
% \\\label{}          {\color{lightgray}\textbf{if} $q = j$: goto \textbf{B}}
% \\\label{}          {\color{lightgray}\textsc{swap}$(S, j-C[j]+s, j-q+s)$}
% \\\label{}          {\color{lightgray}$C[j] \gets q$} 
% \\\label{}          {\color{lightgray}continue\-}
% \\\label{}      {\color{lightgray}\textbf{B:}~~\textbf{if} $j=1$:\+\+}
% \\\label{}              {\color{lightgray}break\-}
% \\\label{}          {\color{lightgray}$s \gets s+1$\-}
% \\\label{}    {\color{lightgray}\textbf{D:}~~$O[j] \gets -O[j], j \gets j-1$\+}
% \\\label{}      {\color{lightgray}goto \textbf{A}}
% \end{nalgo}
% \end{minipage}
% \begin{minipage}[c]{0.35\textwidth}
% \begin{itemize}
%     \item Here is an example of annotating an algorithms \pause
%     \item We have grayed out text to highlight what we want to discuss 
% \end{itemize}
% \end{minipage}
% \end{frame}

% % Of course, not everything is in pseudocode
% % You MUST have this [containsverbatim] option
% % https://www.overleaf.com/learn/latex/Code_Highlighting_with_minted
% \begin{frame}[containsverbatim]{Source Code}
%     \begin{minted}
%     [
%     framesep=2mm,
%     baselinestretch=1.2,
%     bgcolor=black,
%     fontsize=\footnotesize,
%     ]
%     {python}
%     def algorithm_g(n):
%         a = [0 for _ in range(n + 1)]
%         while True:
%             curr = a[1 : n + 1][::-1]
%             yield "".join([str(i) for i in curr])
    
%             a[0] = 1 - a[0]
%             j = 1
%             while a[j - 1] != 1:
%                 j += 1
%             if j == n + 1:
%                 return
%             a[j] = 1 - a[j]
%     \end{minted}
% \end{frame}


% \begin{frame}{Theorems and Lemmas}
%     \begin{lem}
%         The map $\C \times \pqty{\C \setminus \set{0}} \to \C$, $(z, w) \mapsto z / w$ is $C^{\infty}$.
%     \end{lem}
%     \begin{pf}
%         To see this, we identify $\C$ with $\R^{2}$ where $a + bi = (a, b)$.
%         Thus our map is now
%         \begin{align*}
%             \pqty{\R^{2}} \times \pqty{\R^{2} \setminus \set{(0, 0)}} &\to \R^{2} \\
%                             ((a, b), (c, d)) &\mapsto \pqty{\frac{ac + bd}{c^{2} + d^{2}}, \frac{bc - ad}{c^{2} + d^{2}}}
%         \end{align*}
%         which is well defined since at least one of $c$ or $d$ is nonzero.
%         It is simple to verify that this map is equivalent to the original one.
%         Since this new map is $C^{\infty}$ in each component, we have that it is $C^{\infty}$ overall.
%     \end{pf}
% \end{frame}

% \section{Conclusion}
% \frame{\sectionpage}

% Asking questions is fun but we should answer some first
\begin{frame}{}
      \begin{center}
    {\color{sigma@mainblue} \LARGE Questions?}
  \end{center}
\end{frame}

% \begin{frame}{Questions!}
%     \begin{itemize}
%         \item How many zeros of $\zeta(s) = \sum\limits_{i = 1}^\infty \frac{1}{n^s} = \prod\limits_{p \text{ prime}} \frac{1}{1 - p^{-s}}$ have real part equal to $\frac{1}{2}$?
%         \item Find a closed form to the following recurrence:
%         $$
%             f(n) = \begin{cases} 
%                 f\pqty{\frac{n}{2}} & n \text{ even} \\
%                 f{3n + 1} & \text{otherwise}                
%             \end{cases}
%         $$
%     \end{itemize}
% \end{frame}

% If someone can figure out how to horizontally center this and make the text bigger that'd be cool
\begin{frame}{Brainteaser}
      Assume 100 zombies are walking on a straight line, all moving with the same speed. Some are moving towards left, and some towards right. If a collision occurs between two zombies, they both reverse their direction. Initially all zombies are standing at 1 unit intervals. For every zombie, you can see whether it moves left or right, can you predict the number of collisions?
\end{frame}

% Remove this slide if you came up with all the material yourself
\begin{frame}[allowframebreaks]{Bibliography}
    \tiny
    \bibliography{refs}
    \bibliographystyle{alpha}
    
    \nocite{fermatwiki,dixonwiki,qswiki,dixonqsblog,factoringlecture,qfssimple}
    
\end{frame}

% @misc{site:xkcd,
%   author={Randall Munroe},
%   title={RFC 1149.5 specifies 4 as the standard IEEE-vetted random number.},
%   howpublished={\url{https://xkcd.com/221/}},
%   year={2007},
%   note={Accessed: 01-09-2023},
% }
% Moved from refs.bib

\end{document}